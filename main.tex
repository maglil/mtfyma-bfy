\documentclass[a4paper, oneside, 12pt]{memoir}
% Use twoside for a version compatible with two-sided printing.

% Layout
\usepackage[margin=2cm]{geometry} 

% Languated
\usepackage[norsk]{babel}


\appendixfalse % \appendixtrue or \appendixfalse to toggle appendix

\begin{document}

\frontmatter
\tableofcontents
\settocdepth{chapter}

\part{Innledning}
% Rollen og strukturen til dette dokumentet.

\mainmatter

\part{Målsetning}
\label{part:goals}

Denne delen beskriver målsetningene med studieprogrammene MTFYMA, BFY og MSPHYS. En tydelig beskrivelse av målsetninger gjør de enklere å utvikle målrettede læringsaktiviteter.

I et studieprogram som strekker seg over flere år vil det være mange ulike læringsmål. For å gjøre det mulig å få oversikt, må læringsmålene organiseres i et hierarki. På det nederste nivået må læringsmålene være tilstrekkelig detaljerte og konkrete til at de enkelt kan brukes til å utforme læringsaktiviteter og vurdere om læringsmålet er oppnådd.

Denne delene er delt opp i følgende kapitler:

\begin{enumerate}
   \item \textbf{Samfunnsoppdrag} beskriver hvilket behov samfunn eller individ studieprogrammet imøtekommer. Dette vil på overordnet nivå diktere hvilke læringsmål som prioriteres.
   \item \textbf{Vision} beskriver på overordnet nivå hvordan studieprogrammet imøtekommer samfunnsoppdraget.
   \item \textbf{Læringsmål} beskriver hierarkiet av læringsmål for BFY og fellesdelen av MTFYMA.
   \item \textbf{MTFYMA studieretninger} beskriver ulike studieretninger (spesialiseringer/minors) på MTYFMA.
   \item \textbf{BFY spesialiseringer} beskriver ulike retninger man kan spesialisere seg på BFY.
   \item \textbf{MSPHYS} beskriver hierarkiet av læringsmål for MSPHYS.
\end{enumerate}

\chapter{Samfunnsoppdraget}
\label{c:mission}

\section{MTFYMA}
Utvikling av \emph{ny} teknologi vil ofte være basert på \emph{grunnleggende} forståelse av de underliggende prosesser og modeller. Kandidater fra MTFYMA imøtekommer dette behovet med et solid fundament i både fysikk og matematikk. Kandidatene har i tillegg en verktøykasse og personlige ferdigheter som gjør dem ettertraktet for teknologisk utvikling og verdiskapning i et bredt spekter av arbeidslivet.

\section{BFY}
Søken etter å forstå og beskrive verden rundt oss ligger i menneskets natur. Denne søken etter fundamental forståelse har lagt grunnlaget for mye av samfunnets teknologisk utvikling. Ofte er ikke anvendelsen i syne når innsikten utvikles. Kandidater fra BFY imøtekommer menneskets og samfunnets behov for grunnleggende forståelse og kunnskapsutvkling. De har en bred bakgrunn i fysikk og tilhørende fagområder som matematikk og informatikk. Kandidatene er godt kvalifisert for videre studier i fysikk eller andre teknologisk fag, samt å fylle teknologiske stillinger i næringslivet.

\section{MSPHYS}
Teknolgisk utvikling følger forskningsfronten. Kandidater fra MSPHYS har dyp forståelse og innsikt i et begrenset fagområde, nær forskningsfronten. Kunnskapen og ferdighetene kan brukes til avansert teknologisk utvikling eller videreutvikles til å til flytting av kunnskapsfronten i en PhD eller annen forskningskarriere.
\chapter{Visjon}
\label{c:vision}

\section{MTFYMA}
\vision{
Kandidater fra MTFYMA er velkjente og svært ettertraktet hos alle relevante arbeidsgivere for sine kompetansen innen fysikk og matematikk. 
Programmet er annerkjent for sin pedagogiske kvalitet.
}
\section{BFY}

\vision{BFY tiltrekker seg studenter som er interessert i å forstå naturens fundamentale virkemåte.}
\visiontext{
Dette er den primære kandidatprofilen programmet ønsker å tiltrekke seg. 
Sekundært kan det også være studenter som er litt usikre på hva de ønsker og hvor BFY kan være et godt springbrett for andre fagfelt.
}

\vision{BFY gir stor frihet til å følge egen interesse.}
\visiontext{Friheten balanseres mot basiskunnskap, det som alle fysikere \emph{må} kunne bør være obligatorisk. Likevel er studentenes individuell søken etter forståelse en viktig verdi for programmet og dermed skal studentens frithet til å følge egne interesser stå sterkt. Ingen vet hva fremtiden bringer og hvilken kunnskap som vil være viktig og verdifull. Hva som skal være obligatorisk må bestemmes av fagmiljø i samråde med andre interessenter (f.eks. faglige foreninger).}

\vision{BFY gir mulighet for dyp spesialisering innen et avgrenset område av fysikken som et grunnlag for videre fordypning.}
\visiontext{Ved å starte fordypning innen et avgrenset område kan kandidater fra BFY gjennomføre en mastergrad på høyt nivå, nær forskningsfronten. Studiet gir også erfaring med forskningsarbeid, f.eks. gjennom bachelorprosjekt, som også bidrar til å gjennomføre en mastergrad med høy kvalitet.}

\vision{BFY definerer tydelige studieløp for bestemte spesialiseringer}
\visiontext{Disse spesialiseringene er definert av valgfrie emner og er basert på IFY sine nåværende forskningsgrupper. Dette endres dynamisk ettersom faggrupper endrer seg. Emnepakker kan også planlegges sammen med andre enheter, enten på NTNU eller ved andre universitet.}

\vision{BFY bidrar til internasjonalisering av fagområdet og styrkning av nasjonal kompetanse.}
\visiontext{
Gjennom å ha studieprogram som er tilpasset strukturen i European higher education area, definert gjennom Bolognaprosessen, kan BFY-kandidater reise ut for å følge masterprogram, spesielt for å studere fagområder som det ikke eksisterer ekspertise på ved NTNU eller i Norge.
Dette er viktig for nasjonal kompetanseutvikling. 
Samtidig er programmet viktig for et velfungerende masterprogram som kan rekruttere internasjonalt og gi et faglig og kulturelt rikere studiemiljø.
}

\vision{BFY legger til rette for å ta mastergrad i andre fagområder og spesifiserer disse veiene.}
\visiontext{
Noen studenter vil kanskje oppdage at søken etter fundamental forståelse ikke var helt for dem, fant et annet spennende fagfelt eller ønske en mer anvendt utdannelse. 
Studieprogrammet prøver derfor å legge opp til at man kan ta emner som kvalifiserer for mastergrad i tilstøtende fagfelt. 
Disse muligheten spesifiseres og kommuniseres tydelig til studentene
}

\section{MSPHYS}

\vision{studenter ved MSPHYS fordyper seg i et gitt område, ofte nær forskningsfronten.}

\vision{MSPHYS gir en god forskerutdannelse, både mot akademi og instituttsektor}

\vision{MSPHYS gir mulighet til dyp spesialisering for å kunne bidra med kunnskap nær forskningsfronten for arbeidslivet}

\vision{MSPHYS definerer tydelige spesialiseringsløp}





\chapter{Læringsmål}
\label{c:los} % los for learning outcomes

Dette kapittelet beskriver hierarkiet av læringsmål (læringsutbytter). Dersom det ikke er spesielt angitt gjelder læringsmålet for både MTFYMA og BFY. For læringsmål som gjelder for kun det ene studieprogrammet er dette angitt med M/B for henoldsvis MTFYMA og BFY.

Overordnet inndeling av læringsmål

\begin{itemize}
	\item Matematikk
	\item Fysikk
	\item Beregningsorienterte ferdigheter
	\item Eksperimentelle ferdigheter
\end{itemize}

\section{Matematikk}
\label{s:math}
Matematikk


\section{Fysikk}
\label{s:physics}

Fysikk


\chapter{MTFYMA studieretninger}
\label{c:mtfyma-spec}

MTFYMA definerer et sett med spesialiseringer, bestemt av et sett med emner, som gir en spesialisering som angis på vitnemålet. Disse spesialiseringene vil være dynamiske over tid og bestemmes av fagmiljøets (og samarbeidende fagmiljøs) ekspertise og samfunnets behov. Det er ikke nødvendig å velge spesialisering og kan man ta en generell grad i fysikk og matematikk med emner fra begge fagområder.

\begin{itemize}
	\item Computational Science
	\item Data Science
	\item Biophysics and medical technology
	\item Quantum technology and cryptography
	\item Material physics
	\item Atmospheric physics and climate change
	\item Optics and photonics
\end{itemize}

\chapter{BFY spesialiseringer}
\label{c:bfy-spec}

BFY definerer et sett med spesialiseringer, bestemt av et sett med emner som danner et naturlig grunnlag for fortsettelse på en mastergrad. Disse spesialiseringene vil være dynamiske over tid og bestemmes av fagmiljøets ekspertise.

\begin{itemize}
	\item Astrophysics
	\item Fundamental physics
	\item Computational physics
\end{itemize}

I tillegg synliggjør programmet tydelige emnevalg som kvalifiserer for masterprogram i andre fagfelt. Nåvæerende løp som er identifisert er:

\begin{itemize}
	\item Master i matematikk?
	\item Master i informasjonsteknologi?
	\item ...
\end{itemize}
\chapter{MSPHYS}
\label{msphys}

Master i fysikk


% Overordnet målsetning.
%\section{Overordnet målsetning for programmene}

Denne delen beskriver de overordnede målsetningene med programmet. Hensikten med å beskrive målsetningene på dette nivået er at det gir et håndterbart sett med målsetninger som kan fungere som rettesnor og gi en felles forståelse av hva vi ønsker å oppnå med programmet.

En slik beskrivelse skal ikke være et minste felles multiplum som kan aksepteres enstemmig. Det vil være uenighet (og kontinuerlig diskusjon) om innholdet og dette må aksepteres. De overordnede målsetningene representerer fagmiljøets majoritet, eksterne aktører og føringer fra ledelse. 

Utsagn som ikke virker begrensende må innholdet eller som for de fleste innen fagmiljøet er selvfølgelige bør også unngås.

\subsection{MTFYMA}

%Brynjulf
\begin{itemize}
	\item MTFYMA skal være et attraktivt studieprogram som tiltrekker seg elever fra videregående skole som har en spesiell interesse og talent for fysikk og matematikk.
	% Hva ligger i å være attraktivt? Må man ha spesielt talent?
	\item MTFYMA skal ha klare kompetansemål for utdanningen som gjør våre kandidater attraktive for framtidige arbeidsgivere.
	\item MTFYMA skal være et integrert studieprogram med en egen tydelig identitet som i størst mulig grad er uavhengig av spesialisering i senere del av studiet.
	\item MTFYMA skal ha en tydelig profil innenfor fagområdene fysikk og matematikk som gjennom hele studieløpet kjennetegnes ved at det har en høyere målsetning innen disse fagene enn øvrige sivilingeniørprogrammer.
	\item MTFYMA skal være et umiskjennelig sivilingeniørprogram.
	\item MTFYMA skal utdanne kandidater som tilfører arbeidslivet kunnskaper om den til enhver tid nyeste metodikken som er tilgjengelig innen fysikk og matematikk for å arbeide med industrielle problemstillinger.
	\item MTFYMA vil etterstrebe et godt læringsmiljø for studentene, gode studieforhold, aktive undervisningsmetoder og ligge i front når det gjelder vurderingsformer og samstemt undervisning.
	\item MTFYMA skal ha et tett og godt forhold til industri og andre avtakere av våre studenter, lytte til deres ønsker og behov og være åpen for å modernisere og tilpasse studieprogrammet etter hva som er etterspurt
\end{itemize}

%Magnus
\begin{itemize}
	\item MTFYMA gir, I tillegg til det teoretiske grunnlaget, nødvendige ferdigheter i metoder og verktøy som gjør kandidatene til effektive problemløsere.
	\item MTFYMA skal gi studentene trening og trygghet i å anvende sammensatt kunnskap og ferdigheter for å løse komplekse problemstillinger.
	\item MTFYMA er et studieprogram som i stor grad er programmert for å effektivt oppnå ønsket kunnskap og ferdigheter.
	\item MTFYMA gir i tillegg til det faglige fundamentet også kunnskap og ferdigheter ikke-faglige ferdigheter som er viktige som ansatt i en bedrift.
\end{itemize}

\subsection{BFY}
BFY tilstrekkelig valgfrihet mot slutten av studiet til å kunne begynne spesialisering mot ulike fagfelt både innen og utenfor fysikk, og sørge for at studenten kan kvalifisere seg til opptak mot disse studiene. Kombinasjonen BFY + MSPHYS gir mulighet for dypere spesialisering i fagområder innen fysikk, mens muligheten for kombinasjonen BFY + annet masterprogram, gir en bredde av kompetanseprofiler.


% Beskrivelse av egenarten til programmene.

%\chapter{Detaljerte kompetanser}
\label{chap:competencies}
Dette kapittelet beskriver kompetansen som studentene skal opparbeide. De ønskede kompetansene er svært omfattende så for å få oversikt må kompetansene systematiseres på ulike nivåer. De overordnede kompetanseområdene er \footnote{FTS beskriver en kompetanseprofil og et sett med 12 kompetansemål. Disse anses som for lite konkrete til å være effektive virkemiddel for å systematisere innholdet i programmet. Vi har her derfor heller omformulert disse slik at de er mer konkrete for fagområdene fysikk og matematikk. En egen oversikt vil vise hvordan anbefalingene fra FTS er fulgt opp.}

\section{Kompetanseområder}

\begin{itemize}
	\item Matematikk, teori.
	\item Fysikk, teori.
	\item Analytiske ferdigheter.
	\item Modellering.
	\item Numeriske metoder og programmering.
	\item Eksperimentelle ferdigheter.
	\item Kommunikasjon.
	\item Innhenting og vurdering av informasjon.
	\item Samarbeid.
	\item Planlegging og gjennomføring.
	\item Læring.
	\item Komplementære fagområder.
\end{itemize}


%\section{Digitale ferdigheter}

Ettersom det digitale ferdigheter er sammensatt av mange komponenter er nyttig å få systematiserte ferdighetene for å få oversikt og som et verktøy når man skal legge opp studiet. Her følger en overordnet oversikt.

\begin{itemize}
	\item Vitenskapelige ferdigheter
	\item Generelle IT-ferdigheter
	\item Generelle programutviklingsferdigheter
	\item Språk
	\item Algoritmer
	\item Programvare
	\item Praktiske ferdigheter
\end{itemize}

De neste delene gir en ytterligere detaljering av ferdighetene som skal 

\subsection{Vitenskapelige ferdigheter}
\begin{itemize}
	\item Omsette en fysisk modell til kode
	\item Redegjøre for antagelser, 
	\item Tolke numeriske beregninger, fysisk vurderering av realisme
	\item Hensiktsmessig rapportering og presentasjon
\end{itemize}

\subsection{Generelle IT ferdigheter}
\begin{itemize}
	\item Linux
	\item Kommandolinjeverkøy
	\item Packe managers (conda, pip)
	\item Software stacks
	\item Presisjon, numeriske feil, konvergens
	\item Pseudo-random numbers
	\item Dimensjonløse tall
	\item Hardware (minne, lagring, 
	\item Parallellisering
	\item HPC ressurser
	\item Prosessere data, ulike dataformater, uorganiserte, mangelfulle data.
	\item Søke i og lese teknisk litteratur.
	\item Effektiv visualisering
	\item (Datasikkerhet)
\end{itemize}

\subsection{Generelle programutviklingsferdigheter}
\begin{itemize}
	\item Generelle elementer i programmeringsspråk (variabler, kontrollstrukturer osv.). Dekkes av ITGK.
	\item IDE (vscode)
	\item Dokumentasjon av kode
	\item Versjonskontroll
	\item Enhetstesting, validering
	\item Debugging
	\item Style guides
	\item Strukturering av kode
	\item \enquote(Computational cycle)
\end{itemize}

\subsection{Språk}
\begin{itemize}
	\item High level language: Python (Numpy, Matplotlib, SciPy, Pandas, Seaborn, Scikit-learn, Tensorflow, Requests)
	\item Low level language: C++
	\item Symbolic computing: Python/sympy
	\item Markup: LaTeX, Markdown.
	\item (og/eller R, Matlab, C, Julia, Haskel, Fortran...)
\end{itemize}

\subsection{Algoritmer}
\begin{itemize}
	\item ODE...
\end{itemize}


\subsection{Programvare}
\begin{itemize}
	\item Office-pakken
	\item (Comsol multiphysics, Ansys, Zeemax optic studio, Openfoam, Molsim)
\end{itemize}


\subsection{Praktiske ferdigheter}
\begin{itemize}
	\item Samarbeid om kode
\end{itemize}

\subsection{Pedagogisk-stratgiske valg}

All aktivitet skal være så autentisk som mulig.

All koding gjøres i rene tekstfiler. Jupyter notebooks kan brukes i undervisningsformål og til å øve på enkle programmeringselementer.

Fungerende hardware og software er studentens ansvar. De får hjelp så langt ressursene, men om ikke studenten får maskinen til å fungere betyr det at det ikke består aktiviteten.

Alle ferdigheter som studenten ønskes å oppnå skal undervises og kobles til konkrete læringsaktiviteter.

Prosjekter skal bidra til motivasjon og kreativitet

%\section{Eksperimentelle ferdigheter}

\subsection{Læringsmål}

Eksperimentelle ferdigheter har to hovedmål:
\begin{enumerate}
	\item Det skal gjøre studenten i stand til å besvare teknologiske og vitenskapelige spørsmål gjennom eksperimenter. 
	\item  Det skal gjøre studenten i stand til å skape teknologiske løsninger som skaper verdi gjennom å manipulere energi, materie og informasjon. \footnote{Merk at \emph{teoretisk innsikt} ikke er et primært læringsmål. For at læringsaktivitetene skal ha et tydelig mål settes det skarpt organisatorisk skille mellom aktiviteter som skal gi teoretisk innsikt og aktiviteter som skal gi eksperimentelle ferdigheter, slik det er definert her. Praktiske aktiviteter kan likevel ha en verdi for å gi teoretisk innsikt, men slike aktiviteter organiseres i teori-emner. Eksperimentell trening vil også nødvendigvis gi teoretisk innsikt, men dette er en positiv konsekvens og ikke hovedmålsetningen.}
\end{enumerate}

For å oppnå disse overordnede målene må studentene utvikle følgende ferdigheter:

\begin{description}
	\item [Designe eksperiment] Dette inkluderer å konkretisere spørsmål til noe som kan besvares eksperimentelt, utforme hypotese, velge instrumentering, metrologi, kalibrere, teste, videreutvikle, HMS, RRI, m.m.
	\item [Modellere eksperimentet] Lage en matematisk representasjon av eksperimentet for å kunne tolke data, vurdere gyldighet m.m.
	\item [Gjennomføre eksperimentet] Lagre, dokumentere, metadata, FAIR m.m.
	\item [Analysere og tolke data] Filtrere, representere, kondensere, feilkilder, visualisere, sammenstille med modell, tolke resultatene og syntetisere det til nyttig kunnskap, m.m.
	\item [Kommunisere] Ulike format, åpen vitenskap, etisk.
\end{description}

Disse ferdighetene sammenfatter hovedmål 1). Hovedmåle 2) innebærer i stor grad samme prosesser men hvor designprosessen får større plass, man  designere mot spesifikasjoner istedenfor et spørsmål og at produktets kommersielle verdi får større vekt.

Noen sentrale pedaogisk-strategiske valg:

\begin{itemize}
	\item Alle læringsaktiviteter skal være så \emph{autentiske} som mulig. Eksempelvis så bør det benyttes instrumenter, elektronikk og metoder som er vanlige i forskning og teknologi (altså ikke instrumenter som er laget primært for pedagogiske formål). Studentene må være involvert i hele verdikjeden beskrevet nedenfor. Kommunikasjon i autentiske formater (ikke lab-rapport, masteroppgave).
	\item Ikke gruppearbeid men team-arbeid. Det er ikke effektivt eller vanlig at man i arbeidslivet har flere personer som gjør samme ting. Men man jobber ofte i team og fordeler arbeidsoppgaver.
	\item Studentenes evne til å gjennomføre hele verdikjeden må bygges gradvis. Dersom studenten ikke skal gjennomføre noen deler må det gjøres på en autentisk måte. Eksempelvis er det vanlig å \enquote{kjøpe tjenester} (for eksempel fra en kjernefasilitet) som kan inngå som en naturlig del i den eksprimentelle kjeden, men studentene må da være en tydelig \enquote{kunde} i prosessen. Et annet eksempel kan være å bruke åpent tilgjengelige data.
	\item For et gitt område, skal det benyttes felles læringsmateriell gjennom hele studieforløpet.
	\item Den teoretiske bakgrunnen for eksperimentet bør i så stor grad som mulig være kjent. Senere i studiene kan et større element av å måtte sette seg inn i nytt fagstoff før eksperimentet inkluderes, og at denne ferdigheten da er et eget læringsmål.
	\item Ettersom studenten selvstendig skal utføre elementene må man nødvendigvis starte med det som skal til for å kunne gjennomføre en måling. Så å si alle målinger i dag gjøres om til et elektrisk signal som lagres digitalt på en datamaskin. Det første studenten må lære for å kunne gjennomføre et eksperiment er å kunne sette opp en sensor, en transducer, en ADC og en datamaskin.
	\item Ingen av læringsmålene bør ha større eller mindre vekt på vurdering bare fordi de er enklere å vurdere (som eksempelvis en rapport har vært).
	\item Undervisningen deles grovt opp mellom 1) opplæring i de nødvendige elementene (e.g. spesifikke instrumenter, statistiske analyser, rapportformater) og 2) gjennomføring av eksperimenter.
	\item Eksperimenter som gjennomføres på datamaskiner organiseres i begynnelsen av studiene under digitale ferdigheter men dette skillet vil viskes ut senere i studiet hvor man typisk simuleringer og fysiske eksperimenter støtter opp om hverandre.
	\item Prosjektene blir gradvis mer komplekse. En mulig progresjon er: Individuelt arbeid i mindre prosjekter - team i mindre prosjekter - team i større prosjekter - prosjekter over flere studentkull/flere semestre - tverrfaglige prosjekter - industriprosjekter - internasjonale prosjekter.
\end{itemize}




\part{Gjennomføring}
\label{part:tla}

Denne delen beskriver \emph{hvordan} man skal oppnå de målsetninger man har satt seg i forrige del av dokumentet. Denne delen starter med å beskrive det læringsmiljøet man ønsker å etablere og dernest hvilke undervisnings- og vurderingsformer som bør brukes for å bygge dette læringsmiljøet.

	\chapter{Læringsmiljø}
	
	\chapter{Undervisningsformer}
	
	\chapter{Vurderingsformer}

\part{Oppbygging}
\label{part:program-structure}

Denne delen beskriver hvordan man hensiktsmessig kan bygge opp studiet for å oppnå de målsetninger som ble satt i første del, gjennom de læringsaktivitetene som ble beskrevet i andre del. Beskrivelsen starter med et overordnet nivå (emnevegg) og går til enkeltelementer i emnene.

\part{Strategiske valg}

\ifappendix
\appendix

\renewcommand{\appendixtocname}{Tillegg}
\renewcommand{\appendixpagename}{Tillegg}
\part*{Tillegg}
\addcontentsline{toc}{part}{Tillegg}
%\appendixpage*

\chapter{Utviklingsplan}

Denne delen inneholder konkrete planer for endringsprosjekter med mål om realisere målsetningene med studieprogrammet.

\section{Prosjektplan}

\section{Prosjekter}

	\subsection{FTS-revisjon}
Her følger en beskrivelse av overordnet plan for revisjon av studieprogrammet som følge av anbefalingene som er gitt av FTS-prosjektet. Revisjon av studieprogrammet som følge av dybdeevaluering av MTFYMA og BFY for 2020 inkluderes i samme prosjekt.

FTS-revisjon deles opp i fire faser. Prosjektplanen tar utgangspunkt i at en omfattende omlegging kan gi en signifkant forbedring av studieprogrammet. Etter de to første fasene vurderes det om det er tilstrekkelig behov for å gå videre med neste fase eller om en nedskalert revisjon er mer hensiktsmessig. Det kan være naturlig å starte noe arbeid med en senere fase selv om den foregående fasen ikke er ferdigstilt, men det bør ikke legges for mye ressurser i et slikt arbeid før man er sikker på at man ønsker gå videre med neste fase. Tentativ år/måned for avslutning av hver fase er angitt. Ettersom det er vanskelig å vurdere hvor lang tid hver fase vil ta vil disse justeres ettersom prosjektet skrider frem.

\begin{itemize}
	\item Fase 1: Gapsanalyse og visjon (2022/12). Identifisere gap mellom anbefalinger og ønsker om hvordan man ønsker at studieprogrammet skal være og nåværende situasjon. Gapsanalysen vil minimum omfatte FTS prinsipper og kompetansemål samt dybdeevaluering fra 2020. Dersom gapet anses å være tilstrekkelige går man over i Fase 2 for å utvikle og dokumentere ønskede kompetansemål
	\item Fase 2: Kompetansemål (2023/12). Utvikle ønskede kompetansemål fra overordnet til detaljert nivå. Samtidig beskrives hvordan studentene best kan oppnå denne kompetansen. Dersom kompetansemålene ikke enkelt kan realiseres i dagens programstruktur går man videre til Fase 3 for å utvikle en ny programstruktur.
	\item Fase 3: Programstruktur (2024/6). Det utformes en ny programstruktur for studieprogrammet.
	\item Fase 4. Læringsaktiviteter og -ressurser (2025/6). Disse utvikles før oppstart av nytt studieprogram.
\end{itemize}

%Høst Innhold- og strukturprosjekter: Overordnet struktur, beregning, eksperimentelt, læringsmiljø, vurderingsformer, undervisningsformer, mekanikk, elmag, matematikk, ønsker faggrupper.

%Revider prosjektplan. Større/mindre aktiviteter

%Ressursbehov er 20\% frikjøp i prosjekgruppen (50\% i tillegg til programledere) De andre komiteene leverer arbeider som ikke er for arbeidskrevende. 

%2023 

%Vår Overordnet struktur legges litt fastere, undervisningsformer.

%Ressursbehov. som over. Det er relativt usikkert hva ressursbehovet her er, ettersom dette er en relativt ny arbeidsmetode.

% Det kan være behov her for en gruppe som kan gå litt dypere, men det kan være vanskelig å få til på så kort varsel. Høsten 2022 bør det identifiseres evt. ressursbehov for høsten 2023. Må undersøke for raskt man kan få frigjort ressurser. 

%Høst

%Flere innholdsprosjekter

% Testing av formater underveis




	
	%\subsection{Gapsanalyse 2022}

Dette prosjektet er en del av Fase 1 av FTS revisjonen. Målet er å synliggjøre eventuelle gap mellom anbefalinger og nåværende situasjon.

Følgende ressurser vil brukes

\begin{itemize}
	\item FTS prinsippene
	\item FTS kompetanseprofiler
	\item Dybdeevaluering av MTFYMA og BFY 2022
\end{itemize}

Basert på gapsanalysen utvikles det en liste med nødvendige tiltak for å redusere gapet.

Gapsanalysen skal være ferdigstilt innen 2022/12
	%\subsection{Visjon for studieprogrammene 2022}

Som et utgangspunkt for videre diskusjon av innholdet i studieprogrammet kan det være hensiktsmessig å få skrevet ned et sett med overordnede mål for studieprogrammene (MTFYMA og BFY). Det kan være nyttig som et grunnlag for å styre videre diskusjoner og arbeid med et mer detaljert innhold i studieprogrammene. Det utvikles også overordnede visjoner for de eksisterende studieretningene.

Sammen med visjonene utformes også en beskrivelse av hva som er forskjeller og likheter mellom programmene.

Prosess:
\begin{itemize}
	\item Kjernegruppen utvikler et utkast til overordnede målsetninger. Dette legges frem på ledergrupper på IMF og IFY før sommeren 2022. 
	\item Tilbakemelding og innspill gis til kjernegruppen frem til 2022/10. 
	\item Kjernegruppen utformer en endelig versjon innen utgangen av 2022.
\end{itemize}




% Langtidsplan

% Forankring

% Hvilket behov ser faggruppene for kunnskap og ferdigheter som er nødvendige for å kunne ta fag som faggruppen er 'ansvarlig' for i høyere årskurs. Hvilke ferdigheter er nødvendige for å ta masteroppgaver? Er det tidligere observert noen generelle mangler?

% Hvordan bør dette dokumentet struktureres?

\section{Ressursbehov}

% Oppsummering av ressursbehov

\chapter{Gapsanalyser}
\label{chap:gap-analysis}

Dette kapittelet inneholder en sammenlikning av studieprogrammene i forhold til ulike rapporter som har analysert hva som vil være ettertraktet kompetanse hos fremtidige studenter(\ref{sec:fts}-\ref{sec:evaluation2020}). I tillegg er det gjort et forsøk på et sammendrag av tiltak basert på gapsanalysene i \ref{sec:changes}.

\section{Forslag til endringer}

Her følger en punktliste over foreslåtte endringer i studieprogrammene baserte på gapsanalysere i forhold til FTS, dybdeevalueringen (se senere deler i dette kapittelet) og oppsamlede erfaringer som er gjort med dagens studieprogram.

Tilsammen vil mengden av de punktene man mener er fornuftige på denne listen diktere om det er nødvendig med en omfattende omlegging eller mindre justeringer.

Listen er ikke prioritert.

\begin{itemize}
	\item Innholdet i studieprogrammet beskrives i et Hoveddokument.
	\item IMF og IFY må ha et mye tettere samarbeid om innholdet i emnene slik at man oppnår synergier og ikke bare suboptimalisering ved å ha et felles studieprogram.
	\item Studenter ved BFY og MTFYMA kan med fordel ha en dypere og bredere innføring i matematikk enn det som er felles for andre siv.ing. program.
	\item Det opprettes egne emner for å utvikle praktiske ferdigheter, spesielt innen eksperimentelt arbeid og beregninger, gjerne kombinert.
	\item Studentene må i større grad eksponeres for oppgaver som trener integrert kompetanse, eksemplevis som integrerer eksperimenter, simuleringer, prosjektstyring og fysikk fra flere områder.
	\item Det må være en helhetlig plan for opplæring og trening i numeriske metoder og algoritmer.
	\item Studentene må få opplæring og trening i verktøy som er viktige for effektivt å utføre beregningsorienterte oppgaver. Dette inkluderer: Versjonskontroll (git), package-managers (conda), enhetstesting (pytest), terminalvindu, linux, m.m
	\item Studentene må få god opplæring i verktøy som de forventes å benytte.
	\item Studentene må få trening i distribuerte utviklingsprosjekter.
	\item Studentene skal få opplæring og trening i å bruke HPC ressurser, parallellisering og optimalisering (kan muligens være valgbart). 
	\item Studentene skal få trening i metoder for maskinlæring. 
	\item Studentene skal få trening i å håndtere store, ustrukturerte datasett.
	\item Eksperimentelle ferdigheter skal struktureres rundt å kunne gjennomføre hele verdikjeden fra problemformulering til rapportering.
	\item Studentene skal få mye mer trening i rapportskriving enn i dag.
	\item Studentene må gis mulighet for å fordype seg i tema relatert til bærekraft. Studenter med fysikk og matematikk bør bygge på sin styrke i forståelse av fysisk modellering og evne til å forstå å bruke avansert matematiske metoder. Det må tydeliggjøres hvilken kompetanse relatert til bærekraft studentene bør utvikle.
	\item Det bør utformes tydelig emneløp som gir spesialiseringer innen ulike fagområder. Dette må kommuniseres godt til studentene.
	\item Det bør legges ressurser inn på obligatoriske emner som leder opp mot valg av spesialisering.
	\item Istedenfor EiT bør det innføres et emne hvor man tverrfaglig skal løse et gitt teknologisk problem.
	\item Innføring i samarbeid (ala EiT) bør komme i begynnelsen av studiene og studentene bør få trening i dette gjennom hele studiet.
	\item Innføring i teori bør i større grad gjøres i en kontekst slik at studentene tydelig ser verdien av teorien.
	\item Det antas at studenter ved MTFYMA og BFY har tatt fysikk 2. For de som ikke har tatt det eller trenger oppfriskning tilbys et oppfriskningskurs (digitalt eller fysisk).
	\item Emnene i første semester og år skal legge et spesielt fokus på å motivere studentene til videre studier og ha et bevist forhold til at å etablere et liv på universitetet kan være krevende for mange.
	\item Det skal være en bevist og dokumentert plan for når matematiske temaet introduseres i matematikkemnene og når det er behov for dem i fysikk emnene.
	\item Studieprogrammene må ha et tett samarbeid med industri for å kunne bruke dette til å kontekstualisere teorien samt å skaffe oppgaver som er mer sammensatte og åpne.
	\item Studieprogrammene og fagmiljø skal i fellesskap utvikle en felles forståelse for det læringsmiljøet man ønsker å etablere og de undervisnings- og vurderingsformer som gjør at man oppnår det.
	\item Studenten bør få god trening i ulike former for prosjektarbeid, bl.a. arbeidsdeling, ledelser, planlegging, risiko, samarbeid m.m.
\end{itemize}

\section{Fremtidens teknologistudium}
	\label{sec:fts}
	\subsection{FTS prinsippene}
\label{sec:fts-principles}
FTS har definert 10 prinsipper\footnote{\url{https://www.ntnu.no/fremtidensteknologistudier/prinsipper}} som NTNU har vedtatt skal være styrende for teknologi-utdanningene ved NTNU. Det gis her en kortfattet vurdering i hvilken grad MTFYMA og BFY oppfyller disse prinsippene. Dersom det eksisterer et gap mellom ønsket og nåværende situasjon gis det kortfattet forslag om hvilke tiltak som kan iverksettes for å redusere gapet\footnote{Denne gapsanlysen er et utkast og planlagt ferdigstilt i løpet av høsten 2022}.

\begin{quote}
	\textbf{Prinsipp I:} NTNUs teknologistudier skal legge aktivt til rette for at kandidatene, med utgangspunkt i et solid faglig fundament, opparbeider helhetlig og integrert kompetanse, herunder bærekraftkompetanse og digital kompetanse på høyt nivå.
\end{quote}

Studieprogrammet gir studentene et omfattende faglig fundament i fysikk og matematikk, men det er en mangel på trening i å kunne bruke kunnskap og ferdigheter på tvers av flere kunnskapsområder (dvs. integrert kompetanse, e.g. matematisk modellering, multifysikk, eksperimenter, simulering, prosjektstyring, etc.) og da spesielt i mer omfattende og åpne oppgaver. Ofte vil en slik dreining innebære flere prosjektbaserte emner. Disse må legges opp slik at alle studentene oppnår læringsmålene. Både generelt og spesielt for dette området bør studieprogrammet ha en tydeligere beskrivelse av målsetningene til studieprogrammet som reelt sett er avgrensende. 

Det er for lite fokus på å trene studentene i å sjekke egne beregninger og forutsetninger.

Det eksisterer ingen overordnet plan for utvikling av digital kompetanse, noe som medfører at utfallet er usikkert, tilnærmingen fragmentert og nivået for lavt ettersom man ikke vet hva man kan bygge videre på. Matematikk 1-4 kunne med fordel hatt større innslag av beregningsorientering.

Selv om beregningsorienteringen må styrkes så er analytiske ferdigheter også en svært sentral ferdigheten som ikke må svekkes i for stor grad. Et eller flere av matematikk 1-4 kunne med fordel gis dedikerte for MTFYMA ettersom programmet har et annet behov for matematiske ferdigheter enn de fleste andre ingeniørprogrammene som tar disse emnene. De kunne for eksempel ha et større innslag av bevisføring som en forberedelse til mer avanserte emner.

Det er uklart hva som menes med \emph{bærekraftskompetanse} for MTFYMA og BFY. Det må defineres hva som ligger i dette for at man skal kunne vurdere om det er oppfylt.

\textbf{Forslag til tiltak:}
\begin{itemize}
	\item Det anbefales at det i større grad utformes læringsaktiviteter hvor studentene må bruke kunnskap og ferdigheter fra flere kunnskapsområder/ferdigheter. Dette vil være ganske omfattende tidsmessig så vil kanskje måtte gjøres i dedikerte emner. Dette inkluderer trening i å vurdere egne beregninger.
	\item Det utarbeides en helhetlig plan over hvordan de digital ferdigheten i programmet utvikles. 
	\item IMF oppfordres til å integrere mer beregningsorientering i Matematikk 1-4 og vurdere egne emner for MTYFMA.
	\item Det utarbeids en definisjon av hva man legger i bærekraftskompetanse og dernest en plan for hvordan studentene evt. kan oppnå denne kompetansen.
	\item Studieprogrammet utvikler tydelige målsetninger for programmet som er avgrensende.
\end{itemize}

\begin{quote}
	\textbf{Prinsipp II:} NTNU skal legge aktivt til rette for at kandidater fra teknologistudiene opparbeider tverrfaglig samhandlingskompetanse, og for at man over den samlede studentpopulasjonen får et mangfold i kunnskapsprofiler, samtidig som den enkelte student oppnår tilstrekkelig programfaglig dybde.
\end{quote}

Det er mulig å velge ulike kunnskapsprofiler i det nåværende studieprogrammene men det er i liten grad synliggjort overfor studentene hvordan flere emner kan settes sammen til en hensiktsmessig profil. Data på emnevalg indikerer at studentene ikke velger så bredt som man kunne ønske. Det er også i liten grad synliggjort for studentene hvordan de kan bruke emner fra andre institutt for å skape en helhetlig profil.

Utover EiT er det ingen planlagte elementer i studieprogrammet for å bygge tverrfaglighet. Det er usikkert om EiT i tilstrekkelig stor grad definerer en reell teknologisk problemstilling som et tverrfaglig team skal løse i praksis. 

\textbf{Forslag til tiltak:}

\begin{itemize}
    \item Planlegge og designe tydelige fagprofiler som synliggjøres for studentene.
	\item Lage emner med prosjekter hvor man er nødt til å bruke fagkunnskap fra ulike studieprogram for å komme frem til en løsning.
	\item Samhandlingskompetansen som introduseres i EiT bør nok også introduseres på et tidligere tidspunkt i studiene slik at man både kan nyttiggjøre seg den gjennom studiene og få mer trening.
\end{itemize}

\begin{quote}
	\textbf{Prinsipp III:} Kontekstuell læring skal legges til grunn som gjennomgående pedagogisk prinsipp i NTNUs teknologistudier.
\end{quote}

Kontekstuell læring brukes i svært liten grad. Det er noe uklart hva slags kontekst som vil bidra til bedre læring. Er det primært en anvendt, realistisk kontekst som er viktig? 

Selv om kontekstuell læring kan være viktig for motivasjon og forståelse, er det den kontekstoverskridende kunnskapen som er noe av den sentrale verdien i studieprogrammene, og denne må ikke gå tapt.

Ulikhet i sjargong og notasjon kan gjøre det vanskeligere å se sammenheng og kontekst mellom emner. Det er viktig at faglærere har et bevist forhold til dette og peker ut sammenhenger

\textbf{Forslag til tiltak:}

\begin{itemize}
	\item Tydeliggjøre hva vi legger i kontekstuell læring og hva som er det vitenskaplige grunnlaget for denne pedagogiske formen.
	\item Det kontekstuelle er viktig for motivasjon og for å gi dypere forståelse. Det er også en ferdighet i seg selv
    \item Innhente eksempler fra industri som kan brukes som utgangspunkt for å presentere teori for studentene. Dette kan inkludere gjesteforelesninger og at problemstillingen inkluderes i øvinger og mindre prosjekter.
	\item Sørge for at alle emner har åpent tilgjengelige læringsressurser slik at faglærere kan se tydelig hva som gjøres i andre emner.
	\item Ha tettere kontakt med næringsliv som kan være en kilde til kontekstuelle problemstillinger.
\end{itemize}

\begin{quote}
	\textbf{Prinsipp IV:} NTNUs teknologistudier skal benytte kunnskapsbaserte, studentaktive og engasjerende undervisnings- og vurderingsformer som er samstemt med utdanningenes overordnede kompetansemål, fremmer god læringskultur, og gir effektiv dybdelæring
\end{quote}

Undervisningsformer ved programmet er i stor grad basert på tradisjon heller enn en kunnskapsbasert tilnærming til hvordan læring fungerer. Ikke dermed sagt at dagens form er feil, men det må vurderes om dagens form bidrar til å oppnå det man ønsker med studieprogrammet.

Emnene er i stor grad preget av stofftrengsel, spesielt på grunnivå, noe som reduserer muligheten for dybdelæring og gir redusert mestringsfølelse, noe som reduserer engasjement og motivasjon.

Fokusere på \enquote{dobbelt læring} slik at studentene opparbeider flere ferdigheter samtidig (eksempelvis verktøy for visualisering sammen med teori).

Dagens eksamensform bidrar nok i noe grad til å utvikle \enquote{algoritmisk problemløsning} på en snever klasse med problemtyper, men det man egentlig ønsker er dypere forståelse for å kunne angripe en større klasse med problemer. Eksempel på tiltak som motivrker dette kan være mer åpne problemer og muntlig eksamen.

Det er også viktig å huske at studentmassen er svært heterogen slik at det er et variert studietilbud som passer mange. 

\textbf{Forslag til tiltak:}

\begin{itemize}
	\item Tydelig plan om innhold i emner på programnivå for å hindre stofftrengsel og forbedre samkjøring.
	\item Forsterke emnegruppene på IFY for å skape mer dialog om undervisningsformer.
	\item Gjør beviste valg av undervisnings- og vurderingsformer slik at det bidrar til god læringskultur. Undervisnings- og vurderingsformer bør til en hvis grad, spesielt på lavere nivå, være styrt av studieprogrammet (eksempler: flipped classroom, studentresponssystemer). Ha anbefalinger eller pålegg om undervisnings- og vurderingsformer. Spesielt i laver årskurs.
	\item Ha et bevist og planlagt forhold til dobbelt læring.
	\item Gjøre tiltak som motvirker \enquote{algoritmisk problemløsning}.
\end{itemize}

\begin{quote}
	\textbf{Prinsipp VI:} Kvaliteten i NTNUs teknologistudier skal utvikles gjennom en programdrevet tilnærming, i kombinasjon med strategisk porteføljeutvikling og -forvaltning på tvers av programmer og programtyper.
\end{quote}

Studieprogrammene er i svært liten grad programdrevet ettersom det i liten grad eksisterer noen skriftlig dokumentasjon på hva innholdet skal være. Det lille som eksisterer i beskrivelsen av studieprogrammet er for vag og brukes i liten grad. Emnebeskrivelsen kan fritt endres av faglærer og er ikke styrt av programmet. Faglærerer tenker i liten grad utover sitte eget emne og i liten grad på hvordan det bidrar til helheten når de underviser.

Det er i liten grad tenkt på hvordan emnene henger sammen og bygger videre på hverandre. Eventuelle koblinger forvitrer også som følge av den enkelte faglærers frihet til å endre form og innhold i emnene.

Undervisning sees nok fortsatt på som et individuelt ansvar for et gitt emne og ikke et kollektivt ansvar. Dette er kanskje noe i endring på IFY som følge av etablering av emnegrupper. Emnegrupper bør styrkes og man bør utivkle en kultur hvor man man ser på utvikling av programmet som et kollektivt ansvar.

System for kvalitetsutvikling virker tunggrodd med mye byråkrati når det skal gjøres endringer. Økonomiske strukturer og regelverk skaper begrensninger. Trenger kanskje å utvikle en struktur som er mer \enquote{agile}.

\textbf{Forslag til tiltak:}

\begin{itemize}
	\item Utvikle dette dokumentet slik at det dokumenterer en felles forståelse av hvordan studieprogrammene bør drives.
	\item Økt fokus emnegrupper på IFY for å styrke den kollektive tankegangen om undervisning.
\end{itemize}

\begin{quote}
	\textbf{Prinsipp VII:} NTNUs kvalitetsarbeid i teknologistudiene skal stimulere studieprogrammenes utvikling mot utdanningskvalitet i verdensklasse ved å fokusere på kontinuerlig forbedring og systematisk utvikling av kvalitetskultur.
\end{quote}

NTNUs kvalitetssystem har to store mangler. For det første er det primært et rapporteringssystem, men sirkelen er i liten grad sluttet ved at det også lages handlingsplaner som skaper endring. I tillegg er emneevaluering på emnenivå tungt basert på hva studentene \emph{syns}, og i liten grad på hva de faktisk lærer. Det er også for lite oppfølging av innspill, det må legges mer oppmerksomhet på rapportene om man skal få til en kulturendring.

Det er noe benchmarking internasjonalt og interessegrupper i forbindelse med dybdeevalueringer. Dette kunne kanskje styrkes ved å ha fast partnere som man bruker som benchmarking over tid. 

Man bør også kanskje ha blikket mer kontinuerlig rettet utover for å følge med på utviklingen internasjonalt, heller enn å kunne få et blikk på sin egen utdanning en gang hvert femte år.

\begin{itemize}
	\item Skape kontinuerlige kvalitetsutvikling gjennom å ha langsiktige planer. Fokusere mer på utvikling enn rapportering. Utviklingsplaner må koordineres på tvers av emner.
\end{itemize}


\begin{quote}
	\textbf{Prinsipp VIII:} NTNU skal gi høy prioritet til strategisk og operativt internasjonalt samarbeid om utvikling av teknologistudier, med mål om å bli et internasjonalt synlig og anerkjent universitet også på dette området.
\end{quote}

Programmene er liten grad aktive i fora hvor det diskuteres undervisning av ingeniører og realister internasjonalt. IMF har vært mer aktive her. Det bør vurdere om man har kapasitet og om det vil være hensiktsmessig å øke denne aktiviteten.

\begin{itemize}
	\item Delta i CDIO nettverket. Revitalisere nordic5tech.
	\item Utveksling av ansatte for undervisningsformål (i tillegg til forskningstermin).
	\item Programmet har bra utveksling men litt lite innveksling. Se om vi kan promotere oss på noen områder hvor vi er spesielt gode.
	\item Bruke ekstern kompetanse for å komplementere eget studietilbud.
\end{itemize}

\begin{quote}
	\textbf{Prinsipp IX:} NTNUs teknologistudier skal vektlegge systematisk samhandling med arbeidsliv og samfunn, med mål om å fremme arbeidsrelevans, legge til rette for livslang læring, og sikre at studenter kan opparbeide relevant arbeidslivserfaring gjennom studiene.
\end{quote}

Studiene har i liten grad samarbeid med eksternt arbeidsliv i studiene.

Et unntak er medisinsk fysikk hvor ansatte på sykhuset er tett integrert i utforming og gjennomføring av undervisning i disse områdenee.

\begin{itemize}
	\item Større innslag av eksterne masteroppgaver. 
	\item Flere II-stillinger fra industri.
	\item Systematisk arbeid for å skape kontaktflater mot industri.
	\item Etablere studiepoenggivende praksis.
\end{itemize}

\subsubsection{Prinsipper primært for andre nivå}

Prinsipp V og prinsipp X er primært relevant for henholdsvis institutt og sentralt nivå. Disse prinsippene er.

\begin{quote}
	\textbf{Prinsipp V:} NTNU skal stille tydelige forventninger til, og gi solid støtte for, kompetanseutvikling hos undervisningspersonell.
\end{quote}

\begin{quote}
	\textbf{Prinsipp X:} NTNU skal utvikle sitt læringsmiljø – og spesielt sin campus og infrastruktur (både fysisk og digital) – i en retning som understøtter de øvrige FTS-prinsippene I – IX, og som fremmer læring, helse og trivsel blant studenter og ansatte.
\end{quote}


	\subsection{fts-competencies}

\subsubsection{K1: Vise fagkunnskap og faglig fundert perspektiv}

Dagens kandidater ved MTFYMA og BFY opparbeider uten tvil en veldig bred faglig kunnskapsbase. Likevel utdypes dette punktet litt mer i detalj og i forhold til noen av disse områdene er det et vist gap mellom ønsket og dagens situasjon. 

Ordlyden og utdypningen presisserer at det som søkes er dyp, virksom kunnskap som søkes, slik at kandidaten er i stand til å \emph{bruke} kunnskapen kreativt og effektivt i problemløsning. Det kan stilles spørsmålstegn ved om dagens eksamensfokus tester kunnskapen relativt overfladisk og over et kort tidsrom, samt at kandidatene får lite erfaring i å anvende kunnskapen. Det vurderes i liten grad om kunnskapen på sikt er dyp og virksom.

Delrapport 1 deler opp dette punktet i 4 deler

\begin{itemize}
	\item ...
\end{itemize}

generaliserbare konsepter.
kontekstualisering
beregningsorientering
stordata, maskinlæring
foretningsforståelse, invoasjonsprosesser

ii) bredde teknisk: prosjektledelse, muliggjørende teknologier

iii) vei dybde vs bredde

iv) komplementært i forhold til fremtidens behov

-bredde i kunnskapsprofiler

Steam - kreativitet

K2 Analyse av komplekse problemstillinger og systemer

K3 Design og implementering av bærekraftige løsninger.


%Dybdeevaluering 2020
\section{Dybdeevaluering 2020}

En ekstern komite og en studentkomite gjennomførte i 2020 en evaluering av studieprogrammene MTFYMA og BFY. Evalueringen fokuserte på \emph{eksperimentelle ferdigheter} og \emph{numeriske ferdigheter}. Hensikten var å dekke beregningsorientering av både fysikk og matematikk men pga av misforståelser i mandatet fokuserte evalueringene på fysikkemnene.

Her følger en oppsummering av anbefalinger fra rapporten, tolket av studieprogramledelsen.

\subsection{Generelt}

\begin[itemize}
	\item Studentene bør eksponeres for oppgaver som integrerer beregningsorientering og eksperimenter, samt få trening i å vurdere om spørsmål besvares best med eksperimenter, beregninger eller ny teori eller en kombinasjon av disse.
	\item Studentene bør selvstendig eller i team kunne levere på den praktiske gjennomføringen og må dermed få opplæring i og erfaring med hele verdikjeden fra problemformulering til rapportering.
	\item Studentene bør få erfaring med å jobbe i prosjektform hvor man jobber individuelt i et team og må planlegge hensiktsmessig arbeidsdeling. Studentene bør også få prosjektoppgaver hvor de jobber individuelt.
	\item Studentene bør få en tydelig innføring i normer og regler relatert til sitering, plagiat, IPR m.m.
\end{itemize}

\subsection{Eksperimentelle ferdigheter}

\begin[itemize}
	\item Studentene bør få mer trening i rapportskriving og erfaring med at rapporter kan ha ulike format i ulike situasjoner, ikke nødvendigvis bare som vitenskapelig artikkel. Det bør være felles læringsressurser og man må ha kontroll på at alle studieretningen får tilstrekkelig god trening i rapportering. 
	\item Studentene må få en systematisk innføring i dokumentasjonspraksis, journalføring, dataintegritet m.m. Dette gjelder både eksperimentelt og beregningsorientert arbeid. Det bør refereres til internasjonale standarder om dette. Studentene bør også eksponeres for elektronisk journalføring, ikke kun håndskrevne. Studentene bør kjenne og forstå FAIR prinsippene for vitenskapelig arbeid og generelt om tankesett og teknologi relatert til open science.
	\item Studiet bør prioritere generiske laboratorieferdigheter foran demonstrasjoner for å støtte opp om teori. Laboratoriundervisning kan ha ulike målsetninger og det bør være tydelig for en gitt undervisningsaktivitet hva som er hensikten.
	\item Det bør være mindre bruke av ferdige oppsett av eksperimenter og mer fokus på at studenten må gjennomføre hele verdikjeden fra problemformulering til rapportering. Spesielt å kunne formulere hypoteser som kan testes av eksperimenter eller numeriske beregninger. Dette vil også innebære mer fokus på å kunne redegjøre for antagelser som er gjort i en analyse. Dette er en forutsetning for meningsfylt rapportskriving.
	\item For å kunne designe eksperimenter forutsetter det at studentene kan bruke sentrale måleinstrumenter og har kunnskap om sentral måleprinsipper. Metoder for automatisk datainnsamling (av store data) er spesielt viktig.
	\item Aspekter relatert til eksperimentdesign som studenten må få trening i: vurdere nødvendig nøyaktighet, vurdere ulike oppsett, identifisere og kvantisere feil/usikkerhet, kalibrering, avpasse oppsett i henhold til tid/ressurser, 
	\item Studentene bør ha kjennskap til en bredt spekter av metoder for analyser data og trening i å anvende disse.
	\item Det bør være egne emner for eksperimentell aktivitet.
	\item Risikoanalyse bør ikke kun inkludere HMS men også prosjektrisiko.
\end{itemize}

\subsection{Beregningsorienterte ferdigheter}

\begin[itemize}
	\item Studentene bør få en systematisk innføring i dokumentasjonspraksis, versjonskontroll og deling av kode.
	\item Studentene bør få trening i distribuerte utviklingsprosjekter hvor flere arbeider med deler av at større prosjekter., 
	\item Studenten bør få trening i enhetstesting.
	\item Det bør gis en helhetlig innføring i numeriske beregninger og ferdighetsstrengene anses som et positivt virkemiddel for å nå dette målet.
	\item Studentene bør få kompetanse på maskinlæring og bruk av HPC ressurser.
	\item Studentene bør ha kjennskap til optimalisering og parallellisering
	\item Ulikhet i obligatoriske emner mellom MTFYMA og BFY gjør det vanskelig å bygge på tidligere emner, noe som er en forutsetning for at høyere grads emner er tilstrekkelig avanserte.
	\item Studentene bør kunne håndtere store, uorganiserte data.
	\item Studentene bør få trening i å bruke bereningsmetoder mot multifysikk-problemer.
	\item Studente bør få trening i å bruke både kommersielle og åpne programpakker.
	\item Studenene bør kjenne, forstå og kunne anvende sentral numeriske algoritmer/metoder og forstå deres begrensninger. 
	\item Studentene bør få trening i algoritmisk tekning.
	\item Studenene bør få erfaring med programvare for symbolske beregninger.
	\item Studentene bør ha erfaring med både skriptede og kompilerte språk.
\end{itemize}

\subsection{Studentevaluering}

\begin{itemize}
	\item Det bør gis spesifikk opplæring i ferdigheter som studentene forventes å kunne beherske eller kjenne til. Dette inkluderer:
	\begin{itemize}
		\item Rapportskriving
		\item Journalføring
		\item LaTeX
		\item Feilanalyse
		\item Relevante biblioteker i Python
	\end{itemize}
	\item BFY bør kanskje ha obligatoriske emner med mer selvstendig laboratoriarbeid.
	\item Mer relevant ITGK (ITGK blir lagt om høsten 2022 med mer fokus på numerikk.)
\end{itemize}

\chapter{Eksterne rapporter}
\label{chap:external-reports}

Denne delen innholder referanser til og vurderinger av rapporter som er utformet av organer utenfor studieprogrammene eller instituttene. For hver rapport skal det som minimum gis en kort oppsummering av innholdet og i hvilken grad det er relevant for studieprogrammene. Spesifikke elementer kan fremheves dersom det anses som hensiktsmessig.


\section{A European specification for physics bachelor studies}

Dette dokumentet beskriver kjennetegn ved bachelor utdanning i fysikk i en europeisk sammenheng. Hensikten med dokumentet er å følge opp Bolognaprosessen ved å ha en felles europisk spesifikasjon av et bachelor utdanning slik at det skal bli enkelt å gå videre på masterutdanning i det europeiske området.

Dokumentet gir en kort kvalitativ beskrivelse av kjennetegn på innhold og format på fysikkutdannelser i det europiske området og gir noen generelle anbeflinger både om form og innhold. I vedlegget er det i en tabell gitt en oversikt over emner som bør dekkes og omtrentlig vekting av de ulike områdene. Dokumentet kan fungere godt som en sjekkliste og minimum av det som bør dekkes. 

Dokumentets er fra 2009 og er i hovedsak basert på tidligere evalueringer (TUNING, EUPEN, STEPS). Lenker til flere av disse prosjektene er ikke lengre tilgjenglige. I tillegg virker det som mye av innholdet er basert på det subject benchmark statement fra det britiske QAA. 
:



\section{AAPT Computational physics}

(Nedtegnet 16.11.2022)

Rapporten "AAPT Recommendations for computational physics in the undergraduate physics curriculum" (\url{https://www.aapt.org/resources/upload/aapt_uctf_compphysreport_final_b.pdf})  er fra 2016 og er i tråd med andre anbfalinger samt nåværende målsetninger til programmet. Den fremhever at \textit{computational physics} er en tredje pillar i fysikk, på linje med teori og eksperiment.

Rapport en gir et rammeverk for hvordan beregningsorientert ferdigheter kan systematiseres. Det diskuteres også en rekke elementer som er viktig å ta med seg. Det meste av dette ligger allerede inne i studieprogrammenes planer.

Noen aspekter som ble fremhevet som muligens ikke er like tydelige i programmenes nåværende planer

\begin{itemize}
	\item Generelle programmeringsferdigheter og beregningsorientering bør utvikles i samme kontekst heller en separat
	\item Ferdighetene skal være 'authentic to the discipline'
	\item 'Best thaught in a lab setting' (i motsetning til typisk teorikurs).
	\item "How to teach computationl physics is anecdotal". Det er altså rom og behov for å gjøre forskning på endringer som gjøres i programmene våre.
	\item Det er flere 'communities' som jobber med beregninsorientering. Det er kanskje lurt om studieprogrammene og/eller fagmiljøene er involvert i disse.
\end{itemize}

Rapporten gir en liste over fysikk-tema som kan egne seg for beregningsmetorder.

Av spesiell interesse er kanskje en liste over potensielle organisatoriske utfordringer som det er viktig å ta hensyn til (se VI. Curricular issues og VII. Challenges).
Rapporten gir referanse til flere artikler som kan gi enda mer mer bakgrunn, samt noen bøker som er gått litt mer grundig til verks når det gjelder implementering av beregningsorientering.


\section{Curriculum guidelines for undergraduate programs in statistical science}

Rettningslinjene er publisert av American statistical association in 2014\cite{asa-guidelines2014}. 

Rapporten trekker fram nødvendigheten av tilstrekkelig bakgrunn innen nøkkelferdighetene:

\begin{enumerate}
	\item Statistiske metoder og teori
	\item Dataforvaltning
	\item Beregninger
	\item Matematisk fundament
	\item Anvendelse av statistikk (kommunikasjon og tekniske ferdigheter)
\end{enumerate}

Noen sentrale anbefalinger fra rapporten:

\begin{itemize}
	\item \emph{Data science} blir viktigere og viktigere. Forberedelse til karrierer i statistikk og data science krever (i tillegg til tradisjonelle matematikk/statistikk-ferdigheter) at man håndterer høynivå programmering og databasesystemer. Store og ustrukturerte datasett krever metoder for å finne mønstre og sammenhenger i høy-dimensjonelle datasett, og metoder for å unngå bias fra slike data. Dette er et datadrevet perspektiv med mindre fokus på hypoteser og statistisk signifikans.
	\item Virkelige anvendelser er viktige. Data burde være en nøkkelkomponent av statistikkurs. Studieprogram burde vektlegge konsepter og metoder for å jobbe med komplekse data, gi erfaring med design av studier, og det å analysere ikke-tekstbok data.
	\item Eksponering mot varierte metoder og fremgangsmåter. Studenter må eksponeres mot forskjellige prediktive og forklarende modeller i tillegg til metoder for modellering og evaluering. Studenter må forstå utfordringene knyttet til design, \enquote{confounding}, og bias. Studenter må kunne være i stand til å bruke sitt teoretiske fundament til fornuftig og solid dataanalyse.
	\item Evne til kommunikasjon er viktig. Studenter må kunne kommunisere komplekse metoder i enkel terminologi til ledere og variert publikum. Studenter må ha evne til å visualisere resultater på en lett forståelig måte.
	\item Studenter som skal gjøre PhD-studier trenger en sterk bakgrunn i matematikk og teoretisk statistikk.
\end{itemize}


\fi % end if statemnt for toggle appendix

\printbibliography

%section{CDIO}
% Gap analysis relative to CDIO standards

% \section {IOP, APS, ...}
% Gap analysis relative to professional societies

% \chapter{Organisasjon}
% Describe the organization that is necessary to maintain this do:cument and ensure its continued use.

% Hvordan sørge for at prosjektet ikke er personavhengig.

% \chapter{Dokumenthåndtering}
% Description of how the document is manged, updated, communicated.
% Plasseres på github fordi ...., github bruker til ... overføres når...



\end{document}
