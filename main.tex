\documentclass[a4paper, oneside, 12pt]{memoir}
% Use twoside for a version compatible with two-sided printing.

% Layout
\usepackage[margin=2cm]{geometry}

\setsecnumdepth{subsection}

% Language
\usepackage[norsk]{babel}

% References
\usepackage{hyperref}

% For block commenting
\usepackage{comment}

% Typesetting
\usepackage{csquotes}


\appendixtrue % \appendixtrue or \appendixfalse to toggle appendix

\begin{document}

\frontmatter
\tableofcontents
\settocdepth{chapter}

\part{Innledning}
% Rollen og strukturen til dette dokumentet.

\mainmatter

\part{Målsetning}
\label{part:goals}

Denne delen beskriver målsetningene med studieprogrammene MTFYMA, BFY og MSPHYS. En tydelig beskrivelse av målsetninger gjør de enklere å utvikle målrettede læringsaktiviteter.

I et studieprogram som strekker seg over flere år vil det være mange ulike læringsmål. For å gjøre det mulig å få oversikt, må læringsmålene organiseres i et hierarki. På det nederste nivået må læringsmålene være tilstrekkelig detaljerte og konkrete til at de enkelt kan brukes til å utforme læringsaktiviteter og vurdere om læringsmålet er oppnådd.

Denne delene er delt opp i følgende kapitler:

\begin{enumerate}
   \item \textbf{Samfunnsoppdrag} beskriver hvilket behov samfunn eller individ studieprogrammet imøtekommer. Dette vil på overordnet nivå diktere hvilke læringsmål som prioriteres.
   \item \textbf{Vision} beskriver på overordnet nivå hvordan studieprogrammet imøtekommer samfunnsoppdraget.
   \item \textbf{Læringsmål} beskriver hierarkiet av læringsmål for BFY og fellesdelen av MTFYMA.
   \item \textbf{MTFYMA studieretninger} beskriver ulike studieretninger (spesialiseringer/minors) på MTYFMA.
   \item \textbf{BFY spesialiseringer} beskriver ulike retninger man kan spesialisere seg på BFY.
   \item \textbf{MSPHYS} beskriver hierarkiet av læringsmål for MSPHYS.
\end{enumerate}

\chapter{Samfunnsoppdraget}
\label{c:mission}

\section{MTFYMA}
Utvikling av nye teknologi vil være basert på en fundamental forståelse av de underliggende prosesser. For mange områder vil teknolgien være basert på dyp forståelse av de fysiske fenomenene og avanserte matematiske metoder for å beskrive dem. Kandidater fra MTFYMA imøtekommer dette behovet for dyp fysisk og matematisk forståelse og de nødvendige praktiske ferdighetene for å kunne anvende denne kunnskapen.

\section{BFY}
Det ligger i menneskets natur å forstå og beskrive verden rundt oss. Denne søken etter fundamental forståelse har lagt grunnlaget for samfunnets teknologisk utvikling. Ofte er ikke anvendelsen i syne når innsikten utvikles. Kandidater fra BFY imøtekommer menneskets og samfunnets behov for forståelse og kunnskapsutvkling av universets mest fundamentale prosesser.

\section{MSPHYS}
Teknolgisk utvikling følger forskningsfronten. Kandidater fra MSPHYS har dyp forståelse og innsikt i et begrenset fagområde, nær forskningsfronten. Kunnskapen og ferdighetene kan brukes til avansert teknologisk utvikling eller videreutvikles til å bidra til flytting av kunnskapsfronten i en PhD eller annen forskningskarriere.
\chapter{Visjon}
\label{c:vision}

\section{MTFYMA}
\vision{
Kandidater fra MTFYMA er velkjente og svært ettertraktet hos alle relevante arbeidsgivere for sine kompetansen innen fysikk og matematikk. 
Programmet er annerkjent for sin pedagogiske kvalitet.
}
\section{BFY}

\vision{BFY tiltrekker seg studenter som er interessert i å forstå naturens fundamentale virkemåte.}
\visiontext{Dette er den primære kandidatprofilen programmet ønsker å tiltrekke seg. Sekundært kan det også være studenter som er litt usikre på hva de ønsker og hvor en BFY kan være et godt springbrett for andre fagfelt.}

\vision{BFY gir stor frihet til å følge egen interesse.}
\visiontext{Friheten balanseres mot basiskunnskap, det som alle fysikere \emph{må} kunne bør være obligatorisk. Likevel er studentenes individuell søken etter forståelse en viktig verdi for programmet og dermed skal studentens frithet til å følge egne interesser stå sterkt. Ingen vet hva fremtiden bringer og hvilken kunnskap som vil være viktig og verdifull. Hva som skal være obligatorisk må bestemmes av fagmiljø i samråde med andre interessenter (f.eks. faglige foreninger).}

\vision{BFY gir mulighet for dyp spesialisering innen et avgrenset område av fysikken som et grunnlag for videre fordypning.}
\visiontext{Ved å starte fordypning innen et avgrenset område kan kandidater fra BFY gjennomføre en mastergrad på høyt nivå, nær forskningsfronten. Studiet får også erfaring med forskningsarbeid, f.eks. gjennom bachelorprosjekt, som også bidrar til å gjennomføre en mastergrad med høy kvalitet.}

\vision{BFY definerer tydelig studieløp for bestemte spesialiseringer}
\visiontext{Disse spesialiseringene er definert av valgfrie emner og er basert på IFY sine nåværende forskningsgrupper. Dette endres dynamisk ettersom faggrupper endrer seg. Emnepakker kan også planlegges sammen med andre enheter, enten på NTNU eller ved andre universitet.}

\vision{BFY bidrar til internasjonalisering av fagområdet og styrkning av nasjonal kompetanse.}
\visiontext{Gjennom å ha studieprogram som er tilpasset strukturen i European higher education area, definert gjennom Bolognaprosessen, kan BFY-kandidater reise ut for masterprogram, spesielt for å studere fagområder som det ikke eksisterer ekspertise på ved NTNU eller i Norge. Dette er viktig for nasjonal kompetanseutvikling. Samtidig er programmet viktig for et velfungerende masterprogram som kan rekruttere internasjonalt og gi et faglig og kulturelt rikere studiemiljø.}

\vision{BFY legger til rette for å ta mastergrad i andre fagområder og spesifiserer disse veiene.}
\visiontext{Noen studenter vil kanskje oppdage at søken etter fundamental forståelse ikke var helt for dem, fant et annet spennende fagfelt eller ønsket en mer anvendt utdannelse. Studieprogrammet prøver derfor å legge opp til at man kan ta emner som kvalifiserer for mastergrad i tilstøtende fagfelt. Disse muligheten spesifiseres og kommuniseres godt for studentene}
\section{MSPHYS}

\vision{
Kandidater fra MSPHYS har dyp forståelse og innsikt i et begrenset fagområde av fysikken, ofte nær forskningsfronten.
De har opplæring i å jobbe selvstendig med faglige problemsillinger. 
Dette gjør dem ettertaktet for videre forskerutdanning, samt for utvikling og analysearbeid i næringslivet.
}
\chapter{Strategiske valg}
\label{chap:strategy}

\section{MTFYMA - Strategiske prinsipper}

\noindent\textbf{Faglig innhold}

%\vision{MTFYMA tiltrekker seg kandidater med spesiell intresse for fysikk \emph{og} matematikk og er organisert rundt disse fagdisplinene.}
%\visiontext{I dette ligger det at studenter som søker seg til programmet er interessert i \emph{både} fysikk og matematikk, ikke enten eller. I tillegg presiserer det at fysikk og matematikk skal være kjernen i studiet og at det ikke skal være en overvekt av nærliggende fagområder som eksempelvis programmering, data science eller kunstig intelligens. De nevnte tema kan likevel opptre som spesialsieringer i programmet. At fysikk og matematikk er kjernen i programmet står likevel ikke i motsetning til at MTFYMA skal være et tydelig ingeniørprogram med tydelig fokus på teknologiske anvendelser eller at arbeidslivet skal være en viktig premissleverandør for programmets innhold.}

\vision{MTFYMA er et umiskjennelig sivilingeniørprogram}
\visiontext{Programmet skal ha et sterkt fokus på å kunne \emph{anvende} fysikk og matematikk mot teknologiske problemstillinger og inneha de ekstra ferdigheter som er nødvendige for å sette teori ut i praksis.
Dette innebærer et sterkt innslag av nødvendige ferdigheter som digitale og eksperimentelle ferdigheter, samt  personlige ferdigheter for arbeidslivet slik som samarbeid, kommunikasjon og prosjektstyring.
Alternativt kan man si at programmet ikke skal kunne mistas for et klassisk disiplinprogram.
Dette innebærer også at kandidatene skal være effektive problemløsere og må eksponeres for åpne oppgaver hvor de må kombinere matematikk, fysikk og eksperimentelle og digitale ferdigheter.}

\vision{studenter ved MTFYMA innser viktigheten av en god forståelse av samfunn, politikk og økonomi for å effektivt kunne bidra til verdiskapning.}
\visiontext{Dette innebærer at disse temaene må inkluderes i studiet på en måte som er meningsfulle for studentene. Disse temaene må integreres i programmet og det er neppe tilstrekkelig å la dette kun diskuteres i eksterne emner som ikke er knyttet til kjernen i programmet.}

\vision{
Studenter fra MTFYMA har bred, kvantitativ kunnskap om det vitenskapelige grunnlag for bærekraftsspørsmål, og evner å analysere disse i forhold til økonomiske og politiske virkemiddel.
}
\visiontext{
Med sin brede naturvitenskapelige basis, har studenter fra MTFYMA en unik komptanse i å kunne forstå det vitenskapelige grunnlaget som ligger til grunn for mange bærekraftsspørsmål. De kan sette seg inn i den vitenskapelige primærlitteraturen og forstå modeller, konklusjoner og eventuelle begrensninger og usikkerheter. For å kunne bidra til en bærekraftig utvikling må studentene kunne koble denne forståelsen til politiske (e.g.bærekraftsmålene) og økonomiske (e.g.kvotesystem) virkemiddel.
}

\noindent\textbf{Organisering}

\vision{MTFYMA høster synergier fra å være et delt ansvar mellom IFY og IMF og kandidater fra programmet står støtt i begge fagdisipliner}
\visiontext{Fysikk og matematikk er et fagfelt som historisk har inspirert og hatt nytte av hverandre. 
Kandidater som har et solid fundament i begge displiner kan bidra til, og dra nytte av denne tverrfagligheten.
Likevel er det en stor adminstrativ kostnad og kulturelle barrierer i et slikt samarbeid. 
Programmet må derfor tydelig kunne høste syneriger i emneinnhold og læringsaktiviteter. Disse må være større enn bare summen av vært institutt for å være hensiktsmessig.}

\vision{MTFYMA skal være et integrert studieprogram med tydelig identitet som er uavhengig av spesialisering og med sterk vertikal og horisontal integrasjon.}
\visiontext{I dette ligger at studentene ved MTFYMA primært tenker på seg selv som MTFYMA-student, og ikke primært er knyttet til sin spesialisering.
Dette innebærer at det bør være felles emner som strekker seg gjennom store deler, kanskje hele, studiet.
Det skal være tydlige og dokumenterte koblinger mellom emner slik at de horisontalt er gjensidig støttende og vertikalt gir effektiv progresjon.}

\vision{MTFYMA har dynamiske studieretninger som tilpasses utvikling i forskningsfronten, arbeidslivet og samfunnets behov}
\visiontext{Studieprogrammet definerer tydelige studieretninger/spesialiseringer. Disse kan opprettes og legges ned basert på endrede samfunnsbehov og fagmiljøenes endrede ekspertise (men også kunne påvirke utviklingen av fagmijløene).} 

%\vision{MTFYMA har klare kompetansemål for utdanningen som gjør våre kandidater attraktive for framtidige arbeidsgivere, og tilfører arbeidslivet kunnskaper om de til enhver tid nyeste metoder.}
%\visiontext{Hovedpoenget her er at programmets innhold skal være styrt av hva som gjør dem til attraktive kandidater for arbeidslivet, i motsetning til attraktive for masteroppgaver og phd studier (i de tilfeller det skulle være en motsetning).
%Samtidig skal programmet bidra til å heve det teknologiske nivået på arbeidslivet ved å tilføre ny kunnskap og innsikt, med den forutsetning at en solid basis i fysikk og matematikk alltid vil være en nødvendig grunnsten i utvikling av ny teknologi.}

\vision{MTFYMA skal ha et tett og godt forhold til industri og andre avtakere av våre studenter, lytte til deres ønsker og behov og være åpen for å modernisere og tilpasse studieprogrammet etter hva som er etterspurt}
\visiontext{Dette innebærer at studieprogrammet må ha etablerte kanaler til industri, og effektive mekanismer for å justere studieprogrammet}.

\vision{MTFYMA legger stor vekt på en kollegial tilnærming i utvikling og gjennomføring av studieprogrammet}

\vision{MTFYMA legger vekt på CDIO og FTS som viktige rammeverk for utforming av studieprogrammet}

\noindent\textbf{Pedagogisk utforming}

\vision{studenter ved MTFYMA opplever et studiemilljø som er støttende og motiverende, samt et studium som oppleves utviklende og utfordrende.}
\visiontext{Dette ligger føringer for hvordan utdanning utformes, fysiske arealer og sosial aktiviteter. Programmet skal tiltrekke seg studenter som ønsker å utfordre seg, men samtidig får god støtte i læringsprosessen. Læringsaktiviteter og helheten i programmet skal utformes for å styrke motivasjon.}

\vision{Studenter fra MTFYMA har en solid faglig basis og effektive læringsstrategier for å effektivt kunne sette seg inn i nye teknologier og metoder gjennom hele livet}
\visiontext{Studenter vil måtte tilegne seg mye ny kunnskap gjennom karrieren. En forutsetning for at dette skal skje effektivt er at den faglig basisen er tilstrekkelig solid. I tillegg må studentene ha god forståelse for hvordan læring fungerer og ha innarbeidet gode læringsstrategier.}

\vision{MTFYMA bruker effektive vurderingsformer og undervisningsaktiviter for å nå programmets mål}
\visiontext{
Studieprogrammet anvender dokumentert effektive vurderings- og undervisningsmetoder. 
Spesielt i de først semstrene er dette styrt av programmet og fagmiljøet som helhet for å etablere det ønskede studiemiljø.
}

%\vision{studenter fra MTFYMA har god forståelse av UNESCOs 17 bærekraftsmål og hvordan deres kunnskap i fysikk og matematikk er essensielle for å nå målene}
%\visiontext{Studentene har oppnådd læringsmålene som beskrevet i \name{UNESCOs education for sustainable develeopment goals}, og har gjennomført prosjekter hvor de anvender sin kunnskap i fysikk og matematikk for å utorske nye løsninger på bærekraftsutfordringene.}





%\vision{MTFYMA skal utdanne kandidater som tilfører arbeidslivet kunnskaper om den til enhver tid nyeste metodikken som er tilgjengelig innen fysikk og matematikk og sentral verktøy som brukes i industrien.}





 \newpage
\section{BFY}

\subsection{faglig innhold}

\vision{BFY tiltrekker seg studenter som er interessert i å forstå naturens grunnleggende virkemåte.}
\visiontext{
Dette er den primære kandidatprofilen programmet ønsker å tiltrekke seg. 
Sekundært kan det også være studenter som er litt usikre på hva de ønsker og hvor BFY kan være et godt springbrett for andre fagfelt.
}

\vision{BFY gir en bred og allsidig  BSc i fysikk som danner et godt grunnlag for videre studier.}
\visiontext{
Programmet skal gi studentene en brei og solid bakgrunn i fysikk samt støttefagene matematikk og informatikk.
BFY danner grunnlaget for videre studier i fysikk og andre nærliggende fagfelt.
}

\vision{BFY gir mulighet for dyp spesialisering innen et avgrenset område av fysikken som et grunnlag for videre fordypning.}
\visiontext{
Ved å starte fordypning innen et avgrenset område kan kandidater fra BFY gjennomføre en mastergrad på høyt nivå, nær forskningsfronten.
Studiet gir også erfaring med forskningsarbeid, f.eks. gjennom bachelorprosjekt, som også bidrar til å gjennomføre en mastergrad med høy kvalitet.
}

\subsection{Organisering}

\vision{BFY bidrar til internasjonalisering av fagområdet og styrkning av nasjonal kompetanse.}
\visiontext{
Gjennom å ha studieprogram som er tilpasset strukturen i European higher education area, definert gjennom Bolognaprosessen, kan BFY-kandidater reise ut for å følge masterprogram, spesielt for å studere fagområder som det ikke eksisterer ekspertise på ved NTNU eller i Norge.
Dette er viktig for nasjonal kompetanseutvikling. 
Samtidig er programmet viktig for et velfungerende masterprogram som kan rekruttere internasjonalt og gi et faglig og kulturelt rikere studiemiljø.
}

\vision{BFY gir stor frihet til å følge egne interesser.}
\visiontext{
Friheten balanseres mot basiskunnskap, det som alle fysikere \emph{må} kunne bør være obligatorisk.
Likevel er studentenes individuell søken etter forståelse en viktig verdi for programmet og dermed skal studentens frithet til å følge egne interesser stå sterkt.
Ingen vet hva fremtiden bringer og hvilken kunnskap som vil være viktig og verdifull.
Hva som skal være obligatorisk må bestemmes av fagmiljø i samråde med andre interessenter (f.eks. faglige foreninger).
}

\vision{BFY definerer tydelige studieløp rettet mot bestemte spesialiseringer.}
\visiontext{
Disse spesialiseringene er definert av valgfrie emner og er basert på IFY sine nåværende forskningsgrupper.
Dette endres dynamisk ettersom faggrupper endrer seg. 
Emnepakker kan også planlegges sammen med andre enheter, enten på NTNU eller ved andre universitet.
}

\vision{Med passende fagvalg er BFY en selvstendig og avsluttende grad.}
\visiontext{
Om studentene valger tilstrekkelige med informatikk og andre anvendte emner, bør en BFY grad kunne være en stelvstendig og avsluttende grad.  
Kandidatene skulle da kunne gå ut i næringslivet med jobber innen programmering, data analyse etc. (stillinger som BSc ingeniører ofte har idag)
}

\vision{BFY legger til rette for å ta mastergrad i andre fagområder og spesifiserer disse veiene.}
\visiontext{
Noen studenter vil kanskje oppdage at søken etter fundamental forståelse ikke var helt for dem, fant et annet spennende fagfelt eller ønske en mer anvendt utdannelse. 
Studieprogrammet prøver derfor å legge opp til at man kan ta emner som kvalifiserer for mastergrad i tilstøtende fagfelt.
Dette kan bidra til at studentene fullfører BFY-graden før de fortsetter studiene mot et annet fagfelt.
Disse muligheten spesifiseres og kommuniseres tydelig til studentene.
}

\subsection{Pedagogisk utforming}

\vision{Studenter ved BFY opplever et godt studiemiljø som pirrer den faglige nysgjerrigheten og samtidig som fostrer kollegialt samarbeid.}
\visiontext{
Dette ligger føringer for hvordan utdanning utformes, fysiske arealer og sosial aktiviteter. 
Programmet skal tiltrekke seg studenter som ønsker å utfordre seg, men samtidig får god støtte i læringsprosessen.
Læringsaktiviteter og helheten i programmet skal utformes for å styrke motivasjonen.
} \newpage
\section{MSPHYS}
\label{sect:strategy-msphys}

\subsection{Faglig innhold}

\vision{
Studenter ved MSPHYS fordyper seg i et gitt område, ofte nær forskningsfronten
}

\vision{
MSPHYS gir studetene en reel smakebit på hva forsking er, og utvikler kompetanse som er av interesse både for akademia og instituttsektor, eller videre forskerutdanning.
}

\vision{
Studenter ved MSPHYS jobber selvstendig med faglige problemstillinger og er trent i kritisk tenkning, systematisk arbeid, samt å samarbeide med forskere innen sitt fagfelt.
}

\vision{
MSPHYS gir mulighet til dyp spesialisering for å kunne bidra med kunnskap nær forskningsfronten for arbeidslivet.
}

\subsection{Organisering}

\vision{
MSPHYS definerer tydelige spesialiseringsløp.
}

\vision{
Innholdet i MSPHYS kan designes til å rette seg inn mot industriell forskning.
}
\visiontext{
Om studentene er interessert i å få en mer arbeidslivsnær utdanning, har man mulighet for å ta en ekstern MSc oppgave i samarbeid med en industriell aktør.
}

\vision{
Studenter ved MSPHYS får møter definerte tydelige spesialiseringsløp.
}

\subsubsection{Pedagogisk utforming}

\vision{
Studenter ved MSPHYS blir en del av en aktiv forskningsgruppe og lærer å jobbe og diskutere med dens medlemmer (spesielt hovedveileder).
}
\chapter{Læringsmål}
\label{c:los} % los for learning outcomes

Dette kapittelet beskriver hierarkiet av læringsmål (læringsutbytter). Dersom det ikke er spesielt angitt gjelder læringsmålet for både MTFYMA og BFY. For læringsmål som gjelder for kun det ene studieprogrammet er dette angitt med M/B for henoldsvis MTFYMA og BFY.

Overordnet inndeling av læringsmål

\begin{itemize}
	\item Matematikk
	\item Fysikk
	\item Beregningsorienterte ferdigheter
	\item Eksperimentelle ferdigheter
\end{itemize}

\section{Matematikk}
\label{s:math}
Matematikk


\section{Fysikk}
\label{s:physics}

Fysikk



% Overordnet målsetning.
%\chapter{Overordnede målsetninger}

Dette kapittelet beskriver de overordnede målsetningene med programmet. Hensikten med å beskrive målsetningene på dette nivået er at det gir et oversiktlig sett med målsetninger som kan fungere som rettesnor og gi en felles forståelse av hva vi ønsker å oppnå med programmet\footnote{Dette er bare et utkast. Det planlegges at felles målsetninger utarbeides i løpet av høsten 2022}.

En slik beskrivelse skal ikke være et minste felles multiplum som kan aksepteres enstemmig. Det vil være uenighet (og kontinuerlig diskusjon) om innholdet. De overordnede målsetningene representerer fagmiljøets majoritet, eksterne avtakere og føringer fra ledelse.

Utsagn som ikke virker begrensende for innholdet eller som for de fleste innen fagmiljøet er selvfølgelige, bør unngås.

\section{MTFYMA}

%Brynjulf
\begin{itemize}
	\item MTFYMA skal være et attraktivt studieprogram som tiltrekker seg elever fra videregående skole som har en spesiell interesse for fysikk og matematikk.
	% Hva ligger i å være attraktivt? Må man ha spesielt talent?
	\item MTFYMA skal ha klare kompetansemål for utdanningen som gjør våre kandidater attraktive for framtidige arbeidsgivere.
	\item MTFYMA skal være et integrert studieprogram med en egen tydelig identitet som i størst mulig grad er uavhengig av spesialisering i senere del av studiet.
	\item MTFYMA skal ha en tydelig profil innenfor fagområdene fysikk og matematikk som gjennom hele studieløpet kjennetegnes ved at det har en høyere målsetning innen disse fagene enn øvrige sivilingeniørprogrammer.
	\item MTFYMA skal være et umiskjennelig sivilingeniørprogram.
	\item MTFYMA skal utdanne kandidater som tilfører arbeidslivet kunnskaper om den til enhver tid nyeste metodikken som er tilgjengelig innen fysikk og matematikk for å arbeide med industrielle problemstillinger.
	\item MTFYMA vil etterstrebe et godt læringsmiljø for studentene, gode studieforhold, aktive undervisningsmetoder og ligge i front når det gjelder vurderingsformer og samstemt undervisning.
	\item MTFYMA skal ha et tett og godt forhold til industri og andre avtakere av våre studenter, lytte til deres ønsker og behov og være åpen for å modernisere og tilpasse studieprogrammet etter hva som er etterspurt.
\end{itemize}

%Magnus
\begin{itemize}
	\item MTFYMA gir, I tillegg til det teoretiske grunnlaget, nødvendige ferdigheter i metoder og verktøy som gjør kandidatene til effektive problemløsere.
	\item MTFYMA skal gi studentene trening og trygghet i å anvende sammensatt kunnskap og ferdigheter for å løse komplekse problemstillinger.
	\item MTFYMA er et studieprogram som i stor grad er programmert (obligatoriske emner) for å effektivt oppnå ønsket kunnskap og ferdigheter.
	\item MTFYMA gir i tillegg til det faglige fundamentet også kunnskap og ferdigheter som er viktige som ansatt i en bedrift.
\end{itemize}

\section{BFY}
\begin{itemize}
	\item BFY gir tilstrekkelig valgfrihet mot slutten av studiet til å begynne spesialisering mot ulike fagfelt både innen og utenfor fysikk, og sørge for at studenten kan kvalifisere seg til opptak til disse studiene.
\end{itemize}
%Kombinasjonen BFY + MSPHYS gir mulighet for dypere spesialisering i fagområder innen fysikk, mens muligheten for kombinasjonen BFY + annet masterprogram, gir en bredde av kompetanseprofiler.

\section{BFY vs MTFYMA}

Denne delen spesifiserer forskjellene mellom BFY og MTFYMA og de ulike rollene til programmene.

\subsection{MTFYMA}
MTFYMA er et sivilingeniørprogram i fysikk- og matematikk. I tillegg til et bredt fundament i fysikk og matematikk legger studieprogrammet stor vekt på komplementære ferdigheter som er viktige for en ingeniør: praktiske ferdigheter, samarbeid, ledelse, økonomi m.m.  

Studieprogrammet er i høy grad programmert for å effektivt kun gi studenten de ferdighetene som ansees som viktige for en fremtidig ingeniør.

\subsection{BFY}
BFY gir et solid fundament i fysikk og støttende fagområder. Programmet gir mulighet til å utvikle unik kompetanse langs flere akser.

\begin{enumerate}
	\item Studieprogrammet gir mulighet til å spesialisere seg mot et gitt område i fysikk i siste halvdel av programmet. Sammen med videre fordypning på MSPHYS gir dette en mulighet til fordypning nær forskningsfronten og kan være et ideelt springbrett for videre forskningskarriere.
	\item IFY dekker kun en liten del av alle mulige fysikkområder. BFY gir mulighet for å ta en mastergrad ved en annen/utenlandsk institusjon, noe som vil øke nasjonal og regional kompetanse.
	\item BFY gir et godt grunnlag for mange masterprogram også utenfor fysikk. Siste halvdel av studiet kan evt. brukes til ta emner som kreves for opptak på disse studiene. En grad fra BFY sammen med en annen mastergrad (e.g. ingeniørretning, data, matematikk, pedagogikk), gir unik kompetanse.
\end{enumerate}

\subsubsection{Praktiske aspekter}

MSPHYS programmet er viktig for internasjonalisering av fysikkstudiet ved NTNU og dette kan være vanskelig å opprettholde uten BFY programmet.

\subsection{Beskrivelse av nåværende situasjon (2022)}
Studieprogrammene MTFYMA og BFY er begge fysikk programmer. Det første er et integhrert teknologisk 5-årig master program i fysikk og matematikk, mens BFY er et klassisk 3-årig bachelor program i fysikk. Hovedforskjellene mellom de to programmene ligger i at BFY gir studentene mer valgfrihet, hovedsaklig i 3 året samt at man etter står mer fritt til å velge seg hovedprofil for graden. Dessuten gir BFY en grad etter hva som tilsvarer normert 3-års studier. Dette åpner for muligheten for å gå ut i jobb, eller fortsette sine studier med en master grad i fysikk eller andre nærliggende fagfelt ved samme eller annen institusjon i inn og utland. Man har ikke automatisk~\footnote{Rent praktisk har man ofte løst dette ved at MTFYMA studenten blir tatt opp på BFY studiet og blir gitt muligheten til å sitte igjen med en grad etter 3 års. Dette fungere fordi man har et BFY program i fysikk og at fysikkdelen av de 3 første årene er ganske like mellom de to programmene. Merk at om man ikke hadde BFY programmet, ville MTFYMA studenter som forlater programmet før det er fullført, ikke kunne sitte igjen med en grad for sin utdannelse.} samme mulighet innenfor MTFYMA programmet. En annen prinsipiell forskjell mellom programmene er at MTFYMA, som et teknologi program, er underlagt FUS (Forvaltningsutvalget for sivilingeniørutdannelsen) som fremsetter flere krav til fellesemner. Disse fellesemnene er ikke nødvendige i BFY programmet (med mindre programrådet bestemmer dette). Resultatet er at BFY programmet er mer fleksible i hvilke ikke-fysikk fag som kan velges. 

Vi vil nå se nærmere på de konkrete forskjellene mellom de to studie programmene. I løpet av de to første årene, har begge programmene de samme obligatoriske fysikkemnene, mens matematikk delen er idag er noe forskjellig. De to første årene har BFY 6 obligatoriske 7.5vt emner i matematikk/statistikk, det samme som er tilfellet for MTFYMA. Dog skal det nevnes det er tale om forskjellige matematikk emner. F.eks. inngår numerisk matematikk emnet \textit{TMA4320: Introduksjon til vitenskapelige beregninger} i MTFYMA, mens det ikke er obligatorisk for BFY selv om de kan velge dette.

Når det gjelder informatikk og numerikk-emner, er det større forskjeller mellom programmene. MTFYMA har 3 slike obligatoriske emner medregnet TMA4320: Introduksjon til vitenskapelige beregninger, mens BFY bare har \textit{TDT4110: Informasjonsteknologi, grunnkurs} som obligatorisk. Denne forskjellen er utfordrende for den numeriske aktiviteten i senere fysikkemner.

Når man kommer til 3. studieår begynner valgfriheten for BFY programmet å bli tydelig. Her har dette programmet ett obligatorisk fag i hvert av de to semestrene. De resterende 6 emnene er hovedsakelig fysikk emner og de kan velges fra en liste på 9 fysikk og 2 ikke fysikk (romteknologi) mener. Ett av disse valgbare emnene er en 15sp bachelor oppgave. I MTFYMA programmvil innholdet i 3. studieår variere med studieretningen man velger. Om man ser på retningen «Teknisk fysikk», som trolig ligger nærmest opp mot BFY programmet, har studentene ingen valgbare emner. De obligatoriske emnene består at 7 fysikk emner samt teknologiledelse. 



% Beskrivelse av egenarten til programmene.

%\chapter{Detaljerte kompetanser}
\label{chap:competencies}
Dette kapittelet beskriver kompetansen som studentene skal opparbeide. De ønskede kompetansene er svært omfattende så for å få oversikt må kompetansene systematiseres på ulike nivåer. De overordnede kompetanseområdene er \footnote{FTS beskriver en kompetanseprofil og et sett med 12 kompetansemål. Disse anses som for lite konkrete til å være effektive virkemiddel for å systematisere innholdet i programmet. Vi har her derfor heller omformulert disse slik at de er mer konkrete for fagområdene fysikk og matematikk. En egen oversikt vil vise hvordan anbefalingene fra FTS er fulgt opp.}

\section{Kompetanseområder}

\begin{itemize}
	\item Matematikk, teori.
	\item Fysikk, teori.
	\item Analytiske ferdigheter.
	\item Modellering.
	\item Numeriske metoder og programmering.
	\item Eksperimentelle ferdigheter.
	\item Kommunikasjon.
	\item Innhenting og vurdering av informasjon.
	\item Samarbeid.
	\item Planlegging og gjennomføring.
	\item Læring.
	\item Komplementære fagområder.
\end{itemize}


%\section{Digitale ferdigheter}

Ettersom digitale ferdigheter er sammensatt av mange komponenter er det nyttig å få systematiserte ferdighetene for å få oversikt og som et verktøy for å legge opp studiet. Her følger et forslag til overordnet inndeling:

\begin{itemize}
	\item Vitenskapelige ferdigheter
	\item Generelle IT-ferdigheter
	\item Generelle programutviklingsferdigheter
	\item Språk
	\item Algoritmer
	\item Programvare
	\item Praktiske ferdigheter
\end{itemize}

De neste delene gir en ytterligere detaljering av ferdighetene under hvert kompetanseområde:

\subsection{Vitenskapelige ferdigheter}
\begin{itemize}
	\item Omsette en matematisk/fysisk modell til kode
	\item Redegjøre for antagelser, forenklinger
	\item Tolke numeriske beregninger, fysisk vurdere om resultaten er realistiske
	\item Hensiktsmessig rapportering og presentasjon
\end{itemize}

\subsection{Generelle IT ferdigheter}
\begin{itemize}
	\item Linux
	\item Kommandolinjeverkøy
	\item Scripting, automatisering, batch-processing
	\item Package managers (conda, pip)
	\item Software stacks
	\item Presisjon, numeriske feil, konvergens
	\item Pseudo-random numbers
	\item Dimensjonløs representasjon
	\item Hardware (minne, lagring) 
	\item Parallellisering
	\item HPC ressurser
	\item Prosessere data, ulike dataformater, uorganiserte, mangelfulle data.
	\item Søke i og lese teknisk litteratur.
	\item Effektiv visualisering
	\item (Datasikkerhet)
\end{itemize}

\subsection{Generelle programutviklingsferdigheter}
\begin{itemize}
	\item Generelle elementer i programmeringsspråk (variabler, kontrollstrukturer osv.). Dekkes av ITGK.
	\item IDE (vscode)
	\item Dokumentasjon av kode
	\item Versjonskontroll
	\item Enhetstesting, validering (analytisk løsning, method of manufactured solution)
	\item Debugging
	\item Style guides
	\item Strukturering av kode
	\item Effektivisere kode
	\item \enquote{Computational cycle}
\end{itemize}

\subsection{Språk}
\begin{itemize}
	\item High level language: Python (Numpy, Matplotlib, SciPy, Pandas, Seaborn, Scikit-learn, Tensorflow, Requests, Pytest)
	\item Low level language: C++
	\item Symbolic computing: Python/sympy
	\item Markup: LaTeX, Markdown.
	\item (og/eller R, Matlab, C, Julia, Haskel, Fortran...)
\end{itemize}

\subsection{Algoritmer}
\begin{itemize}
	\item Numerisk integrasjon
	\item ODE
	\item PDE (elementmetoder)
	\item Linear algebra
	\item Optimization (root-finding, lineær programming)
	\item Transformasjoner (fourier, wavelet)
	\item Stokastiske metoder (Monte Carlo, metropolis)
	\item Statistikk
	\item Kurvetilpasning
\end{itemize}


\subsection{Programvare}
\begin{itemize}
	\item Office-pakken
	\item (Comsol multiphysics, Ansys, Zeemax optic studio, Openfoam, Molsim)
\end{itemize}


\subsection{Praktiske ferdigheter}
\begin{itemize}
	\item Planlegge og gjennomføre utviklingsprosjekter
	\item Samarbeid om programutvikling
\end{itemize}

\subsection{Pedagogisk-stratgiske valg}

\begin{itemize}
	\item All aktivitet skal være så autentisk som mulig.
	\item All koding gjøres i rene tekstfiler. Jupyter notebooks kan brukes til presentasjon av teori og til å øve på enkle programmeringselementer.
	\item Fungerende hardware og software er studentens ansvar. De får hjelp så langt ressursene strekker til, men om ikke studenten får maskinen til å fungere betyr det at det ikke består aktiviteten.
	\item Alle ferdigheter som studenten skal oppnå, undervises og kobles til konkrete læringsaktiviteter.
	\item Prosjekter skal bidra til motivasjon og kreativitet,
	\item Matematiske algoritmer bør gjenbrukes på flere anvendelser slik at studentene for en forståelse av deres generelle anvendbarhet.
\end{itemize}

%\section{Eksperimentelle ferdigheter}

\subsection{Læringsmål}

Eksperimentelle ferdigheter har to hovedmål:
\begin{enumerate}
	\item Det skal gjøre studenten i stand til å besvare teknologiske og vitenskapelige spørsmål gjennom eksperimenter. 
	\item  Det skal gjøre studenten i stand til å skape teknologiske løsninger som skaper verdi gjennom å manipulere energi, materie og informasjon. \footnote{Merk at \emph{fysisk forståelse} ikke er et læringsmål. For at læringsaktivitetene skal ha et tydelig mål settes det skarpt organisatorisk skille mellom aktiviteter som skal gi fysisk forståelse og aktiviteter som skal gi eksperimentelle ferdigheter, slik det er definert her. Praktiske aktiviteter kan likevel ha en verdi for å gi fysisk forståelse, men slike aktiviteter organiseres i gjennomføring av teori-emner. Eksperimentell trening vil også nødvendigvis gi dypere fysisk forståelse, men dette er en positiv konsekvens og ikke hovedmålsetningen.}
\end{enumerate}

For å oppnå disse overordnede målene må studentene utvikle følgende ferdigheter:

\begin{description}
	\item [Designe eksperiment] Dette inkluderer å konkretisere spørsmål til noe som kan besvares eksperimentelt, utforme hypotese, velge instrumentering, metrologi, kalibrere, teste, videreutvikle, HMS, RRI, m.m.
	\item [Modellere eksperimentet] Lage en matematisk representasjon av eksperimentet for å kunne tolke data, vurdere gyldighet m.m.
	\item [Gjennomføre eksperimentet] Lagre, dokumentere, metadata, FAIR m.m.
	\item [Analysere og tolke data] Filtrere, representere, kondensere, feilkilder, visualisere, sammenstille med modell, tolke resultatene og syntetisere det til nyttig kunnskap, m.m.
	\item [Kommunisere] Ulike format, åpen vitenskap, etisk.
\end{description}

Disse ferdighetene sammenfatter hovedmål 1). Hovedmåle 2) innebærer i stor grad samme prosesser men hvor designprosessen får større plass, man  designere mot spesifikasjoner istedenfor et spørsmål og at produktets kommersielle verdi får større vekt.

Noen sentrale pedaogisk-strategiske valg:

\begin{itemize}
	\item Alle læringsaktiviteter skal være så \emph{autentiske} som mulig. Eksempelvis så bør det benyttes instrumenter, elektronikk og metoder som er vanlige i forskning og teknologi (altså ikke instrumenter som er laget primært for pedagogiske formål). Studentene må være involvert i hele verdikjeden beskrevet nedenfor. Kommunikasjon i autentiske formater (ikke lab-rapport, masteroppgave).
	\item Ikke gruppearbeid men team-arbeid. Det er ikke effektivt eller vanlig at man i arbeidslivet har flere personer som gjør samme ting. Men man jobber ofte i team og fordeler arbeidsoppgaver.
	\item Studentenes evne til å gjennomføre hele verdikjeden må bygges gradvis. Dersom studenten ikke skal gjennomføre noen deler må det gjøres på en autentisk måte. Eksempelvis er det vanlig å \enquote{kjøpe tjenester} (for eksempel fra en kjernefasilitet) som kan inngå som en naturlig del i den eksprimentelle kjeden, men studentene må da være en tydelig \enquote{kunde} i prosessen. Et annet eksempel kan være å bruke åpent tilgjengelige data.
	\item For et gitt område, skal det benyttes felles læringsmateriell gjennom hele studieforløpet.
	\item Den teoretiske bakgrunnen for eksperimentet bør i så stor grad som mulig være kjent. Senere i studiene kan et større element av å måtte sette seg inn i nytt fagstoff før eksperimentet inkluderes, og at denne ferdigheten da er et eget læringsmål.
	\item Ettersom studenten selvstendig skal utføre elementene må man nødvendigvis starte med det som skal til for å kunne gjennomføre en måling. Så å si alle målinger i dag gjøres om til et elektrisk signal som lagres digitalt på en datamaskin. Det første studenten må lære for å kunne gjennomføre et eksperiment er å kunne sette opp en sensor, en transducer, en ADC og en datamaskin.
	\item Ingen av læringsmålene bør ha større eller mindre vekt på vurdering bare fordi de er enklere å vurdere (som eksempelvis en rapport har vært).
	\item Undervisningen deles grovt opp mellom 1) opplæring i de nødvendige elementene (e.g. spesifikke instrumenter, statistiske analyser, rapportformater) og 2) gjennomføring av eksperimenter.
	\item Eksperimenter som gjennomføres på datamaskiner organiseres i begynnelsen av studiene under digitale ferdigheter men dette skillet vil viskes ut senere i studiet hvor man typisk simuleringer og fysiske eksperimenter støtter opp om hverandre.
	\item Prosjektene blir gradvis mer komplekse. En mulig progresjon er: Individuelt arbeid i mindre prosjekter - team i mindre prosjekter - team i større prosjekter - prosjekter over flere studentkull/flere semestre - tverrfaglige prosjekter - industriprosjekter - internasjonale prosjekter.
\end{itemize}




\part{Gjennomføring}
\label{part:tla}

Denne delen beskriver \emph{hvordan} man skal oppnå de målsetninger man har satt seg i forrige del av dokumentet. Denne delen starter med å beskrive det læringsmiljøet man ønsker å etablere og dernest hvilke undervisnings- og vurderingsformer som bør brukes for å bygge dette læringsmiljøet.

	\chapter{Læringsmiljø}
	
	\chapter{Undervisningsformer}
	
	\chapter{Vurderingsformer}

\part{Oppbygging}
\label{part:program-structure}

Denne delen beskriver hvordan man hensiktsmessig kan bygge opp studiet for å oppnå de målsetninger som ble satt i første del, gjennom de læringsaktivitetene som ble beskrevet i andre del. Beskrivelsen starter med et overordnet nivå (emnevegg) og går til enkeltelementer i emnene.

\chapter{MTFYMA studieretninger}
\label{c:mtfyma-spec}

Studieretninger

\chapter{BFY spesialiseringer}
\label{c:bfy-spec}
Beskrivelse av muligheter for spesialisering på BFY.
\chapter{MSPHYS}
\label{msphys}

Følgende spesialiseringer er definert for MSPHYS:


\ifappendix
\appendix

\renewcommand{\appendixtocname}{Tillegg}
\renewcommand{\appendixpagename}{Tillegg}
\part*{Tillegg}
\addcontentsline{toc}{part}{Tillegg}
%\appendixpage*

\chapter{Utviklingsplan}

Denne delen inneholder konkrete planer for endringsprosjekter med mål om realisere målsetningene med studieprogrammet.

\section{Prosjektplan}

\section{Prosjekter}

	\subsection{FTS-revisjon}
Her følger en beskrivelse av overordnet plan for revisjon av studieprogrammet som følge av anbefalingene som er gitt av FTS-prosjektet. Revisjon av studieprogrammet som følge av dybdeevaluering av MTFYMA og BFY for 2020 inkluderes i samme prosjekt.

FTS-revisjon deles opp i fire faser. Prosjektplanen tar utgangspunkt i at en omfattende omlegging kan gi en signifkant forbedring av studieprogrammet. Etter de to første fasene vurderes det om det er tilstrekkelig behov for å gå videre med neste fase eller om en nedskalert revisjon er mer hensiktsmessig. Det kan være naturlig å starte noe arbeid med en senere fase selv om den foregående fasen ikke er ferdigstilt, men det bør ikke legges for mye ressurser i et slikt arbeid før man er sikker på at man ønsker gå videre med neste fase. Tentativ år/måned for avslutning av hver fase er angitt. Ettersom det er vanskelig å vurdere hvor lang tid hver fase vil ta vil disse justeres ettersom prosjektet skrider frem.

\begin{itemize}
	\item Fase 1: Gapsanalyse og visjon (2022/12). Identifisere gap mellom anbefalinger og ønsker om hvordan man ønsker at studieprogrammet skal være og nåværende situasjon. Gapsanalysen vil minimum omfatte FTS prinsipper og kompetansemål samt dybdeevaluering fra 2020. Dersom gapet anses å være tilstrekkelige går man over i Fase 2 for å utvikle og dokumentere ønskede kompetansemål
	\item Fase 2: Kompetansemål (2023/12). Utvikle ønskede kompetansemål fra overordnet til detaljert nivå. Samtidig beskrives hvordan studentene best kan oppnå denne kompetansen. Dersom kompetansemålene ikke enkelt kan realiseres i dagens programstruktur går man videre til Fase 3 for å utvikle en ny programstruktur.
	\item Fase 3: Programstruktur (2024/6). Det utformes en ny programstruktur for studieprogrammet.
	\item Fase 4. Læringsaktiviteter og -ressurser (2025/6). Disse utvikles før oppstart av nytt studieprogram.
\end{itemize}

\subsubsection{Fase 1}
Fase 1 gjennomføres høsten 2022. Følgende aktiviteter gjennomføres:

\begin{itemize}
	\item \textbf{FTS prinsipper og kompetanseprofiler}. Det gjennomføres et seminar hvor deltagerne deles opp i grupper. Oppgaven vil være å vurdere hvert prinsipp og hver delkompetanse anbefalt av FTS og konkretisere hva disse betyr eller hvordan de kan omformuleres for MTFYMA og BFY. Dernest å vurdere om disse aspektene er imøtekommet i dagens studieprogram og hvis ikke, hvilke tiltak som kan treffes for å imøtekomme dem. Resultatet av diskusjonen kan gjerne formuleres som overordnede målsetninger for studieprogrammet. Det legges opp til frivillig deltagelse på seminaret og alle inviteres (fast og midlertidig ansatte og studenter). En halv dag (ca. 3 timer) er trolig tilstrekkelig. Noen fra FTS inviteres til å gi en kort innledning. Gruppene leverer skriftlige forslag (i tillegg til muntlige presentasjon på seminaret). Det gis også mulighet for de som ikke har anledning til å delta på seminaret å sende inn skriftlige innspill. Forslagene fra seminaret resulterer for hoveddokumentet i oppdaterte gapsanalyse i forhold til FTS (\ref{sec:fts-principles} og \ref{sec:fts-competencies}) , oppdatert tiltaksliste (\ref{sec:changes}) samt oppdaterte (overordnede) målsetninger for programmet (\ref{part:goals}) der det er relevant. Gjennomført innen utgangen av 9/2022.
	\item \textbf{Dybdeevaluering 2020}. Det gjennomføres to seminarer av samme format som over, et for digitale ferdigheter (numerikk og programmering) og et for eksperimentelle ferdigheter. Tilbakemeldinger inkluderes i gapsanalysen av evalueringen (\ref{sec:evaluation2020}) og tiltakslisten (\ref{sec:changes}). Basert på forslagene utformes begynnelsen på en oversikt over målsetninger for disse kompetanseområdene. Gjennomført innen utgangen av 11/2022.
	\item \textbf{Sammenlikning med andre programmer}. Det samles inn en oversikt over hvordan studieprogram som likner på MTFYMA og BFY ved andre studiesteder som vi ønsker å sammenlikne oss med har lagt opp sine studieløp. Dette gjøres primært på emnenivå. Dette resulterer i en kort rapport som beskriver studieprogrammene. Opplegg som virker spesielt spennende for MTFYMA og BFY å lære noe av fremheves. Dette arbeidet gjennomføres av en liten arbeidsgruppe hvor minst en person som kan representere hver studieretning er representert. Denne rapporten blir viktig for at revisjonen ikke skal bli for fastlåst i eksisterende struktur og blir viktig når programmets struktur skal legges senere i prosjektet. Ferdigstilt innen 10/2022.
	\item \textbf{Fagmiljøenes ønsker}. Fagmiljøene (seksjoner, faggrupper) inviteres til å sende inn synspunkter på hva som er viktig å inkludere i de første årene av studieprogrammet for videre fordypning i deres fagområde. Dette kan være ting som er viktig å bevare eller ting som er viktig å inkludere. Dette kan omfatte teoretisk kunnskap men også praktiske eller personlig ferdigheter. Dette materialet vil brukes til å begynne utvikling av målsetninger for studieprogrammene. Ferdigstilt innen 11/2022.
	\item Studieprogrammet gjennomfører i tillegg følgende aktiviteter.
		\begin{itemize}
			\item Samtaler med andre program som har gjennomført endringer i studieprogram for å lære av disse (minimum elsys...).
			\item Innledende gjennomgang av CDIO standardene for å vurdere hvordan de eventuelt skal brukes i den videre styringen av studieprogrammene.
			\item Samtaler med CEED for å diskutere hvordan de kan involveres i prosjektet.
			\item Gjennomgang av rapporter fra IOP og APS (eventuelt flere) om studieprogram i fysikk for å ta med eventuelle anbefalinger derfra.
		\end{itemize}
\end{itemize}

I løpet av 12/2022 gjennomføres møte med fakultet- og instituttledelse for å beslutte om man skal gå videre med planene om en gjennomgripende omlegging av studieprogrammene eller om man skal begrense seg til mindre inkrementelle justeringer. Dette vil baseres på gapsanalysen og tiltakslisten som er utviklte gjennom høsten.

%Høst Innhold- og strukturprosjekter: Overordnet struktur, beregning, eksperimentelt, læringsmiljø, vurderingsformer, undervisningsformer, mekanikk, elmag, matematikk, ønsker faggrupper.

%Revider prosjektplan. Større/mindre aktiviteter

%Ressursbehov er 20\% frikjøp i prosjekgruppen (50\% i tillegg til programledere) De andre komiteene leverer arbeider som ikke er for arbeidskrevende. 

%2023 

%Vår Overordnet struktur legges litt fastere, undervisningsformer.

%Ressursbehov. som over. Det er relativt usikkert hva ressursbehovet her er, ettersom dette er en relativt ny arbeidsmetode.

% Det kan være behov her for en gruppe som kan gå litt dypere, men det kan være vanskelig å få til på så kort varsel. Høsten 2022 bør det identifiseres evt. ressursbehov for høsten 2023. Må undersøke for raskt man kan få frigjort ressurser. 

%Høst

%Flere innholdsprosjekter

% Testing av formater underveis




	
	%\input{appendix/plan/project-gapanalysis2022}
	%\subsection{Visjon for studieprogrammene 2022}

Som et utgangspunkt for videre diskusjon av innholdet i studieprogrammet kan det være hensiktsmessig å få skrevet ned et sett med overordnede mål for studieprogrammene (MTFYMA og BFY). Det kan være nyttig som et grunnlag for å styre videre diskusjoner og arbeid med et mer detaljert innhold i studieprogrammene. Det utvikles også overordnede visjoner for de eksisterende studieretningene.

Sammen med visjonene utformes også en beskrivelse av hva som er forskjeller og likheter mellom programmene.

En ny versjon utvikles i løpet av høsten 2022 gjennom fase 1 i FTS-revisjonen.


% Langtidsplan

% Forankring

% Hvilket behov ser faggruppene for kunnskap og ferdigheter som er nødvendige for å kunne ta fag som faggruppen er 'ansvarlig' for i høyere årskurs. Hvilke ferdigheter er nødvendige for å ta masteroppgaver? Er det tidligere observert noen generelle mangler?

% Hvordan bør dette dokumentet struktureres?

\section{Ressursbehov}

% Oppsummering av ressursbehov

\chapter{Gapsanalyser}
\label{chap:gap-analysis}

Dette kapittelet inneholder en sammenlikning av studieprogrammene i forhold til ulike rapporter som har analysert hva som vil være ettertraktet kompetanse hos fremtidige studenter(\ref{sec:fts}-\ref{sec:evaluation2020}). I tillegg er det gjort et forsøk på et sammendrag av tiltak basert på gapsanalysene i \ref{sec:changes}.

\section{Forslag til endringer}
\label{sec:changes}

Her følger en punktliste over foreslåtte endringer i studieprogrammene baserte på gapsanalysere i forhold til FTS, dybdeevalueringen (se senere deler i dette kapittelet) og oppsamlede erfaringer som er gjort med dagens studieprogram.

Tilsammen vil mengden av de punktene man mener er fornuftige på denne listen diktere om det er nødvendig med en omfattende omlegging eller mindre justeringer.

Listen er ikke prioritert.

\begin{itemize}
	\item Innholdet i studieprogrammet beskrives i et Hoveddokument.
	\item IMF og IFY må ha et mye tettere samarbeid om innholdet i emnene slik at man oppnår synergier og ikke bare suboptimalisering ved å ha et felles studieprogram.
	\item Studenter ved BFY og MTFYMA kan med fordel ha en dypere og bredere innføring i matematikk enn det som er felles for andre siv.ing. program.
	\item Det opprettes egne emner for å utvikle praktiske ferdigheter, spesielt innen eksperimentelt arbeid og beregninger, gjerne kombinert.
	\item Studentene må i større grad eksponeres for oppgaver som trener integrert kompetanse, eksempelvis som integrerer eksperimenter, simuleringer, prosjektstyring og fysikk fra flere områder.
	\item Det må være en helhetlig plan for opplæring og trening i numeriske metoder og algoritmer.
	\item Studentene må få opplæring og trening i verktøy som er viktige for effektivt å utføre beregningsorienterte oppgaver. Dette inkluderer: Versjonskontroll (git), package-managers (conda), enhetstesting (pytest), terminalvindu, linux, m.m
	\item Studentene må få god opplæring i verktøy som de forventes å benytte.
	\item Studentene må få trening i distribuerte utviklingsprosjekter.
	\item Studentene skal få opplæring og trening i å bruke HPC ressurser, parallellisering og optimalisering (kan muligens være valgbart).
	\item Studentene skal få trening i metoder for maskinlæring.
	\item Studentene skal få trening i å håndtere store, ustrukturerte datasett.
	\item Eksperimentelle ferdigheter skal struktureres rundt å kunne gjennomføre hele verdikjeden fra problemformulering til rapportering.
	\item Studentene skal få mye mer trening i rapportskriving enn i dag.
	\item Studentene må gis mulighet for å fordype seg i tema relatert til bærekraft. Studenter med fysikk og matematikk bør bygge på sin styrke i forståelse av fysisk modellering og evne til å forstå å bruke avansert matematiske metoder. Det må tydeliggjøres hvilken kompetanse relatert til bærekraft studentene bør utvikle.
	\item Det bør utformes tydelig emneløp som gir spesialiseringer innen ulike fagområder. Dette må kommuniseres godt til studentene.
	\item Det bør legges ressurser inn på obligatoriske emner som leder opp mot valg av spesialisering.
	\item Istedenfor EiT bør det innføres et emne hvor man tverrfaglig skal løse et gitt teknologisk problem.
	\item Innføring i samarbeid (ala EiT) bør komme i begynnelsen av studiene og studentene bør få trening i dette gjennom hele studiet.
	\item Innføring i teori bør i større grad gjøres i en kontekst slik at studentene tydelig ser verdien av teorien.
	\item Det antas at studenter ved MTFYMA og BFY har tatt fysikk 2. For de som ikke har tatt det eller trenger oppfriskning tilbys et oppfriskningskurs (digitalt eller fysisk).
	\item Emnene i første semester og år skal legge et spesielt fokus på å motivere studentene til videre studier og ha et bevist forhold til at å etablere et liv på universitetet kan være krevende for mange.
	\item Det skal være en bevist og dokumentert plan for når matematiske temaet introduseres i matematikkemnene og når det er behov for dem i fysikk emnene.
	\item Studieprogrammene må ha et tett samarbeid med industri for å kunne bruke dette til å kontekstualisere teorien samt å skaffe oppgaver som er mer sammensatte og åpne.
	\item Studieprogrammene og fagmiljø skal i fellesskap utvikle en felles forståelse for det læringsmiljøet man ønsker å etablere og de undervisnings- og vurderingsformer som gjør at man oppnår det.
	\item Studenten bør få god trening i ulike former for prosjektarbeid, bl.a. arbeidsdeling, ledelser, planlegging, risiko, samarbeid m.m.
\end{itemize}


\section{Fremtidens teknologistudium}
	\label{sec:fts}
	\subsection{FTS prinsippene}
\label{sec:fts-principles}
FTS har definert 10 prinsipper\footnote{\url{https://www.ntnu.no/fremtidensteknologistudier/prinsipper}} som NTNU har vedtatt skal være styrende for teknologi-utdanningene ved NTNU. Det gis her en kortfattet vurdering i hvilken grad MTFYMA og BFY oppfyller disse prinsippene. Dersom det eksisterer et gap mellom ønsket og nåværende situasjon gis det kortfattet forslag om hvilke tiltak som kan iverksettes for å redusere gapet\footnote{Denne gapsanlysen er et utkast og planlagt ferdigstilt i løpet av høsten 2022}.

\begin{quote}
	\textbf{Prinsipp 1:} NTNUs teknologistudier skal legge aktivt til rette for at kandidatene, med utgangspunkt i et solid faglig fundament, opparbeider helhetlig og integrert kompetanse, herunder bærekraftkompetanse og digital kompetanse på høyt nivå.
\end{quote}

Studieprogrammet gir studentene et omfattende faglig fundament i fysikk og matematikk, men det er en mangel på trening i å kunne bruke kunnskap og ferdigheter på tvers av flere kunnskapsområder (dvs. integrert kompetanse, e.g. matematisk modellering, multifysikk, eksperimenter, numerikk, prosjektstyring, etc.) og da spesielt i mer omfattende og åpne oppgaver.

Det eksisterer ingen overordnet plan for utvikling av digital kompetanse, noe som medfører at utfallet er usikkert, tilnærmingen fragmentert og nivået for lavt ettersom man ikke vet hva man kan bygge videre på.

Det er uklart hva som menes med \emph{bærekraftskompetanse} for MTFYMA og BFY. Det må defineres hva som ligger i dette for at man skal kunne vurdere om det er oppfylt.

\textbf{Forslag til tiltak:}
\begin{itemize}
	\item Det anbefales at det i større grad utformes læringsaktiviteter hvor studentene må bruke kunnskap og ferdigheter fra flere kunnskapsområder/ferdigheter. Dette vil være ganske omfattende tidsmessig så vil kanskje måtte gjøres i dedikerte emner.
	\item Det utarbeides en helhetlig plan over hvordan de digital ferdigheten i programmet utvikles.
	\item Det utarbeids en definisjon av hva man legger i bærekraftskompetanse og dernest en plan for hvordan studentene evt. kan oppnå denne kompetansen.
\end{itemize}

\begin{quote}
	\textbf{Prinsipp 2:} NTNU skal legge aktivt til rette for at kandidater fra teknologistudiene opparbeider tverrfaglig samhandlingskompetanse, og for at man over den samlede studentpopulasjonen får et mangfold i kunnskapsprofiler, samtidig som den enkelte student oppnår tilstrekkelig programfaglig dybde.
\end{quote}

Det er mulig å velge ulike kunnskapsprofiler i det nåværende studieprogrammene men det er i liten grad synliggjort overfor studentene hvordan flere emner kan settes sammen til en hensiktsmessig profil. Data på emnevalg indikerer at studentene ikke velger så bredt som man kunne ønske. Det er også i liten grad synliggjort for studentene hvordan de kan bruke emner fra andre institutt for å skape en helhetlig profil.

Utover EiT er det ingen planlagte elementer i studieprogrammet for å bygge tverrfaglighet. Det er usikkert om EiT i tilstrekkelig stor grad definerer en reell teknologisk problemstilling som et tverrfaglig team skal løse i praksis. 

\textbf{Forslag til tiltak:}

\begin{itemize}
    \item Planlegge og designe tydelige fagprofiler som synliggjøres for studentene.
	\item Lage emner med prosjekter hvor man er nødt til å bruke fagkunnskap fra ulike studieprogram for å komme frem til en løsning.
	\item Samhandlingskompetansen som introduseres i EiT bør nok også introduseres på et tidligere tidspunkt i studiene slik at man både kan nyttiggjøre seg den gjennom studiene og få mer trening.
\end{itemize}

\begin{quote}
	\textbf{Prinsipp 3:} Kontekstuell læring skal legges til grunn som gjennomgående pedagogisk prinsipp i NTNUs teknologistudier.
\end{quote}

Kontekstuell læring brukes i svært liten grad. Det er noe uklart hva slags kontekst som vil bidra til bedre læring. Er det primært en anvendt, realistisk kontekst som er viktig?

\textbf{Forslag til tiltak:}

\begin{itemize}
    \item Innhente eksempler fra industri som kan brukes som utgangspunkt for å presentere teori for studentene. Dette kan inkludere gjesteforelesninger og at problemstillingen inkluderes i øvinger og mindre prosjekter.
\end{itemize}

\begin{quote}
	\textbf{Prinsipp 4:} NTNUs teknologistudier skal benytte kunnskapsbaserte, studentaktive og engasjerende undervisnings- og vurderingsformer som er samstemt med utdanningenes overordnede kompetansemål, fremmer god læringskultur, og gir effektiv dybdelæring
\end{quote}

Undervisningsformer ved programmet er i stor grad basert på tradisjon heller enn en kunnskapsbasert tilnærming til hvordan læring fungerer. Ikke dermed sagt at dagens form er feil, men det må vurderes om dagens form bidrar til å oppnå det man ønsker med studieprogrammet.

Emnene er i stor grad preget av stofftrengsel, spesielt på grunnivå, noe som reduserer muligheten for dybdelæring og gir redusert mestringsfølelse, noe som reduserer engasjement og motivasjon.

\textbf{Forslag til tiltak:}

\begin{itemize}
	\item Tydelig plan om innhold i emner på programnivå for å hindre stofftrengsel og forbedre samkjøring.
	\item Forsterke emnegruppene på IFY for å skape mer dialog om undervisningsformer.
	\item Gjør beviste valg av undervisnings- og vurderingsformer slik at det bidrar til god læringskultur.
\end{itemize}

\begin{quote}
	\textbf{Prinsipp VI:} Kvaliteten i NTNUs teknologistudier skal utvikles gjennom en programdrevet tilnærming, i kombinasjon med strategisk porteføljeutvikling og -forvaltning på tvers av programmer og programtyper.
\end{quote}

Studieprogrammene er i svært liten grad programdrevet ettersom det i liten grad eksisterer noen skriftlig dokumentasjon på hva innholdet skal være. Det lille som eksisterer i beskrivelsen av studieprogrammet er for vag og brukes i liten grad. Emnebeskrivelsen kan fritt endres av faglærer og er ikke styrt av programmet. Faglærerer tenker i liten grad utover sitte eget emne og i liten grad på hvordan det bidrar til helheten når de underviser.

Det er i liten grad tenkt på hvordan emnene henger sammen og bygger videre på hverandre. Eventuelle koblinger forvitrer også som følge av den enkelte faglærers frihet til å endre form og innhold i emnene.

Undervisning sees nok fortsatt på som et individuelt ansvar for et gitt emne og ikke et kollektivt ansvar. Dette er kanskje noe i endring på IFY som følge av etablering av emnegrupper

\textbf{Forslag til tiltak:}

\begin{itemize}
	\item Utvikle dette dokumentet slik at det dokumenterer en felles forståelse av hvordan studieprogrammene bør drives.
	\item Økt fokus emnegrupper på IFY for å styrke den kollektive tankegangen om undervisning.
\end{itemize}

\begin{quote}
	\textbf{Prinsipp VII:} NTNUs kvalitetsarbeid i teknologistudiene skal stimulere studieprogrammenes utvikling mot utdanningskvalitet i verdensklasse ved å fokusere på kontinuerlig forbedring og systematisk utvikling av kvalitetskultur.
\end{quote}

NTNUs kvalitetssystem har to store mangler. For det første er det primært et rapporteringssystem, men sirkelen er i liten grad sluttet ved at det også lages handlingsplaner som skaper endring. I tillegg er emneevaluering på emnenivå tungt basert på hva studentene \emph{syns}, og i liten grad på hva de faktisk lærer. 

Det er noe benchmarking internasjonalt og interessegrupper i forbindelse med dybdeevalueringer. Dette kunne kanskje styrkes ved å ha fast partnere som man bruker som benchmarking over tid. 

Man bør også kanskje ha blikket mer kontinuerlig rettet utover for å følge med på utviklingen internasjonalt, heller enn å kunne få et blikk på sin egen utdanning en gang hvert femte år.

\begin{quote}
	\textbf{Prinsipp VIII:} NTNU skal gi høy prioritet til strategisk og operativt internasjonalt samarbeid om utvikling av teknologistudier, med mål om å bli et internasjonalt synlig og anerkjent universitet også på dette området.
\end{quote}

Programmene er liten grad aktive i fora hvor det diskuteres undervisning av ingeniører og realister internasjonalt. IMF har vært mer aktive her. Det bør vurdere om man har kapasitet og om det vil være hensiktsmessig å øke denne aktiviteten.

\begin{quote}
	\textbf{Prinsipp IX:} NTNUs teknologistudier skal vektlegge systematisk samhandling med arbeidsliv og samfunn, med mål om å fremme arbeidsrelevans, legge til rette for livslang læring, og sikre at studenter kan opparbeide relevant arbeidslivserfaring gjennom studiene.
\end{quote}

Studiene har i liten grad samarbeid med eksternt arbeidsliv i studiene.

Et unntak er medisinsk fysikk hvor ansatte på sykhuset er tett integrert i utforming og gjennomføring av undervisning i disse områdenee.

\subsection{Prinsipper primært for andre nivå}

Prinsipp V og prinsipp X er primært relevant for henholdsvis institutt og sentralt nivå. Disse prinsippene er.

\begin{quote}
	\textbf{Prinsipp V:} NTNU skal stille tydelige forventninger til, og gi solid støtte for, kompetanseutvikling hos undervisningspersonell.
\end{quote}

\begin{quote}
	\textbf{Prinsipp X:} NTNU skal utvikle sitt læringsmiljø – og spesielt sin campus og infrastruktur (både fysisk og digital) – i en retning som understøtter de øvrige FTS-prinsippene I – IX, og som fremmer læring, helse og trivsel blant studenter og ansatte.
\end{quote}


	\subsection{FTS kompetansemål}
\label{sec:fts-competencies}
FTS delrapport 1\footnote{\url{https://www.ntnu.no/fremtidensteknologistudier}} introduserer begrepet kompetanseprofil for et studium og beskriver 12 kompetansemål som tilsammen gir en kompetanseprofil som skal kjennetegne studenter fra et teknologistudium ved NTNU.

Her følger en analyse av eventuelle gap mellom ønsket kompetanseprofile (i følge FTS) og nåværende studieprogram\footnote{Foreløpig bare et utkast, planlagt ferdigstilt i samarbeid med fagmiljøene i løpet av høsten 2022}.

\subsubsection{K1: Vise fagkunnskap og faglig fundert perspektiv}

Dette utgjør stammen i studieprogrammets innhold og er et omfattende punkt å detaljere. Rapporten foreslår en måte å strukturere innholdet og legger frem noen overordnede aspekter som er viktig. Fagmiljøene og studieprogrammene må spesifisere hva dette betyr i praksis og hvordan det konkret skal implementeres.

Dagens kandidater ved MTFYMA og BFY opparbeider uten tvil en bred faglig kunnskapsbase. Likevel utdypes dette punktet litt mer i detalj og i forhold til noen av disse områdene er det et vist gap mellom ønsket og dagens situasjon.

For eksempel så presiseres det at det som søkes er dyp, virksom kunnskap som søkes, slik at kandidaten er i stand til å \emph{bruke} kunnskapen kreativt og effektivt i problemløsning. Det kan stilles spørsmålstegn ved om dagens eksamensfokus tester kunnskapen relativt overfladisk og over et kort tidsrom, samt at kandidatene får lite erfaring i å \emph{anvende} kunnskapen. Det vurderes i liten grad om kunnskapen på sikt er dyp og virksom.

\begin{comment}
Delrapport 1 deler opp dette punktet i 4 deler

\begin{itemize}
	\item basiskunnskap
	\item breddekunnskap
	\item dybdekunnskap
	\item kompelementær kunnskap
\end{itemize}

generaliserbare konsepter.
kontekstualisering
beregningsorientering
stordata, maskinlæring
foretningsforståelse, invoasjonsprosesser

ii) bredde teknisk: prosjektledelse, muliggjørende teknologier

iii) vei dybde vs bredde

iv) komplementært i forhold til fremtidens behov

-bredde i kunnskapsprofiler

Steam - kreativitet
\end{comment}

\subsubsection{K2: Analyse av komplekse problemstillinger og systemer}

I dette punktet fremheves (for masternivået) at dette skal omfatte \textquote{\emph{selvstendig} problemformulering og analyse} samt at problemene skal være umedgjørlige. Dagens program har muligens et for stort fokus på å løse ferdig oppstilte problemer som har et entydig svar. Man kan også vurdere om det er for lite krav til at studentene er selvstendige aktører. Selv på masteroppgaver møter kandidatene ofte ferdige problemstillinger og ferdige eksperimentelle oppsett som skal brukes.

\subsubsection{K3: Design og implementering av bærekraftige løsninger}

Det er ikke bærekraftige løsninger som er det sentral i dette punktet men evnen til å design og implementere løsninger på problemer; men at disse løsningen skal vurderes i en samfunnsmessig/bærekraftig kontekst.

Utover et prosjekt i Instrumentering ser det ut som studentene får lite trening i dette. Målet må være at studentene får trening i å finne løsninger som representerer det settet med avanserte kunnskaper og ferdigheter de har.

\subsubsection{K4: Benytte relevante metoder og verktøy}
Det sentrale innholdet i dette punktet er at kandidatene skal beherske metoder og verktøy som gjør dem i stand til å \emph{effektivt anvende} sin fagkunnskap for å finne løsninger på problemer. Rapporten fremhever særskilt noen områder:

\begin{itemize}
	\item Håndtere store, ustrukturerte datasett.
	\item Maskinlæring.
	\item Digital sikkerhet og dømmekraft.
	\item Prosjektplanlegging.
	\item Effektiv kommunikasjon.
	\item Sensorteknologi, automatisering.
\end{itemize}

Generelt for MTFYMA og BFY legges det for stor vekt på teoretisk kunnskap og for lite på verktøy og metoder som er nødvendig for å effektivt bruke denne kunnskapen. Alle punktene i listen over er relevante (kanskje med unntak av digital sikkerhet) og har rom for forbedring. Flere slike metoder er også fremhevet i dybdeevalueringen fra 2020. Det anbefales at programmene, spesielt MTFYMA legger større vekt på opplæring i relevante metoder og verktøy for å anvende kunnskapen og at studentene får tid til å opparbeide erfaring.

\subsubsection{K5: Konsekvensanalyse, risikovurdering og scenariotenkning}
Utvikling av disse ferdighetene krever at studentene eksponeres for problemer hvor slike analyser er relevant. På lokal skala kan dette innebære å måtte gjøre beviste valg av metodikk for å besvare et problem selv når beslutningsgrunnlaget er usikkert. For å utvikle disse ferdighetene på større nivå (samfunn, økonomi, klima) er det nok nødvendig med realistiske problemstillinger, noe som vil kreve at man har et samarbeid med industri.

\subsubsection{K6: Kjenne til forskning og bidra til teknologiutvikling}
Å kunne å bidra til forskningsprosjekter samt lede utviklingsprosjekter er trolig sentrale arbeidsoppgaver for kandidater fra MTFYMA og BFY. Det første er til en hvis grad dekket av masteroppgaven. Det andre er i stor grad fraværende. Det bør tydeliggjøres hva som er forventet nivå på disse punktene innenfor dagens studium.

\subsubsection{K7: Innhente og kritisk vurdere informasjon}
For studenter ved MTFYMA og BFY anses det som spesielt viktig at de skal være i stand til å vurdere kvaliteten på forskningsresultater og konklusjonene som blir trukket basert på dem. Det virker å være veldig lite innslag av dette i studiet per i dag.

\subsubsection{K8: Livslang læring}
Dette punktet peker på at studiet ikke bare skal gi studentene kunnskap og ferdigheter men også skal gi dem effektive strategier for læring. Den velkjente frasen fra studenter om at det man lærte mest ved studiet var å lære, er nok en sannhet med modifikasjoner. Mange studenter bruker nok ganske ineffektive metoder for læring. Punktet omhandler også å utvikle en positiv holdning til omstilling og evne til å vurdere sine egne ferdigheter.

Per i dag er det ingen bevist tilnærming ved studiene for hvordan man sørger for at studentene tilegner seg hensiktsmessige læringsstrategier. Det bør nok utvikles.

\subsubsection{K9: Bruk og refleksjon over normer og helhetstenkning rundt etikk og bærekraft} 
I arbeidslivet vil studentene møte problemstillinger som ikke har en ren teknologisk løsning. Det bør vurderes om studentene skal få mer trening i å angripe slike problemstillinger i løpet av studiet.

\subsubsection{K10: Målrettethet, samhandlingsevne og lederskap}

Studentene bør få trening i alle disse ferdighetene gjennom hele studiet.

\subsubsection{K11: Kommunikasjon, formidling og dialog}

Studentene bør få trening i alle disse ferdighetene gjennom hele studiet.

\subsubsection{K12: Nyskaping}

Dette er trolig vanskelig å realisere i de sentral emnene i studiet. Det er også et spørsmål om alle skal ha kompetanse på dette eller om det bør være en mulighet for dem som er spesielt interessert. Dette kan for eksempel løses gjennom en pakke med emner som omfatter teknologiledelse, k/p-emner og i-emner.


%Dybdeevaluering 2020
\section{Dybdeevaluering 2020}
\label{sec:evaluation2020}

En ekstern komite og en studentkomite gjennomførte i 2020 en evaluering av studieprogrammene MTFYMA og BFY. Evalueringen fokuserte på \emph{eksperimentelle ferdigheter} og \emph{numeriske ferdigheter}. Hensikten var å dekke beregningsorientering av både fysikk og matematikk men på grunn av misforståelser av mandatet fokuserte evalueringene primært på fysikkemnene.

Her følger en oppsummering av anbefalinger fra rapporten, tolket av studieprogramledelsen.

\subsection{Generelt}

\begin{itemize}
	\item Studentene bør eksponeres for oppgaver som integrerer beregningsorientering og eksperimenter, samt få trening i å vurdere om spørsmål besvares best med eksperimenter, beregninger eller ny teori eller en kombinasjon av disse.
	\item Studentene bør selvstendig eller i team kunne levere på den praktiske gjennomføringen og må dermed få opplæring i og erfaring med hele verdikjeden fra problemformulering til rapportering.
	\item Studentene bør få erfaring med å jobbe i prosjektform hvor man jobber individuelt i et team og må planlegge hensiktsmessig arbeidsdeling. Studentene bør også få prosjektoppgaver hvor de jobber individuelt.
	\item Studentene bør få en tydelig innføring i normer og regler relatert til sitering, plagiat, IPR m.m.
\end{itemize}

\subsection{Eksperimentelle ferdigheter}

\begin{itemize}
	\item Studentene bør få mer trening i rapportskriving og erfaring med at rapporter kan ha ulike format i ulike situasjoner, ikke nødvendigvis bare som vitenskapelig artikkel. Det bør være felles læringsressurser og man må ha kontroll på at alle studieretningen får tilstrekkelig god trening i rapportering.
	\item Studentene må få en systematisk innføring i dokumentasjonspraksis, journalføring, dataintegritet m.m. Dette gjelder både eksperimentelt og beregningsorientert arbeid. Det bør refereres til internasjonale standarder om dette. Studentene bør også eksponeres for elektronisk journalføring, ikke kun håndskrevne. Studentene bør kjenne og forstå FAIR prinsippene for vitenskapelig arbeid og generelt om tankesett og teknologi relatert til open science.
	\item Studiet bør prioritere generiske laboratorieferdigheter foran demonstrasjoner for å støtte opp om teori. Laboratoriundervisning kan ha ulike målsetninger og det bør være tydelig for en gitt undervisningsaktivitet hva som er hensikten.
	\item Det bør være mindre bruk av ferdige oppsett av eksperimenter og mer fokus på at studenten må gjennomføre hele verdikjeden fra problemformulering til rapportering. Spesielt å kunne formulere hypoteser som kan testes av eksperimenter eller numeriske beregninger. Dette vil også innebære mer fokus på å kunne redegjøre for antagelser som er gjort i en analyse. Dette er en forutsetning for meningsfylt rapportskriving.
	\item For å kunne designe eksperimenter forutsetter det at studentene kan bruke sentrale måleinstrumenter og har kunnskap om sentral måleprinsipper. Metoder for automatisk datainnsamling (av store data) er spesielt viktig.
	\item Aspekter relatert til eksperimentdesign som studenten må få trening i: vurdere nødvendig nøyaktighet, vurdere ulike oppsett, identifisere og kvantisere feil/usikkerhet, kalibrering, avpasse oppsett i henhold til tid/ressurser,
	\item Studentene bør ha kjennskap til en bredt spekter av metoder for å analysere data og trening i å anvende disse.
	\item Det bør være egne emner for eksperimentell aktivitet.
	\item Risikoanalyse bør ikke kun inkludere HMS men også prosjektrisiko.
\end{itemize}

\subsection{Beregningsorienterte ferdigheter}

\begin{itemize}
	\item Studentene bør få en systematisk innføring i dokumentasjonspraksis, versjonskontroll og deling av kode.
	\item Studentene bør få trening i distribuerte utviklingsprosjekter hvor flere arbeider med deler av et større prosjekt.
	\item Studenten bør få trening i enhetstesting.
	\item Det bør gis en helhetlig innføring i numeriske beregninger og ferdighetsstrengene anses som et positivt virkemiddel for å nå dette målet.
	\item Studentene bør få kompetanse på maskinlæring og bruk av HPC ressurser.
	\item Studentene bør ha kjennskap til optimalisering og parallellisering
	\item Ulikhet i obligatoriske emner mellom MTFYMA og BFY gjør det vanskelig å bygge på tidligere emner, noe som er en forutsetning for at høyere grads emner er tilstrekkelig avanserte.
	\item Studentene bør kunne håndtere store, uorganiserte data.
	\item Studentene bør få trening i å bruke bereningsmetoder mot multifysikk-problemer.
	\item Studentene bør få trening i å bruke både kommersielle og åpne programpakker.
	\item Studentene bør kjenne, forstå og kunne anvende sentral numeriske algoritmer/metoder og forstå deres begrensninger.
	\item Studentene bør få trening i algoritmisk tekning.
	\item Studentene bør få erfaring med programvare for symbolske beregninger.
	\item Studentene bør ha erfaring med både skriptede og kompilerte språk.
\end{itemize}

\subsection{Studentevaluering}

\begin{itemize}
	\item Det bør gis spesifikk opplæring i ferdigheter som studentene forventes å kunne beherske eller kjenne til. Dette inkluderer:
	\begin{itemize}
		\item Rapportskriving
		\item Journalføring
		\item LaTeX
		\item Feilanalyse
		\item Relevante biblioteker i Python
	\end{itemize}
	\item BFY bør kanskje ha obligatoriske emner med mer selvstendig laboratoriarbeid.
	\item Mer relevant ITGK (ITGK blir lagt om høsten 2022 med mer fokus på numerikk.)
\end{itemize}


\chapter{Eksterne rapporter}
\label{chap:external-reports}

Denne delen innholder referanser til og vurderinger av rapporter som er utformet av organer utenfor studieprogrammene eller instituttene. For hver rapport skal det som minimum gis en kort oppsummering av innholdet og i hvilken grad det er relevant for studieprogrammene. Spesifikke elementer kan fremheves dersom det anses som hensiktsmessig.


\section{A European specification for physics bachelor studies}

Dette dokumentet beskriver kjennetegn ved bachelor utdanning i fysikk i en europeisk sammenheng. Hensikten med dokumentet er å følge opp Bolognaprosessen ved å ha en felles europisk spesifikasjon av et bachelor utdanning slik at det skal bli enkelt å gå videre på masterutdanning i det europeiske området.

Dokumentet gir en kort kvalitativ beskrivelse av kjennetegn på innhold og format på fysikkutdannelser i det europiske området og gir noen generelle anbeflinger både om form og innhold. I vedlegget er det i en tabell gitt en oversikt over emner som bør dekkes og omtrentlig vekting av de ulike områdene. Dokumentet kan fungere godt som en sjekkliste og minimum av det som bør dekkes. 

Dokumentets er fra 2009 og er i hovedsak basert på tidligere evalueringer (TUNING, EUPEN, STEPS). Lenker til flere av disse prosjektene er ikke lengre tilgjenglige. I tillegg virker det som mye av innholdet er basert på det subject benchmark statement fra det britiske QAA. 
:



\section{AAPT Computational physics}

(Nedtegnet 16.11.2022)

Rapporten "AAPT Recommendations for computational physics in the undergraduate physics curriculum" (\url{https://www.aapt.org/resources/upload/aapt_uctf_compphysreport_final_b.pdf})  er fra 2016 og er i tråd med andre anbfalinger samt nåværende målsetninger til programmet. Den fremhever at \textit{computational physics} er en tredje pillar i fysikk, på linje med teori og eksperiment.

Rapport en gir et rammeverk for hvordan beregningsorientert ferdigheter kan systematiseres. Det diskuteres også en rekke elementer som er viktig å ta med seg. Det meste av dette ligger allerede inne i studieprogrammenes planer.

Noen aspekter som ble fremhevet som muligens ikke er like tydelige i programmenes nåværende planer

\begin{itemize}
	\item Generelle programmeringsferdigheter og beregningsorientering bør utvikles i samme kontekst heller en separat
	\item Ferdighetene skal være 'authentic to the discipline'
	\item 'Best thaught in a lab setting' (i motsetning til typisk teorikurs).
	\item "How to teach computationl physics is anecdotal". Det er altså rom og behov for å gjøre forskning på endringer som gjøres i programmene våre.
	\item Det er flere 'communities' som jobber med beregninsorientering. Det er kanskje lurt om studieprogrammene og/eller fagmiljøene er involvert i disse.
\end{itemize}

Rapporten gir en liste over fysikk-tema som kan egne seg for beregningsmetorder.

Av spesiell interesse er kanskje en liste over potensielle organisatoriske utfordringer som det er viktig å ta hensyn til (se VI. Curricular issues og VII. Challenges).
Rapporten gir referanse til flere artikler som kan gi enda mer mer bakgrunn, samt noen bøker som er gått litt mer grundig til verks når det gjelder implementering av beregningsorientering.


\section{Curriculum guidelines for undergraduate programs in statistical science}

Rettningslinjene er publisert av American statistical association in 2014\cite{asa-guidelines2014}. 

Rapporten trekker fram nødvendigheten av tilstrekkelig bakgrunn innen nøkkelferdighetene:

\begin{enumerate}
	\item Statistiske metoder og teori
	\item Dataforvaltning
	\item Beregninger
	\item Matematisk fundament
	\item Anvendelse av statistikk (kommunikasjon og tekniske ferdigheter)
\end{enumerate}

Noen sentrale anbefalinger fra rapporten:

\begin{itemize}
	\item \emph{Data science} blir viktigere og viktigere. Forberedelse til karrierer i statistikk og data science krever (i tillegg til tradisjonelle matematikk/statistikk-ferdigheter) at man håndterer høynivå programmering og databasesystemer. Store og ustrukturerte datasett krever metoder for å finne mønstre og sammenhenger i høy-dimensjonelle datasett, og metoder for å unngå bias fra slike data. Dette er et datadrevet perspektiv med mindre fokus på hypoteser og statistisk signifikans.
	\item Virkelige anvendelser er viktige. Data burde være en nøkkelkomponent av statistikkurs. Studieprogram burde vektlegge konsepter og metoder for å jobbe med komplekse data, gi erfaring med design av studier, og det å analysere ikke-tekstbok data.
	\item Eksponering mot varierte metoder og fremgangsmåter. Studenter må eksponeres mot forskjellige prediktive og forklarende modeller i tillegg til metoder for modellering og evaluering. Studenter må forstå utfordringene knyttet til design, \enquote{confounding}, og bias. Studenter må kunne være i stand til å bruke sitt teoretiske fundament til fornuftig og solid dataanalyse.
	\item Evne til kommunikasjon er viktig. Studenter må kunne kommunisere komplekse metoder i enkel terminologi til ledere og variert publikum. Studenter må ha evne til å visualisere resultater på en lett forståelig måte.
	\item Studenter som skal gjøre PhD-studier trenger en sterk bakgrunn i matematikk og teoretisk statistikk.
\end{itemize}


\fi % end if statemnt for toggle appendix

\printbibliography

%section{CDIO}
% Gap analysis relative to CDIO standards

% \section {IOP, APS, ...}
% Gap analysis relative to professional societies

% \chapter{Organisasjon}
% Describe the organization that is necessary to maintain this do:cument and ensure its continued use.

% Hvordan sørge for at prosjektet ikke er personavhengig.

% \chapter{Dokumenthåndtering}
% Description of how the document is manged, updated, communicated.
% Plasseres på github fordi ...., github bruker til ... overføres når...



\end{document}
