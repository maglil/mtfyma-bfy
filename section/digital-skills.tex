\section{Digitale ferdigheter}

Ettersom digitale ferdigheter er sammensatt av mange komponenter er det nyttig å få systematiserte ferdighetene for å få oversikt og som et verktøy for å legge opp studiet. Her følger et forslag til overordnet inndeling:

\begin{itemize}
	\item Vitenskapelige ferdigheter
	\item Generelle IT-ferdigheter
	\item Generelle programutviklingsferdigheter
	\item Språk
	\item Algoritmer
	\item Programvare
	\item Samarbeid og kommunikasjon
\end{itemize}

De neste delene gir en ytterligere detaljering av ferdighetene under hvert kompetanseområde:

\subsection{Vitenskapelige ferdigheter}
\begin{itemize}
	\item Omsette en matematisk/fysisk modell til kode
	\item Redegjøre for antagelser, forenklinger
	\item Tolke numeriske beregninger, fysisk vurdere om resultaten er realistiske
	\item Hensiktsmessig rapportering og presentasjon
\end{itemize}

\subsection{Generelle IT ferdigheter}
\begin{itemize}
	\item Linux
	\item Kommandolinjeverkøy
	\item Scripting, automatisering, batch-processing
	\item Package managers (conda, pip)
	\item Software stacks
	\item Presisjon, numeriske feil, konvergens
	\item Pseudo-random numbers
	\item Dimensjonløs representasjon
	\item Hardware (minne, lagring) 
	\item Parallellisering
	\item HPC ressurser
	\item Prosessere data, ulike dataformater, uorganiserte, mangelfulle data.
	\item Søke i og lese teknisk litteratur.
	\item Effektiv visualisering
	\item (Datasikkerhet)
\end{itemize}

\subsection{Generelle programutviklingsferdigheter}
\begin{itemize}
	\item Generelle elementer i programmeringsspråk (variabler, kontrollstrukturer osv.). Dekkes av ITGK.
	\item IDE (vscode)
	\item Dokumentasjon av kode
	\item Versjonskontroll
	\item Enhetstesting, validering (analytisk løsning, method of manufactured solution)
	\item Debugging
	\item Style guides
	\item Strukturering av kode
	\item Effektivisere kode
	\item \enquote{Computational cycle}
\end{itemize}

\subsection{Språk}
\begin{itemize}
	\item High level language: Python (Numpy, Matplotlib, SciPy, Pandas, Seaborn, Scikit-learn, Tensorflow, Requests, Pytest)
	\item Low level language: C++
	\item Symbolic computing: Python/sympy
	\item Markup: LaTeX, Markdown.
	\item (og/eller R, Matlab, C, Julia, Haskel, Fortran...)
\end{itemize}

\subsection{Algoritmer}
\begin{itemize}
	\item Numerisk integrasjon
	\item ODE
	\item PDE (elementmetoder)
	\item Linear algebra
	\item Optimization (root-finding, lineær programming)
	\item Transformasjoner (fourier, wavelet)
	\item Stokastiske metoder (Monte Carlo, metropolis)
	\item Statistikk
	\item Kurvetilpasning
\end{itemize}


\subsection{Programvare}
\begin{itemize}
	\item Office-pakken
	\item (Comsol multiphysics, Ansys, Zeemax optic studio, Openfoam, Molsim)
\end{itemize}


\subsection{Samarbeid og kommunikasjon}
\begin{itemize}
	\item Planlegge og gjennomføre utviklingsprosjekter
	\item Samarbeid om programutvikling
\end{itemize}

\subsection{Pedagogisk-stratgiske valg}

\begin{itemize}
	\item All aktivitet skal være så autentisk som mulig.
	\item All koding gjøres i rene tekstfiler. Jupyter notebooks kan brukes til presentasjon av teori og til å øve på enkle programmeringselementer.
	\item Fungerende hardware og software er studentens ansvar. De får hjelp så langt ressursene strekker til, men om ikke studenten får maskinen til å fungere betyr det at det ikke består aktiviteten.
	\item Alle ferdigheter som studenten skal oppnå, undervises og kobles til konkrete læringsaktiviteter.
	\item Prosjekter skal bidra til motivasjon og kreativitet,
	\item Matematiske algoritmer bør gjenbrukes på flere anvendelser slik at studentene for en forståelse av deres generelle anvendbarhet.
	\item Må gjøre en avgrensning mellom kjerneelementer som alle må kunne og spesialisering som noen studenter kan velge å fordype seg i.
	\item Det bør være en holistisk og programstyrt tilnærming som går på tvers av mange emner og gjennom hele programmet. 
	\item Det er en balansegang mellom generelle algoritmer,  algoritmisk tenkning (computational thinking) og det å få opplæring i konkrete verktøy.
	\item Må tenke inkrementelt, men viktig at computational physics/mathematics are helhetlige fagfelt og denne helheten må også komme frem.
	\item Må få oppgaver som krever ulike verktøy (python vs c++, linux, cluster) for å sikre at de får trening i alle.
	\item Et viktig element er å gjøre studentene i stand til å lære seg nye ting. Kanskje er det de pratiske, ofte uuttalte ferdigheten som er viktigst å gi studetene.
	\item Skal alle få innsikt i quantum computing? Minst noen?
	\item Maskinlæring er viktig men likevel bare en komponent av mange.
	\item Egen ITGK for BFY?
	\item Kompetanse hos fagmiljøet må bygges opp.
	\item Bør ha en forståelse for software engineering. Forstå at større prosjekter trenger annen tilnærming enn mindre prosjekter.
	\item Innholdet i grunnskolen er i endring. Følge med på dette.
	\item Visualisering og presentasjon er veldig viktig
	\item Testing er ekstermt viktig
	\item Prosjektemner gjør det mulig å fokusere på prosess, ikke bare sluttresultat.
	\item Studentene for en mye mer komplett verktøykasse om programmet styrerer innholdet enn om studentene skal google seg frem til de ferdighetene de trenger (ref IMFs PhD på transferable IT-skills.)
	\item Studentene må eksponeres for større prosjekter hvor ulike moduler må sys sammen og arbeidsdeling er viktig.
	\item Bevistgjøring av hva som er viktig å spare på. Hvordan ting dokumenteres for ettertiden. Metadata.
	\item Det er veldig stort spenn i ferdigheter hos studentene. Hvordan gi alle utfordringer?
\end{itemize}
