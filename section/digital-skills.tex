\section{Digitale ferdigheter}

Ettersom digitale ferdigheter er sammensatt av mange komponenter er det nyttig å få systematiserte ferdighetene for å få oversikt og som et verktøy for å legge opp studiet. Her følger et forslag til overordnet inndeling:

\begin{itemize}
	\item Vitenskapelige ferdigheter
	\item Generelle IT-ferdigheter
	\item Generelle programutviklingsferdigheter
	\item Språk
	\item Algoritmer
	\item Programvare
	\item Praktiske ferdigheter
\end{itemize}

De neste delene gir en ytterligere detaljering av ferdighetene under hvert kompetanseområde:

\subsection{Vitenskapelige ferdigheter}
\begin{itemize}
	\item Omsette en fysisk modell til kode
	\item Redegjøre for antagelser, forenklinger
	\item Tolke numeriske beregninger, fysisk vurdere om resultaten er realistiske
	\item Hensiktsmessig rapportering og presentasjon
\end{itemize}

\subsection{Generelle IT ferdigheter}
\begin{itemize}
	\item Linux
	\item Kommandolinjeverkøy
	\item Scripting, automatisering, batch-processing
	\item Package managers (conda, pip)
	\item Software stacks
	\item Presisjon, numeriske feil, konvergens
	\item Pseudo-random numbers
	\item Dimensjonløs representasjon
	\item Hardware (minne, lagring) 
	\item Parallellisering
	\item HPC ressurser
	\item Prosessere data, ulike dataformater, uorganiserte, mangelfulle data.
	\item Søke i og lese teknisk litteratur.
	\item Effektiv visualisering
	\item (Datasikkerhet)
\end{itemize}

\subsection{Generelle programutviklingsferdigheter}
\begin{itemize}
	\item Generelle elementer i programmeringsspråk (variabler, kontrollstrukturer osv.). Dekkes av ITGK.
	\item IDE (vscode)
	\item Dokumentasjon av kode
	\item Versjonskontroll
	\item Enhetstesting, validering (analytisk løsning, method of manufactured solution)
	\item Debugging
	\item Style guides
	\item Strukturering av kode
	\item Effektivisere kode
	\item \enquote{Computational cycle}
\end{itemize}

\subsection{Språk}
\begin{itemize}
	\item High level language: Python (Numpy, Matplotlib, SciPy, Pandas, Seaborn, Scikit-learn, Tensorflow, Requests, Pytest)
	\item Low level language: C++
	\item Symbolic computing: Python/sympy
	\item Markup: LaTeX, Markdown.
	\item (og/eller R, Matlab, C, Julia, Haskel, Fortran...)
\end{itemize}

\subsection{Algoritmer}
\begin{itemize}
	\item Numerisk integrasjon
	\item ODE
	\item PDE (elementmetoder)
	\item Linear algebra
	\item Optimization (root-finding, lineær programming)
	\item Transformasjoner (fourier, wavelet)
	\item Stokastiske metoder (Monte Carlo, metropolis)
	\item Statistikk
	\item Kurvetilpasning
\end{itemize}


\subsection{Programvare}
\begin{itemize}
	\item Office-pakken
	\item (Comsol multiphysics, Ansys, Zeemax optic studio, Openfoam, Molsim)
\end{itemize}


\subsection{Praktiske ferdigheter}
\begin{itemize}
	\item Planlegge og gjennomføre utviklingsprosjekter
	\item Samarbeid om programutvikling
\end{itemize}

\subsection{Pedagogisk-stratgiske valg}

\begin{itemize}
	\item All aktivitet skal være så autentisk som mulig.
	\item All koding gjøres i rene tekstfiler. Jupyter notebooks kan brukes til presentasjon av teori og til å øve på enkle programmeringselementer.
	\item Fungerende hardware og software er studentens ansvar. De får hjelp så langt ressursene strekker til, men om ikke studenten får maskinen til å fungere betyr det at det ikke består aktiviteten.
	\item Alle ferdigheter som studenten skal oppnå, undervises og kobles til konkrete læringsaktiviteter.
	\item Prosjekter skal bidra til motivasjon og kreativitet,
	\item Matematiske algoritmer bør gjenbrukes på flere anvendelser slik at studentene for en forståelse av deres generelle anvendbarhet.
\end{itemize}
