\section{Digitale ferdigheter}

Ettersom det digitale ferdigheter er sammensatt av mange komponenter er nyttig å få systematiserte ferdighetene for å få oversikt og som et verktøy når man skal legge opp studiet. Her følger en overordnet oversikt.

\begin{itemize}
	\item Vitenskapelige ferdigheter
	\item Generelle IT-ferdigheter
	\item Generelle programutviklingsferdigheter
	\item Språk
	\item Algoritmer
	\item Programvare
	\item Praktiske ferdigheter
\end{itemize}

De neste delene gir en ytterligere detaljering av ferdighetene som skal 

\subsection{Vitenskapelige ferdigheter}
\begin{itemize}
	\item Omsette en fysisk modell til kode
	\item Redegjøre for antagelser, 
	\item Tolke numeriske beregninger, fysisk vurderering av realisme
	\item Hensiktsmessig rapportering og presentasjon
\end{itemize}

\subsection{Generelle IT ferdigheter}
\begin{itemize}
	\item Linux
	\item Kommandolinjeverkøy
	\item Packe managers (conda, pip)
	\item Software stacks
	\item Presisjon, numeriske feil, konvergens
	\item Pseudo-random numbers
	\item Dimensjonløse tall
	\item Hardware (minne, lagring, 
	\item Parallellisering
	\item HPC ressurser
	\item Prosessere data, ulike dataformater, uorganiserte, mangelfulle data.
	\item Søke i og lese teknisk litteratur.
	\item Effektiv visualisering
	\item (Datasikkerhet)
\end{itemize}

\subsection{Generelle programutviklingsferdigheter}
\begin{itemize}
	\item Generelle elementer i programmeringsspråk (variabler, kontrollstrukturer osv.). Dekkes av ITGK.
	\item IDE (vscode)
	\item Dokumentasjon av kode
	\item Versjonskontroll
	\item Enhetstesting, validering
	\item Debugging
	\item Style guides
	\item Strukturering av kode
	\item \enquote(Computational cycle)
\end{itemize}

\subsection{Språk}
\begin{itemize}
	\item High level language: Python (Numpy, Matplotlib, SciPy, Pandas, Seaborn, Scikit-learn, Tensorflow, Requests)
	\item Low level language: C++
	\item Symbolic computing: Python/sympy
	\item Markup: LaTeX, Markdown.
	\item (og/eller R, Matlab, C, Julia, Haskel, Fortran...)
\end{itemize}

\subsection{Algoritmer}
\begin{itemize}
	\item ODE...
\end{itemize}


\subsection{Programvare}
\begin{itemize}
	\item Office-pakken
	\item (Comsol multiphysics, Ansys, Zeemax optic studio, Openfoam, Molsim)
\end{itemize}


\subsection{Praktiske ferdigheter}
\begin{itemize}
	\item Samarbeid om kode
\end{itemize}

\subsection{Pedagogisk-stratgiske valg}

All aktivitet skal være så autentisk som mulig.

All koding gjøres i rene tekstfiler. Jupyter notebooks kan brukes i undervisningsformål og til å øve på enkle programmeringselementer.

Fungerende hardware og software er studentens ansvar. De får hjelp så langt ressursene, men om ikke studenten får maskinen til å fungere betyr det at det ikke består aktiviteten.

Alle ferdigheter som studenten ønskes å oppnå skal undervises og kobles til konkrete læringsaktiviteter.

Prosjekter skal bidra til motivasjon og kreativitet
