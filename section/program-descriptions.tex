\chapter{Overordnede målsetninger}

Dette kapittelet beskriver de overordnede målsetningene med programmet. Hensikten med å beskrive målsetningene på dette nivået er at det gir et oversiktlig sett med målsetninger som kan fungere som rettesnor og gi en felles forståelse av hva vi ønsker å oppnå med programmet\footnote{Dette er bare et utkast. Det planlegges at felles målsetninger utarbeides i løpet av høsten 2022}.

En slik beskrivelse skal ikke være et minste felles multiplum som kan aksepteres enstemmig. Det vil være uenighet (og kontinuerlig diskusjon) om innholdet. De overordnede målsetningene representerer fagmiljøets majoritet, eksterne avtakere og føringer fra ledelse.

Utsagn som ikke virker begrensende for innholdet eller som for de fleste innen fagmiljøet er selvfølgelige, bør unngås.

\section{MTFYMA}

%Brynjulf
\begin{itemize}
	\item MTFYMA skal være et attraktivt studieprogram som tiltrekker seg elever fra videregående skole som har en spesiell interesse for fysikk og matematikk.
	% Hva ligger i å være attraktivt? Må man ha spesielt talent?
	\item MTFYMA skal ha klare kompetansemål for utdanningen som gjør våre kandidater attraktive for framtidige arbeidsgivere.
	\item MTFYMA skal være et integrert studieprogram med en egen tydelig identitet som i størst mulig grad er uavhengig av spesialisering i senere del av studiet.
	\item MTFYMA skal ha en tydelig profil innenfor fagområdene fysikk og matematikk som gjennom hele studieløpet kjennetegnes ved at det har en høyere målsetning innen disse fagene enn øvrige sivilingeniørprogrammer.
	\item MTFYMA skal være et umiskjennelig sivilingeniørprogram.
	\item MTFYMA skal utdanne kandidater som tilfører arbeidslivet kunnskaper om den til enhver tid nyeste metodikken som er tilgjengelig innen fysikk og matematikk for å arbeide med industrielle problemstillinger.
	\item MTFYMA vil etterstrebe et godt læringsmiljø for studentene, gode studieforhold, aktive undervisningsmetoder og ligge i front når det gjelder vurderingsformer og samstemt undervisning.
	\item MTFYMA skal ha et tett og godt forhold til industri og andre avtakere av våre studenter, lytte til deres ønsker og behov og være åpen for å modernisere og tilpasse studieprogrammet etter hva som er etterspurt.
\end{itemize}

%Magnus
\begin{itemize}
	\item MTFYMA gir, I tillegg til det teoretiske grunnlaget, nødvendige ferdigheter i metoder og verktøy som gjør kandidatene til effektive problemløsere.
	\item MTFYMA skal gi studentene trening og trygghet i å anvende sammensatt kunnskap og ferdigheter for å løse komplekse problemstillinger.
	\item MTFYMA er et studieprogram som i stor grad er programmert (obligatoriske emner) for å effektivt oppnå ønsket kunnskap og ferdigheter.
	\item MTFYMA gir i tillegg til det faglige fundamentet også kunnskap og ferdigheter som er viktige som ansatt i en bedrift.
\end{itemize}

\section{BFY}
\begin{itemize}
	\item BFY gir tilstrekkelig valgfrihet mot slutten av studiet til å begynne spesialisering mot ulike fagfelt både innen og utenfor fysikk, og sørge for at studenten kan kvalifisere seg til opptak til disse studiene.
\end{itemize}
%Kombinasjonen BFY + MSPHYS gir mulighet for dypere spesialisering i fagområder innen fysikk, mens muligheten for kombinasjonen BFY + annet masterprogram, gir en bredde av kompetanseprofiler.

\section{BFY vs MTFYMA}

Denne delen spesifiserer forskjellene mellom BFY og MTFYMA og de ulike rollene til programmene.

\subsection{MTFYMA}
MTFYMA er et sivilingeniørprogram i fysikk- og matematikk. I tillegg til et bredt fundament i fysikk og matematikk legger studieprogrammet stor vekt på komplementære ferdigheter som er viktige for en ingeniør: praktiske ferdigheter, samarbeid, ledelse, økonomi m.m.  

Studieprogrammet er i høy grad programmert for å effektivt kun gi studenten de ferdighetene som ansees som viktige for en fremtidig ingeniør.

\subsection{BFY}
BFY gir et solid fundament i fysikk og støttende fagområder. Programmet gir mulighet til å utvikle unik kompetanse langs flere akser.

\begin{enumerate}
	\item Studieprogrammet gir mulighet til å spesialisere seg mot et gitt område i fysikk i siste halvdel av programmet. Sammen med videre fordypning på MSPHYS gir dette en mulighet til fordypning nær forskningsfronten og kan være et ideelt springbrett for videre forskningskarriere.
	\item IFY dekker kun en liten del av alle mulige fysikkområder. BFY gir mulighet for å ta en mastergrad ved en annen/utenlandsk institusjon, noe som vil øke nasjonal og regional kompetanse.
	\item BFY gir et godt grunnlag for mange masterprogram også utenfor fysikk. Siste halvdel av studiet kan evt. brukes til ta emner som kreves for opptak på disse studiene. En grad fra BFY sammen med en annen mastergrad (e.g. ingeniørretning, data, matematikk, pedagogikk), gir unik kompetanse.
\end{enumerate}

\subsubsection{Praktiske aspekter}

MSPHYS programmet er viktig for internasjonalisering av fysikkstudiet ved NTNU og dette kan være vanskelig å opprettholde uten BFY programmet.

\subsection{Beskrivelse av nåværende situasjon (2022)}
Studieprogrammene MTFYMA og BFY er begge fysikk programmer. Det første er et integhrert teknologisk 5-årig master program i fysikk og matematikk, mens BFY er et klassisk 3-årig bachelor program i fysikk. Hovedforskjellene mellom de to programmene ligger i at BFY gir studentene mer valgfrihet, hovedsaklig i 3 året samt at man etter står mer fritt til å velge seg hovedprofil for graden. Dessuten gir BFY en grad etter hva som tilsvarer normert 3-års studier. Dette åpner for muligheten for å gå ut i jobb, eller fortsette sine studier med en master grad i fysikk eller andre nærliggende fagfelt ved samme eller annen institusjon i inn og utland. Man har ikke automatisk~\footnote{Rent praktisk har man ofte løst dette ved at MTFYMA studenten blir tatt opp på BFY studiet og blir gitt muligheten til å sitte igjen med en grad etter 3 års. Dette fungere fordi man har et BFY program i fysikk og at fysikkdelen av de 3 første årene er ganske like mellom de to programmene. Merk at om man ikke hadde BFY programmet, ville MTFYMA studenter som forlater programmet før det er fullført, ikke kunne sitte igjen med en grad for sin utdannelse.} samme mulighet innenfor MTFYMA programmet. En annen prinsipiell forskjell mellom programmene er at MTFYMA, som et teknologi program, er underlagt FUS (Forvaltningsutvalget for sivilingeniørutdannelsen) som fremsetter flere krav til fellesemner. Disse fellesemnene er ikke nødvendige i BFY programmet (med mindre programrådet bestemmer dette). Resultatet er at BFY programmet er mer fleksible i hvilke ikke-fysikk fag som kan velges. 

Vi vil nå se nærmere på de konkrete forskjellene mellom de to studie programmene. I løpet av de to første årene, har begge programmene de samme obligatoriske fysikkemnene, mens matematikk delen er idag er noe forskjellig. De to første årene har BFY 6 obligatoriske 7.5vt emner i matematikk/statistikk, det samme som er tilfellet for MTFYMA. Dog skal det nevnes det er tale om forskjellige matematikk emner. F.eks. inngår numerisk matematikk emnet \textit{TMA4320: Introduksjon til vitenskapelige beregninger} i MTFYMA, mens det ikke er obligatorisk for BFY selv om de kan velge dette.

Når det gjelder informatikk og numerikk-emner, er det større forskjeller mellom programmene. MTFYMA har 3 slike obligatoriske emner medregnet TMA4320: Introduksjon til vitenskapelige beregninger, mens BFY bare har \textit{TDT4110: Informasjonsteknologi, grunnkurs} som obligatorisk. Denne forskjellen er utfordrende for den numeriske aktiviteten i senere fysikkemner.

Når man kommer til 3. studieår begynner valgfriheten for BFY programmet å bli tydelig. Her har dette programmet ett obligatorisk fag i hvert av de to semestrene. De resterende 6 emnene er hovedsakelig fysikk emner og de kan velges fra en liste på 9 fysikk og 2 ikke fysikk (romteknologi) mener. Ett av disse valgbare emnene er en 15sp bachelor oppgave. I MTFYMA programmvil innholdet i 3. studieår variere med studieretningen man velger. Om man ser på retningen «Teknisk fysikk», som trolig ligger nærmest opp mot BFY programmet, har studentene ingen valgbare emner. De obligatoriske emnene består at 7 fysikk emner samt teknologiledelse. 


