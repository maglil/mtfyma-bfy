\section{Eksperimentelle ferdigheter}

\subsection{Læringsmål}

Eksperimentelle ferdigheter har to hovedmål:
\begin{enumerate}
	\item Det skal gjøre studenten i stand til å besvare teknologiske og vitenskapelige spørsmål gjennom eksperimenter. 
	\item  Det skal gjøre studenten i stand til å skape teknologiske løsninger som skaper verdi gjennom å manipulere energi, materie og informasjon. \footnote{Merk at \emph{teoretisk innsikt} ikke er et primært læringsmål. For at læringsaktivitetene skal ha et tydelig mål settes det skarpt organisatorisk skille mellom aktiviteter som skal gi teoretisk innsikt og aktiviteter som skal gi eksperimentelle ferdigheter, slik det er definert her. Praktiske aktiviteter kan likevel ha en verdi for å gi teoretisk innsikt, men slike aktiviteter organiseres i teori-emner. Eksperimentell trening vil også nødvendigvis gi teoretisk innsikt, men dette er en positiv konsekvens og ikke hovedmålsetningen.}
\end{enumerate}

For å oppnå disse overordnede målene må studentene utvikle følgende ferdigheter:

\begin{description}
	\item [Designe eksperiment] Dette inkluderer å konkretisere spørsmål til noe som kan besvares eksperimentelt, utforme hypotese, velge instrumentering, metrologi, kalibrere, teste, videreutvikle, HMS, RRI, m.m.
	\item [Modellere eksperimentet] Lage en matematisk representasjon av eksperimentet for å kunne tolke data, vurdere gyldighet m.m.
	\item [Gjennomføre eksperimentet] Lagre, dokumentere, metadata, FAIR m.m.
	\item [Analysere og tolke data] Filtrere, representere, kondensere, feilkilder, visualisere, sammenstille med modell, tolke resultatene og syntetisere det til nyttig kunnskap, m.m.
	\item [Kommunisere] Ulike format, åpen vitenskap, etisk.
\end{description}

Disse ferdighetene sammenfatter hovedmål 1). Hovedmåle 2) innebærer i stor grad samme prosesser men hvor designprosessen får større plass, man  designere mot spesifikasjoner istedenfor et spørsmål og at produktets kommersielle verdi får større vekt.

Noen sentrale pedaogisk-strategiske valg:

\begin{itemize}
	\item Alle læringsaktiviteter skal være så \emph{autentiske} som mulig. Eksempelvis så bør det benyttes instrumenter, elektronikk og metoder som er vanlige i forskning og teknologi (altså ikke instrumenter som er laget primært for pedagogiske formål). Studentene må være involvert i hele verdikjeden beskrevet nedenfor. Kommunikasjon i autentiske formater (ikke lab-rapport, masteroppgave).
	\item Ikke gruppearbeid men team-arbeid. Det er ikke effektivt eller vanlig at man i arbeidslivet har flere personer som gjør samme ting. Men man jobber ofte i team og fordeler arbeidsoppgaver.
	\item Studentenes evne til å gjennomføre hele verdikjeden må bygges gradvis. Dersom studenten ikke skal gjennomføre noen deler må det gjøres på en autentisk måte. Eksempelvis er det vanlig å \enquote{kjøpe tjenester} (for eksempel fra en kjernefasilitet) som kan inngå som en naturlig del i den eksprimentelle kjeden, men studentene må da være en tydelig \enquote{kunde} i prosessen. Et annet eksempel kan være å bruke åpent tilgjengelige data.
	\item For et gitt område, skal det benyttes felles læringsmateriell gjennom hele studieforløpet.
	\item Den teoretiske bakgrunnen for eksperimentet bør i så stor grad som mulig være kjent. Senere i studiene kan et større element av å måtte sette seg inn i nytt fagstoff før eksperimentet inkluderes, og at denne ferdigheten da er et eget læringsmål.
	\item Ettersom studenten selvstendig skal utføre elementene må man nødvendigvis starte med det som skal til for å kunne gjennomføre en måling. Så å si alle målinger i dag gjøres om til et elektrisk signal som lagres digitalt på en datamaskin. Det første studenten må lære for å kunne gjennomføre et eksperiment er å kunne sette opp en sensor, en transducer, en ADC og en datamaskin.
	\item Ingen av læringsmålene bør ha større eller mindre vekt på vurdering bare fordi de er enklere å vurdere (som eksempelvis en rapport har vært).
	\item Undervisningen deles grovt opp mellom 1) opplæring i de nødvendige elementene (e.g. spesifikke instrumenter, statistiske analyser, rapportformater) og 2) gjennomføring av eksperimenter.
	\item Eksperimenter som gjennomføres på datamaskiner organiseres i begynnelsen av studiene under digitale ferdigheter men dette skillet vil viskes ut senere i studiet hvor man typisk simuleringer og fysiske eksperimenter støtter opp om hverandre.
	\item Prosjektene blir gradvis mer komplekse. En mulig progresjon er: Individuelt arbeid i mindre prosjekter - team i mindre prosjekter - team i større prosjekter - prosjekter over flere studentkull/flere semestre - tverrfaglige prosjekter - industriprosjekter - internasjonale prosjekter.
\end{itemize}


