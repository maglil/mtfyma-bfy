\chapter{MSPHYS-spesialiseringer}
\label{c:msphys-spec}

MSPHYS bygger på en BSc fysikk, eller tilsvarende. BSc-graden gir en bred og solid bakgrunn i fysikk og man velger typisk en hovedretning med de valgbare emnene man har. I MSPHYS vil man typisk spesialisere seg videre innen denne hovedretningen.

De spesilaliseringen man til en hver tid tilbyr vil være bestemmes av fagmiljøets (og samarbeidende fagmiljøs) ekspertise og samfunnets behov. Siden en master oppgave (60 stp) er en sentral del av MSPHYS, vil veiledningskapasitet i fagmiljøet ved instituttet kunne sette begrensisniger for hvor mange studenter hver spesialisering kan håndtere.

Det bør bemerkes at om man sammenlikner størrelsen på den avsluttende oppgaven som er en del av MTFYMA (30 stp), er den tilsvarende oppgaven for MSPHYS dobbelt så stor (60 stp). Videre jobber i MSPHYS-graden ofte med oppgaven over tre semestere, mens for MTFYMA er den tilsvarende periden ett semster. Dette betyr at i MSPHYS er oppgavene ofte mer forskningnære da man har mer tid til å \enquote{finne ut av ting}.

Per nå tilbys følgende spesialiseringer for MSPHYS:

\begin{itemize}
	\item Astrofysikk
	\item Computational Science 
	\item Data Science
	\item Optics and photonics   
	\item Quantum technology and cryptography
	\item Material physics
	\item Atmospheric physics and climate change
\end{itemize}