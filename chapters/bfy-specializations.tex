\chapter{BFY spesialiseringer}
\label{c:bfy-spec}

Kandidater fra BFY har en bred og alsidig bakgrunn i fysikk samt støttefagene matematikk og informatikk. Denne bakgrunnen gjør dem ettertraktet i næringslivet samt for master-studier i fysikk og et utvalg av andre teknologiske og naturvitenskaplige fagfelt. 

BFY definerer et sett med faglige hovedretninger (spesialiseringer), bestemt av et sett med emner som danner et naturlig grunnlag for fortsettelse på en mastergrad. 
Disse hovedretningene vil være dynamiske over tid og bestemmes av fagmiljøets ekspertise.
I eventuelt påfølgende master-studier, kan man spesialisere seg ytterlgere innen den hovedretningen man har valgt i sin BSc-grad.
BFY graden kan designes, gjennom de emnevalgene man tar, til å være rettet inn mot næringlivet og på den måten representere en avsluttende og selvstendig grad.      

Følgende hovedretninger defineres:

\begin{itemize}
	\item Eksperimentell  fysikk  
	\item Numerisk fysikk (Computational physics)
	\item Fundamentalfysikk
	\item Astrofysikk
\end{itemize}

Spesielt de to første retnigene egner seg for å designes mot næringslivet som en avsluttende og selvstendig

I tillegg synliggjør programmet tydelige emnevalg som kvalifiserer for masterprogram i andre fagfelt. Nåvæerende forslag er:

\begin{itemize}
	\item Master i computational science
	\item Master i matematikk?
	\item Master i informasjonsteknologi?
	\item Master i peteroleumsteknologi
\end{itemize}