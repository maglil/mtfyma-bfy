\chapter{Samfunnsoppdraget}
\label{c:mission}

\section{MTFYMA}
Utvikling av \emph{ny} teknologi vil ofte være basert på \emph{grunnleggende} forståelse av de underliggende prosesser og modeller. Kandidater fra MTFYMA imøtekommer dette behovet med et solid fundament i både fysikk og matematikk. Kandidatene har i tillegg en verktøykasse og personlige ferdigheter som gjør dem ettertraktet for teknologisk utvikling og verdiskapning i et bredt spekter av arbeidslivet.

\section{BFY}
Søken etter å forstå og beskrive verden rundt oss ligger i menneskets natur. Denne søken etter fundamental forståelse har lagt grunnlaget for mye av samfunnets teknologisk utvikling. Ofte er ikke anvendelsen i syne når innsikten utvikles. Kandidater fra BFY imøtekommer menneskets og samfunnets behov for grunnleggende forståelse og kunnskapsutvkling. De har en bred bakgrunn i fysikk og tilhørende fagområder som matematikk og informatikk. Kandidatene er godt kvalifisert for videre studier i fysikk eller andre teknologisk fag, samt å fylle teknologiske stillinger i næringslivet.

\section{MSPHYS}
Teknolgisk utvikling følger forskningsfronten. Kandidater fra MSPHYS har dyp forståelse og innsikt i et begrenset fagområde, nær forskningsfronten. Kunnskapen og ferdighetene kan brukes til avansert teknologisk utvikling eller videreutvikles til å til flytting av kunnskapsfronten i en PhD eller annen forskningskarriere.