\section{Curriculum guidelines for undergraduate programs in statistical science}

Rettningslinjene er publisert av American statistical association in 2014\cite{asa-guidelines2014}. 

Rapporten trekker fram nødvendigheten av tilstrekkelig bakgrunn innen nøkkelferdighetene:

\begin{enumerate}
	\item Statistiske metoder og teori
	\item Dataforvaltning
	\item Beregninger
	\item Matematisk fundament
	\item Anvendelse av statistikk (kommunikasjon og tekniske ferdigheter)
\end{enumerate}

Noen sentrale anbefalinger fra rapporten:

\begin{itemize}
	\item \emph{Data science} blir viktigere og viktigere. Forberedelse til karrierer i statistikk og data science krever (i tillegg til tradisjonelle matematikk/statistikk-ferdigheter) at man håndterer høynivå programmering og databasesystemer. Store og ustrukturerte datasett krever metoder for å finne mønstre og sammenhenger i høy-dimensjonelle datasett, og metoder for å unngå bias fra slike data. Dette er et datadrevet perspektiv med mindre fokus på hypoteser og statistisk signifikans.
	\item Virkelige anvendelser er viktige. Data burde være en nøkkelkomponent av statistikkurs. Studieprogram burde vektlegge konsepter og metoder for å jobbe med komplekse data, gi erfaring med design av studier, og det å analysere ikke-tekstbok data.
	\item Eksponering mot varierte metoder og fremgangsmåter. Studenter må eksponeres mot forskjellige prediktive og forklarende modeller i tillegg til metoder for modellering og evaluering. Studenter må forstå utfordringene knyttet til design, \enquote{confounding}, og bias. Studenter må kunne være i stand til å bruke sitt teoretiske fundament til fornuftig og solid dataanalyse.
	\item Evne til kommunikasjon er viktig. Studenter må kunne kommunisere komplekse metoder i enkel terminologi til ledere og variert publikum. Studenter må ha evne til å visualisere resultater på en lett forståelig måte.
	\item Studenter som skal gjøre PhD-studier trenger en sterk bakgrunn i matematikk og teoretisk statistikk.
\end{itemize}

