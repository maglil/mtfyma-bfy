

\section{AAPT Computational physics}

(Nedtegnet 16.11.2022)

Rapporten "AAPT Recommendations for computational physics in the undergraduate physics curriculum" (\url{https://www.aapt.org/resources/upload/aapt_uctf_compphysreport_final_b.pdf})  er fra 2016 og er i tråd med andre anbfalinger samt nåværende målsetninger til programmet. Den fremhever at \textit{computational physics} er en tredje pillar i fysikk, på linje med teori og eksperiment.

Rapport en gir et rammeverk for hvordan beregningsorientert ferdigheter kan systematiseres. Det diskuteres også en rekke elementer som er viktig å ta med seg. Det meste av dette ligger allerede inne i studieprogrammenes planer.

Noen aspekter som ble fremhevet som muligens ikke er like tydelige i programmenes nåværende planer

\begin{itemize}
	\item Generelle programmeringsferdigheter og beregningsorientering bør utvikles i samme kontekst heller en separat
	\item Ferdighetene skal være 'authentic to the discipline'
	\item 'Best thaught in a lab setting' (i motsetning til typisk teorikurs).
	\item "How to teach computationl physics is anecdotal". Det er altså rom og behov for å gjøre forskning på endringer som gjøres i programmene våre.
	\item Det er flere 'communities' som jobber med beregninsorientering. Det er kanskje lurt om studieprogrammene og/eller fagmiljøene er involvert i disse.
\end{itemize}

Rapporten gir en liste over fysikk-tema som kan egne seg for beregningsmetorder.

Av spesiell interesse er kanskje en liste over potensielle organisatoriske utfordringer som det er viktig å ta hensyn til (se VI. Curricular issues og VII. Challenges).
Rapporten gir referanse til flere artikler som kan gi enda mer mer bakgrunn, samt noen bøker som er gått litt mer grundig til verks når det gjelder implementering av beregningsorientering.
