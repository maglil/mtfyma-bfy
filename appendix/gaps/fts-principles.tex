\subsection{FTS prinsippene}
\label{sec:fts-principles}
FTS har definert 10 prinsipper\footnote{\url{https://www.ntnu.no/fremtidensteknologistudier/prinsipper}} som NTNU har vedtatt skal være styrende for teknologi-utdanningene ved NTNU. Det gis her en kortfattet vurdering i hvilken grad MTFYMA og BFY oppfyller disse prinsippene. Dersom det eksisterer et gap mellom ønsket og nåværende situasjon gis det kortfattet forslag om hvilke tiltak som kan iverksettes for å redusere gapet\footnote{Denne gapsanlysen er et utkast og planlagt ferdigstilt i løpet av høsten 2022}.

\begin{quote}
	\textbf{Prinsipp 1:} NTNUs teknologistudier skal legge aktivt til rette for at kandidatene, med utgangspunkt i et solid faglig fundament, opparbeider helhetlig og integrert kompetanse, herunder bærekraftkompetanse og digital kompetanse på høyt nivå.
\end{quote}

Studieprogrammet gir studentene et omfattende faglig fundament i fysikk og matematikk, men det er en mangel på trening i å kunne bruke kunnskap og ferdigheter på tvers av flere kunnskapsområder (dvs. integrert kompetanse, e.g. matematisk modellering, multifysikk, eksperimenter, numerikk, prosjektstyring, etc.) og da spesielt i mer omfattende og åpne oppgaver.

Det eksisterer ingen overordnet plan for utvikling av digital kompetanse, noe som medfører at utfallet er usikkert, tilnærmingen fragmentert og nivået for lavt ettersom man ikke vet hva man kan bygge videre på.

Det er uklart hva som menes med \emph{bærekraftskompetanse} for MTFYMA og BFY. Det må defineres hva som ligger i dette for at man skal kunne vurdere om det er oppfylt.

\textbf{Forslag til tiltak:}
\begin{itemize}
	\item Det anbefales at det i større grad utformes læringsaktiviteter hvor studentene må bruke kunnskap og ferdigheter fra flere kunnskapsområder/ferdigheter. Dette vil være ganske omfattende tidsmessig så vil kanskje måtte gjøres i dedikerte emner.
	\item Det utarbeides en helhetlig plan over hvordan de digital ferdigheten i programmet utvikles.
	\item Det utarbeids en definisjon av hva man legger i bærekraftskompetanse og dernest en plan for hvordan studentene evt. kan oppnå denne kompetansen.
\end{itemize}

\begin{quote}
	\textbf{Prinsipp 2:} NTNU skal legge aktivt til rette for at kandidater fra teknologistudiene opparbeider tverrfaglig samhandlingskompetanse, og for at man over den samlede studentpopulasjonen får et mangfold i kunnskapsprofiler, samtidig som den enkelte student oppnår tilstrekkelig programfaglig dybde.
\end{quote}

Det er mulig å velge ulike kunnskapsprofiler i det nåværende studieprogrammene men det er i liten grad synliggjort overfor studentene hvordan flere emner kan settes sammen til en hensiktsmessig profil. Data på emnevalg indikerer at studentene ikke velger så bredt som man kunne ønske. Det er også i liten grad synliggjort for studentene hvordan de kan bruke emner fra andre institutt for å skape en helhetlig profil.

Utover EiT er det ingen planlagte elementer i studieprogrammet for å bygge tverrfaglighet. Det er usikkert om EiT i tilstrekkelig stor grad definerer en reell teknologisk problemstilling som et tverrfaglig team skal løse i praksis. 

\textbf{Forslag til tiltak:}

\begin{itemize}
    \item Planlegge og designe tydelige fagprofiler som synliggjøres for studentene.
	\item Lage emner med prosjekter hvor man er nødt til å bruke fagkunnskap fra ulike studieprogram for å komme frem til en løsning.
	\item Samhandlingskompetansen som introduseres i EiT bør nok også introduseres på et tidligere tidspunkt i studiene slik at man både kan nyttiggjøre seg den gjennom studiene og få mer trening.
\end{itemize}

\begin{quote}
	\textbf{Prinsipp 3:} Kontekstuell læring skal legges til grunn som gjennomgående pedagogisk prinsipp i NTNUs teknologistudier.
\end{quote}

Kontekstuell læring brukes i svært liten grad. Det er noe uklart hva slags kontekst som vil bidra til bedre læring. Er det primært en anvendt, realistisk kontekst som er viktig?

\textbf{Forslag til tiltak:}

\begin{itemize}
    \item Innhente eksempler fra industri som kan brukes som utgangspunkt for å presentere teori for studentene. Dette kan inkludere gjesteforelesninger og at problemstillingen inkluderes i øvinger og mindre prosjekter.
\end{itemize}

\begin{quote}
	\textbf{Prinsipp 4:} NTNUs teknologistudier skal benytte kunnskapsbaserte, studentaktive og engasjerende undervisnings- og vurderingsformer som er samstemt med utdanningenes overordnede kompetansemål, fremmer god læringskultur, og gir effektiv dybdelæring
\end{quote}

Undervisningsformer ved programmet er i stor grad basert på tradisjon heller enn en kunnskapsbasert tilnærming til hvordan læring fungerer. Ikke dermed sagt at dagens form er feil, men det må vurderes om dagens form bidrar til å oppnå det man ønsker med studieprogrammet.

Emnene er i stor grad preget av stofftrengsel, spesielt på grunnivå, noe som reduserer muligheten for dybdelæring og gir redusert mestringsfølelse, noe som reduserer engasjement og motivasjon.

\textbf{Forslag til tiltak:}

\begin{itemize}
	\item Tydelig plan om innhold i emner på programnivå for å hindre stofftrengsel og forbedre samkjøring.
	\item Forsterke emnegruppene på IFY for å skape mer dialog om undervisningsformer.
	\item Gjør beviste valg av undervisnings- og vurderingsformer slik at det bidrar til god læringskultur.
\end{itemize}

\begin{quote}
	\textbf{Prinsipp VI:} Kvaliteten i NTNUs teknologistudier skal utvikles gjennom en programdrevet tilnærming, i kombinasjon med strategisk porteføljeutvikling og -forvaltning på tvers av programmer og programtyper.
\end{quote}

Studieprogrammene er i svært liten grad programdrevet ettersom det i liten grad eksisterer noen skriftlig dokumentasjon på hva innholdet skal være. Det lille som eksisterer i beskrivelsen av studieprogrammet er for vag og brukes i liten grad. Emnebeskrivelsen kan fritt endres av faglærer og er ikke styrt av programmet. Faglærerer tenker i liten grad utover sitte eget emne og i liten grad på hvordan det bidrar til helheten når de underviser.

Det er i liten grad tenkt på hvordan emnene henger sammen og bygger videre på hverandre. Eventuelle koblinger forvitrer også som følge av den enkelte faglærers frihet til å endre form og innhold i emnene.

Undervisning sees nok fortsatt på som et individuelt ansvar for et gitt emne og ikke et kollektivt ansvar. Dette er kanskje noe i endring på IFY som følge av etablering av emnegrupper

\textbf{Forslag til tiltak:}

\begin{itemize}
	\item Utvikle dette dokumentet slik at det dokumenterer en felles forståelse av hvordan studieprogrammene bør drives.
	\item Økt fokus emnegrupper på IFY for å styrke den kollektive tankegangen om undervisning.
\end{itemize}

\begin{quote}
	\textbf{Prinsipp VII:} NTNUs kvalitetsarbeid i teknologistudiene skal stimulere studieprogrammenes utvikling mot utdanningskvalitet i verdensklasse ved å fokusere på kontinuerlig forbedring og systematisk utvikling av kvalitetskultur.
\end{quote}

NTNUs kvalitetssystem har to store mangler. For det første er det primært et rapporteringssystem, men sirkelen er i liten grad sluttet ved at det også lages handlingsplaner som skaper endring. I tillegg er emneevaluering på emnenivå tungt basert på hva studentene \emph{syns}, og i liten grad på hva de faktisk lærer. 

Det er noe benchmarking internasjonalt og interessegrupper i forbindelse med dybdeevalueringer. Dette kunne kanskje styrkes ved å ha fast partnere som man bruker som benchmarking over tid. 

Man bør også kanskje ha blikket mer kontinuerlig rettet utover for å følge med på utviklingen internasjonalt, heller enn å kunne få et blikk på sin egen utdanning en gang hvert femte år.

\begin{quote}
	\textbf{Prinsipp VIII:} NTNU skal gi høy prioritet til strategisk og operativt internasjonalt samarbeid om utvikling av teknologistudier, med mål om å bli et internasjonalt synlig og anerkjent universitet også på dette området.
\end{quote}

Programmene er liten grad aktive i fora hvor det diskuteres undervisning av ingeniører og realister internasjonalt. IMF har vært mer aktive her. Det bør vurdere om man har kapasitet og om det vil være hensiktsmessig å øke denne aktiviteten.

\begin{quote}
	\textbf{Prinsipp IX:} NTNUs teknologistudier skal vektlegge systematisk samhandling med arbeidsliv og samfunn, med mål om å fremme arbeidsrelevans, legge til rette for livslang læring, og sikre at studenter kan opparbeide relevant arbeidslivserfaring gjennom studiene.
\end{quote}

Studiene har i liten grad samarbeid med eksternt arbeidsliv i studiene.

Et unntak er medisinsk fysikk hvor ansatte på sykhuset er tett integrert i utforming og gjennomføring av undervisning i disse områdenee.

\subsubsection{Prinsipper primært for andre nivå}

Prinsipp V og prinsipp X er primært relevant for henholdsvis institutt og sentralt nivå. Disse prinsippene er.

\begin{quote}
	\textbf{Prinsipp V:} NTNU skal stille tydelige forventninger til, og gi solid støtte for, kompetanseutvikling hos undervisningspersonell.
\end{quote}

\begin{quote}
	\textbf{Prinsipp X:} NTNU skal utvikle sitt læringsmiljø – og spesielt sin campus og infrastruktur (både fysisk og digital) – i en retning som understøtter de øvrige FTS-prinsippene I – IX, og som fremmer læring, helse og trivsel blant studenter og ansatte.
\end{quote}

