\section{Forslag til endringer}

Her følger en punktliste over foreslåtte endringer i studieprogrammene baserte på gapsanalysere i forhold til FTS, dybdeevalueringen (se senere deler i dette kapittelet) og oppsamlede erfaringer som er gjort med dagens studieprogram.

Tilsammen vil mengden av de punktene man mener er fornuftige på denne listen diktere om det er nødvendig med en omfattende omlegging eller mindre justeringer.

Listen er ikke prioritert.

\begin{itemize}
	\item Innholdet i studieprogrammet beskrives i et Hoveddokument.
	\item IMF og IFY må ha et mye tettere samarbeid om innholdet i emnene slik at man oppnår synergier og ikke bare suboptimalisering ved å ha et felles studieprogram.
	\item Studenter ved BFY og MTFYMA kan med fordel ha en dypere og bredere innføring i matematikk enn det som er felles for andre siv.ing. program.
	\item Det opprettes egne emner for å utvikle praktiske ferdigheter, spesielt innen eksperimentelt arbeid og beregninger, gjerne kombinert.
	\item Studentene må i større grad eksponeres for oppgaver som trener integrert kompetanse, eksemplevis som integrerer eksperimenter, simuleringer, prosjektstyring og fysikk fra flere områder.
	\item Det må være en helhetlig plan for opplæring og trening i numeriske metoder og algoritmer.
	\item Studentene må få opplæring og trening i verktøy som er viktige for effektivt å utføre beregningsorienterte oppgaver. Dette inkluderer: Versjonskontroll (git), package-managers (conda), enhetstesting (pytest), terminalvindu, linux, m.m
	\item Studentene må få god opplæring i verktøy som de forventes å benytte.
	\item Studentene må få trening i distribuerte utviklingsprosjekter.
	\item Studentene skal få opplæring og trening i å bruke HPC ressurser, parallellisering og optimalisering (kan muligens være valgbart). 
	\item Studentene skal få trening i metoder for maskinlæring. 
	\item Studentene skal få trening i å håndtere store, ustrukturerte datasett.
	\item Eksperimentelle ferdigheter skal struktureres rundt å kunne gjennomføre hele verdikjeden fra problemformulering til rapportering.
	\item Studentene skal få mye mer trening i rapportskriving enn i dag.
	\item Studentene må gis mulighet for å fordype seg i tema relatert til bærekraft. Studenter med fysikk og matematikk bør bygge på sin styrke i forståelse av fysisk modellering og evne til å forstå å bruke avansert matematiske metoder. Det må tydeliggjøres hvilken kompetanse relatert til bærekraft studentene bør utvikle.
	\item Det bør utformes tydelig emneløp som gir spesialiseringer innen ulike fagområder. Dette må kommuniseres godt til studentene.
	\item Det bør legges ressurser inn på obligatoriske emner som leder opp mot valg av spesialisering.
	\item Istedenfor EiT bør det innføres et emne hvor man tverrfaglig skal løse et gitt teknologisk problem.
	\item Innføring i samarbeid (ala EiT) bør komme i begynnelsen av studiene og studentene bør få trening i dette gjennom hele studiet.
	\item Innføring i teori bør i større grad gjøres i en kontekst slik at studentene tydelig ser verdien av teorien.
	\item Det antas at studenter ved MTFYMA og BFY har tatt fysikk 2. For de som ikke har tatt det eller trenger oppfriskning tilbys et oppfriskningskurs (digitalt eller fysisk).
	\item Emnene i første semester og år skal legge et spesielt fokus på å motivere studentene til videre studier og ha et bevist forhold til at å etablere et liv på universitetet kan være krevende for mange.
	\item Det skal være en bevist og dokumentert plan for når matematiske temaet introduseres i matematikkemnene og når det er behov for dem i fysikk emnene.
	\item Studieprogrammene må ha et tett samarbeid med industri for å kunne bruke dette til å kontekstualisere teorien samt å skaffe oppgaver som er mer sammensatte og åpne.
	\item Studieprogrammene og fagmiljø skal i fellesskap utvikle en felles forståelse for det læringsmiljøet man ønsker å etablere og de undervisnings- og vurderingsformer som gjør at man oppnår det.
	\item Studenten bør få god trening i ulike former for prosjektarbeid, bl.a. arbeidsdeling, ledelser, planlegging, risiko, samarbeid m.m.
\end{itemize}