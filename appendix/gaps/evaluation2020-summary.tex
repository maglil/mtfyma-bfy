\section{Dybdeevaluering 2020}

En ekstern komite og en student-komite gjennomførte i 2020 en evaluering av studieprogrammene MTFYMA og BFY. Evalueringen fokuserte på \emph{eksperimentelle ferdigheter} og \emph{numeriske ferdigheter}. Hensikten var å dekke beregningsorientering av både fysikk og matematikk men pga av misforståelser i mandatet ble hovedfokus på fysikkemnene.

Her følger en oppsummering av anbefalinger fra rapporten, tolket av studieprogramledelsen.

\subsection{Generelt}

\begin[itemize}
	\item Studentene bør eksponeres for oppgaver som integrerer beregningsorientering og eksperimenter, samt få trening i å vurdere om spørsmål besvares best med eksperimenter, beregninger eller ny teori.
\end{itemize}

\subsection{Eksperimentelle ferdigheter}

\begin[itemize}
	\item Studentene bør få mer trening i rapportskriving og erfaring med rapporter kan lages i mange ulike format, ikke bare vitenskapelig artikkel. Det bør være felles læringsressurser og man må ha kontroll på at alle studieretningen får tilstrekkelig god trening i dette. 
	\item Studentene må få en systematisk innføring i dokumentasjonspraksis, journalføring m.m. Dette bør også refereres til internasjonale standarder om dette.
	\item Studiet bør prioritere generiske laboratorieferdigheter foran demonstrasjoner for å støtte opp om teori.
	\item Det bør være mindre bruke av ferdige oppsett av eksperimenter og mer fokus på at studenten må gjennomføre hele verdikjeden fra problemformulering til rapportering. Spesielt å kunne formulere en testbar hypotese som kan testes av eksperimenter eller numeriske beregninger. Dette er en forutsetning for meningsfylt rapportskriving.
	\item Det bør være egne kurs for eksperimentell aktivitet.
	\item Studentene bør få erfaring med å jobbe i prosjektform hvor man jobber individuelt i team og må planlegge hensiktsmessig arbeidsinndeling.
\end{itemize}

\subsection{Beregningsorienterte ferdigheter}

\begin[itemize}
	\item Systematisk innføring i dokumentasjonspraksis og versjonskontroll
	\item Det bør gis en helhetlig innføring i numeriske beregninger og ferdighetsstrengene virker som et positivt virkemiddel for dette.
	\item Studentene bør få kompetanse på maskinlæring og bruk av HPC ressurser.
	\item Ulikhet i obligatoriske emner mellom MTFYMA og BFY gjør det vanskelig å bygge på tidligere fag og gjøre masteremner tilstrekkelig avansert.
\end{itemize}
