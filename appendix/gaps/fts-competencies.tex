\subsection{FTS kompetansemål}
\label{sec:fts-competencies}
FTS delrapport 1\footnote{\url{https://www.ntnu.no/fremtidensteknologistudier}} introduserer begrepet kompetanseprofil for et studium og beskriver 12 kompetansemål som tilsammen gir en kompetanseprofil som skal kjennetegne studenter fra et teknologistudium ved NTNU.

Her følger en analyse av eventuelle gap mellom ønsket kompetanseprofile (i følge FTS) og nåværende studieprogram\footnote{Foreløpig bare et utkast, planlagt ferdigstilt i samarbeid med fagmiljøene i løpet av høsten 2022}.

\subsubsection{K1: Vise fagkunnskap og faglig fundert perspektiv}
\begin{description}
\item[Konkretisering:] Dette utgjør stammen i studieprogrammets innhold og er et omfattende punkt å detaljere. Rapporten foreslår en måte å strukturere innholdet og legger frem noen overordnede aspekter som er viktig. Fagmiljøene og studieprogrammene må spesifisere hva dette betyr i praksis og hvordan det konkret skal implementeres.
\item[Gap:] Dagens kandidater ved MTFYMA og BFY opparbeider uten tvil en bred faglig kunnskapsbase. Likevel utdypes dette punktet litt mer i detalj og i forhold til noen av disse områdene er det et vist gap mellom ønsket og dagens situasjon. For eksempel så presiseres det at det er dyp, virksom kunnskap som søkes, slik at kandidaten er i stand til å \emph{bruke} kunnskapen kreativt og effektivt i problemløsning. Det kan stilles spørsmålstegn ved om dagens eksamensfokus tester kunnskapen relativt overfladisk og over et kort tidsrom, samt at kandidatene får lite erfaring i å \emph{anvende} kunnskapen. Det vurderes i liten grad om kunnskapen på sikt er dyp og virksom.
\item[Tiltak:] 
\end{description}

\subsubsection{K2: Analyse av komplekse problemstillinger og systemer}
\begin{description}
\item[Konkretisering:] I dette punktet fremheves (for masternivået) at dette skal omfatte \textquote{\emph{selvstendig} problemformulering og analyse} samt at problemene skal være \emph{umedgjørlige}. 
\item[Gap:] Dagens program har muligens et for stort fokus på å løse ferdig oppstilte problemer som har et entydig svar. Studentene bør eksponeres for er mer sammensatte. Man kan ikke alltid \enquote{a white spherical cow}, studentene må trenes i å selv gjøre nødvendig og rimelige forenklinger og finne passende metode. Man kan også vurdere om det er for lite krav til at studentene er selvstendige aktører. Selv på masteroppgaver møter kandidatene ofte ferdige problemstillinger og ferdige eksperimentelle oppsett som skal brukes.
\item[Tiltak:] 
\end{description}

\subsubsection{K3: Design og implementering av bærekraftige løsninger}
\begin{description}
\item[Konkretisering:]Det er ikke bærekraftige løsninger som er det sentrale i dette punktet men evnen til å design og implementere løsninger på problemer. Dernest at disse løsningen skal vurderes i en samfunnsmessig/bærekraftig kontekst. Når det gjelder bærekraftsdelen må dette konkretiseres og man må kartlegge hva som gjøres i fellesemnene. Forslag som kan være relevante for bærekraft: Inkludere klimafysikk der det passer, identifisere hvilken unik kompetanse studentene har i et klimaperspektiv, challenge-based learning.
\item[Gap:]Utover et prosjekt i Instrumentering ser det ut som studentene får lite trening i dette.
\item[Tiltak:]Målet må være at studentene får trening i å finne løsninger som representerer det settet med avanserte kunnskaper og ferdigheter de har. Det er også viktig å stimulere kreativiteten.
\end{description}

\subsubsection{K4: Benytte relevante metoder og verktøy}
\begin{description}
\item[Konkretisering:] Det sentrale innholdet i dette punktet er at kandidatene skal beherske metoder og verktøy som gjør dem i stand til å \emph{effektivt anvende} sin fagkunnskap for å finne løsninger på problemer. Rapporten fremhever særskilt noen områder:
\begin{itemize}
	\item Håndtere store, ustrukturerte datasett.
	\item Maskinlæring.
	\item Digital sikkerhet og dømmekraft.
	\item Prosjektplanlegging.
	\item Effektiv kommunikasjon.
	\item Sensorteknologi, automatisering.
\end{itemize}

\item[Gap:]Generelt for MTFYMA og BFY legges det for stor vekt på teoretisk kunnskap og for lite på verktøy og metoder som er nødvendig for å effektivt bruke denne kunnskapen. Alle punktene i listen over er relevante (kanskje med unntak av digital sikkerhet) og har rom for forbedring. Flere slike metoder er også fremhevet i dybdeevalueringen fra 2020. 
\item[Tiltak:]Det anbefales at programmene, spesielt MTFYMA legger større vekt på opplæring i relevante metoder og verktøy for å anvende kunnskapen og at studentene får tid til å opparbeide erfaring.
\end{description}


\subsubsection{K5: Konsekvensanalyse, risikovurdering og scenariotenkning}
\begin{description}
\item[Konkretisering:] Punktet innebærer å vurdere konsekvensen av teknologiske løsninger i et større samfunnsperspektiv. Det er uklart hvordan dette skal konkretiseres for MTFYMA og BFY som ikke jobber så direkte med teknologiske løsninger.
\item[Gap:]Disse elementene er i liten grad dekket av dagens studieprogram (bortsett fra risikovurdering ved laboratoriarbeid)
\item[Tiltak:]Utvikling av disse ferdighetene krever at studentene eksponeres for problemer hvor slike analyser er relevant. På lokal skala kan dette innebære å måtte gjøre beviste valg av metodikk for å besvare et problem selv når beslutningsgrunnlaget er usikkert. For å utvikle disse ferdighetene på større nivå (samfunn, økonomi, klima) er det nok nødvendig med realistiske problemstillinger, noe som vil kreve at man har et samarbeid med industri. Et mulig fremgangsmåte kan være å se på historisk dårlige valg (KFK, blytilsetning) og hva som kunne vært gjort annerledes.
\end{description}

\subsubsection{K6: Kjenne til forskning og bidra til teknologiutvikling}
\begin{description}
\item[Konkretisering:] Studenter ved MTFYMA og BFY går ikke inn i enn konkret bransje og kan ikke forventes å kjenne til forskningsfronten i den bransjen de måtte ende opp i. Det som er å viktig er å ha et solid faglig fundament slik at man effektiv kan sette seg inn i forskningsfronten innenfor et nytt område, samt at man har trening i å sette seg inn i primærlitteratur og få oversikt over et fagfelt. Å bidra til effektivt til teknologiutvikling fordrer også endel personlig egenskaper men de er dekket av andre kompetansemål.
\item[Gap:] Studentene får lite trening i å sette seg inn i primærlitteratur utover master-/prosjektoppgave.
\item[Tiltak:] Innføre flere læringsaktiviteter i emner hvor studentene må finne frem og vurdere primærlitteratur. 
\end{description}

\subsubsection{K7: Innhente og kritisk vurdere informasjon}
\begin{description}
\item[Konkretisering:] For studenter ved MTFYMA og BFY anses det som spesielt viktig at de skal være i stand til å finne fram og vurdere kvaliteten på forskningsresultater og konklusjonene som blir trukket basert på dem. 
\item[Gap:] Det virker å være veldig lite innslag av dette i studiet per i dag. Kritisk tenkning er en essensiell ferdighet som må utvikles.
\item[Tiltak:] Innføre flere læringsaktiviteter i emner hvor studentene må finne frem og vurdere primærlitteratur. Eksempelvis analysere artikler: Hva er feil i artikkelen? Kunne det vært presentert bedre. Gruppearbeid med å forstå artikkel, må grave videre i referanser.
\end{description}


\subsubsection{K8: Livslang læring}
\begin{description}
\item[Konkretisering:] Dette punktet peker på at studiet ikke bare skal gi studentene kunnskap og ferdigheter men også skal gi dem effektive strategier for læring. Punktet omhandler også å utvikle en positiv holdning til omstilling og evne til å vurdere sine egne ferdigheter. Viktig å kunne vurdere både egne og andres kompetanse.  
\item[Gap:] Den velkjente frasen fra studenter om at det man lærte mest ved studiet var å lære, er nok en sannhet med modifikasjoner. Mange studenter bruker nok ganske ineffektive metoder for læring. 
\item[Tiltak:] Konkretiser hvilke læringsaktiviteter og vurderingsformer som bidrar til gode læringsstrategier og -vaner. Fokusere mer på hvordan man lærer, metakognisjon.
\end{description}  

Per i dag er det ingen bevist tilnærming ved studiene for hvordan man sørger for at studentene tilegner seg hensiktsmessige læringsstrategier. Det bør nok utvikles.

\subsubsection{K9: Bruk og refleksjon over normer og helhetstenkning rundt etikk og bærekraft} 
\begin{description}
\item[Konkretisering:] I arbeidslivet vil studentene møte problemstillinger som ikke har en ren teknologisk løsning. De må derfor kunne reflektere over samfunnets etiske normer og bruke disse som beslutningsgrunnlag. Hva bærekraftskompetanse skal være for MFTYMA og BFY må konkretiseres. Skal man forstå det akademisk eller skal man utvikle holdninger.
\item[Gap:] Etisk refleksjon er dekket via ExPhil. Det må konkretiseres hva som menes med bærekraftskompetanse før et evt. gap kan identifiseres.
\item[Tiltak:] Konkretisere hva som menes med bærekraftskompetanse.
\end{description}

 Det bør vurderes om studentene skal få mer trening i å angripe slike problemstillinger i løpet av studiet.

\subsubsection{K10: Målrettethet, samhandlingsevne og lederskap}
\begin{description}
\item[Konkretisering:] Denne kompetansen handler om å kunne gjennomføre prosjekter i samhandling med andre på en effektiv måte. Punktet er på den måten nært knyttet opp til neste punkt om kommunikasjon og samhandling. 
\item[Gap:]Dette punktet er delvis dekket av EiT og der med spesielt fokus på tverrfaglig samarbeid. Men for å opparbeid gode samhandlingsferdigheter er det trolig nødvendig med en gjennomgående utvikling gjennom hele studiet. Studiet inneholder heller ingen trening i ledelse, prosjektstyring i større faglige prosjekter.
\item[Tiltak:] Det børe legges en plan for hvordan studentene kan opparbeide disse ferdighetene gjennom hele studiet. Det bør også legges opp til trening i faglig samhandling (i  motsetning til tverrrfaglig) hvor man gjennomfører konkrete faglige prosjekter som krever arbeidsfordeling, lederskap og prosjektstyring.
\end{description}

\subsubsection{K11: Kommunikasjon, formidling og dialog}
\begin{description}
\item[Konkretisering:] Dette punktet omhandler alle former for kommunikasjon, skriftlig og muntlig, til forskjellige målgrupper og med varierende hensikt.
\item[Gap:] Det eksisterer per i dag ingen overordnet plan for hvordan disse ferdighetene utvikles på programnivå. Det er per i dag alt for lite trening i skriving, og spesielt i å skrive i ulike formater og for ulike målgrupper. Det samme gjelder muntlig presentasjon og dialog.
\item[Tiltak:] Det bør utarbeides en plan for hvordan og hvor studentene skal få trening i disse ferdighetene. Planene bør sikre at det er en progresjon i nivå og at ulike former for kommunikasjon dekkes. Dette vil nødvendigvis medføre føringer på undervisningsformene i ulike emner.
\end{description}

\subsubsection{K12: Nyskaping, entreprenørskap og forretningsforståelse}
\begin{description}
\item[Konkretisering:] Dette kompetansemålet kan deles i to deler, de 1) forretningsmessige og 2) de personlige. De forretningsmessige aspektene er relativt konkrete og inkluderer finansiering, markedsforståelse, bedriftsutvikling m.m. De personlige ferdighetene som kreves i en innovasjonsprosess er mer vage og inkuderer kreativitet, initiativ, selvstendighet, motstandsdyktighet m.m.
\item[Gap:] De forretningsmessige aspektene er relativt godt dekket gjennom teknologiledelse og k-emner. For de personlige ferdighetene eksisterer det per i dag ingen gjennomtenkt plan. Studieprogrammenes emner er i stor grad tradisjonelle. Det er svært få elementer som fordrer kreativitet, karakterfokus bidrar til en ekstern motivasjonsorientering, og høy stofftetthet bidrar ikke til mestringsfølelse. 
\item[Tiltak:] Det bør lages en tydelig plan for om og hvordan man ønsker å bygge de personlige ferdighetene som er viktig i innovasjonsprosesser. Hvordan dette gjøres best er et vanskelig spørsmål men vil nødvendigvis medføre føringer på formen på undervisningsaktivitetene i flere emner. For de forretningsmessige aspektene kan man vurdere om man skal synliggjøre en pakke med k-emner for de som er spesielt interessert i denne retningen.
\end{description}