\subsection{fts-competencies}

\subsubsection{K1: Vise fagkunnskap og faglig fundert perspektiv}

Dagens kandidater ved MTFYMA og BFY opparbeider uten tvil en veldig bred faglig kunnskapsbase. Likevel utdypes dette punktet litt mer i detalj og i forhold til noen av disse områdene er det et vist gap mellom ønsket og dagens situasjon. 

Ordlyden og utdypningen presisserer at det som søkes er dyp, virksom kunnskap som søkes, slik at kandidaten er i stand til å \emph{bruke} kunnskapen kreativt og effektivt i problemløsning. Det kan stilles spørsmålstegn ved om dagens eksamensfokus tester kunnskapen relativt overfladisk og over et kort tidsrom, samt at kandidatene får lite erfaring i å anvende kunnskapen. Det vurderes i liten grad om kunnskapen på sikt er dyp og virksom.

Delrapport 1 deler opp dette punktet i 4 deler

\begin{itemize}
	\item ...
\end{itemize}

generaliserbare konsepter.
kontekstualisering
beregningsorientering
stordata, maskinlæring
foretningsforståelse, invoasjonsprosesser

ii) bredde teknisk: prosjektledelse, muliggjørende teknologier

iii) vei dybde vs bredde

iv) komplementært i forhold til fremtidens behov

-bredde i kunnskapsprofiler

Steam - kreativitet

K2 Analyse av komplekse problemstillinger og systemer

K3 Design og implementering av bærekraftige løsninger.
