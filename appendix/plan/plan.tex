\subsection{FTS-revisjon}
Her følger en beskrivelse av overordnet plan for revisjon av studieprogrammet som følge av anbefalingene som er gitt av FTS-prosjektet. Revisjon av studieprogrammet som følge av dybdeevaluering av MTFYMA og BFY for 2020 inkluderes i samme prosjekt.

FTS-revisjon deles opp i fire faser. Prosjektplanen tar utgangspunkt i at en omfattende omlegging kan gi en signifkant forbedring av studieprogrammet. Etter de to første fasene vurderes det om det er tilstrekkelig behov for å gå videre med neste fase eller om en nedskalert revisjon er mer hensiktsmessig. Det kan være naturlig å starte noe arbeid med en senere fase selv om den foregående fasen ikke er ferdigstilt, men det bør ikke legges for mye ressurser i et slikt arbeid før man er sikker på at man ønsker gå videre med neste fase. Tentativ år/måned for avslutning av hver fase er angitt. Ettersom det er vanskelig å vurdere hvor lang tid hver fase vil ta vil disse justeres ettersom prosjektet skrider frem.

\begin{itemize}
	\item Fase 1: Gapsanalyse og visjon (2022/12). Identifisere gap mellom anbefalinger og ønsker om hvordan man ønsker at studieprogrammet skal være og nåværende situasjon. Gapsanalysen vil minimum omfatte FTS prinsipper og kompetansemål samt dybdeevaluering fra 2020. Dersom gapet anses å være tilstrekkelige går man over i Fase 2 for å utvikle og dokumentere ønskede kompetansemål
	\item Fase 2: Kompetansemål (2023/12). Utvikle ønskede kompetansemål fra overordnet til detaljert nivå. Samtidig beskrives hvordan studentene best kan oppnå denne kompetansen. Dersom kompetansemålene ikke enkelt kan realiseres i dagens programstruktur går man videre til Fase 3 for å utvikle en ny programstruktur.
	\item Fase 3: Programstruktur (2024/6). Det utformes en ny programstruktur for studieprogrammet.
	\item Fase 4. Læringsaktiviteter og -ressurser (2025/6). Disse utvikles før oppstart av nytt studieprogram.
\end{itemize}

%Høst Innhold- og strukturprosjekter: Overordnet struktur, beregning, eksperimentelt, læringsmiljø, vurderingsformer, undervisningsformer, mekanikk, elmag, matematikk, ønsker faggrupper.

%Revider prosjektplan. Større/mindre aktiviteter

%Ressursbehov er 20\% frikjøp i prosjekgruppen (50\% i tillegg til programledere) De andre komiteene leverer arbeider som ikke er for arbeidskrevende. 

%2023 

%Vår Overordnet struktur legges litt fastere, undervisningsformer.

%Ressursbehov. som over. Det er relativt usikkert hva ressursbehovet her er, ettersom dette er en relativt ny arbeidsmetode.

% Det kan være behov her for en gruppe som kan gå litt dypere, men det kan være vanskelig å få til på så kort varsel. Høsten 2022 bør det identifiseres evt. ressursbehov for høsten 2023. Må undersøke for raskt man kan få frigjort ressurser. 

%Høst

%Flere innholdsprosjekter

% Testing av formater underveis

