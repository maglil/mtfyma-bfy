\subsection{FTS-revisjon}
Her følger en beskrivelse av overordnet plan for revisjon av studieprogrammet som følge av anbefalingene som er gitt av FTS-prosjektet. Revisjon av studieprogrammet som følge av dybdeevaluering av MTFYMA og BFY for 2020 inkluderes i samme prosjekt.

FTS-revisjon deles opp i fire faser. Prosjektplanen tar utgangspunkt i at en omfattende omlegging kan gi en signifkant forbedring av studieprogrammet. Etter de to første fasene vurderes det om det er tilstrekkelig behov for å gå videre med neste fase eller om en nedskalert revisjon er mer hensiktsmessig. Det kan være naturlig å starte noe arbeid med en senere fase selv om den foregående fasen ikke er ferdigstilt, men det bør ikke legges for mye ressurser i et slikt arbeid før man er sikker på at man ønsker gå videre med neste fase. Tentativ år/måned for avslutning av hver fase er angitt. Ettersom det er vanskelig å vurdere hvor lang tid hver fase vil ta vil disse justeres ettersom prosjektet skrider frem.

\begin{itemize}
	\item Fase 1: Gapsanalyse og visjon (2022/12). Identifisere gap mellom anbefalinger og ønsker om hvordan man ønsker at studieprogrammet skal være og nåværende situasjon. Gapsanalysen vil minimum omfatte FTS prinsipper og kompetansemål samt dybdeevaluering fra 2020. Dersom gapet anses å være tilstrekkelige går man over i Fase 2 for å utvikle og dokumentere ønskede kompetansemål
	\item Fase 2: Kompetansemål (2023/12). Utvikle ønskede kompetansemål fra overordnet til detaljert nivå. Samtidig beskrives hvordan studentene best kan oppnå denne kompetansen. Dersom kompetansemålene ikke enkelt kan realiseres i dagens programstruktur går man videre til Fase 3 for å utvikle en ny programstruktur.
	\item Fase 3: Programstruktur (2024/6). Det utformes en ny programstruktur for studieprogrammet.
	\item Fase 4. Læringsaktiviteter og -ressurser (2025/6). Disse utvikles før oppstart av nytt studieprogram.
\end{itemize}

\subsubsection{Fase 1}
Fase 1 gjennomføres høsten 2022 (Dette følger opp bestilling fra NV-fak av 22.6.22). Følgende aktiviteter gjennomføres:

\begin{itemize}
	\item \textbf{FTS prinsipper og kompetanseprofiler}. Det gjennomføres et seminar hvor deltagerne deles opp i grupper. Oppgaven vil være å vurdere hvert prinsipp og hver delkompetanse anbefalt av FTS og konkretisere hva disse betyr eller hvordan de kan omformuleres for MTFYMA og BFY. Dernest å vurdere om disse aspektene er imøtekommet i dagens studieprogram og hvis ikke, hvilke tiltak som kan treffes for å imøtekomme dem. Resultatet av diskusjonen kan gjerne formuleres som overordnede målsetninger for studieprogrammet. Det legges opp til frivillig deltagelse på seminaret og alle inviteres (fast og midlertidig ansatte og studenter). En halv dag (ca. 3 timer) er trolig tilstrekkelig. Noen fra FTS inviteres til å gi en kort innledning. Gruppene leverer skriftlige forslag (i tillegg til muntlige presentasjon på seminaret). Det gis også mulighet for de som ikke har anledning til å delta på seminaret å sende inn skriftlige innspill. Forslagene fra seminaret resulterer for hoveddokumentet i oppdaterte gapsanalyse i forhold til FTS (\ref{sec:fts-principles} og \ref{sec:fts-competencies}) , oppdatert tiltaksliste (\ref{sec:changes}) samt oppdaterte (overordnede) målsetninger for programmet (\ref{part:goals}) der det er relevant. Gjennomført innen utgangen av 9/2022.
	\item \textbf{Dybdeevaluering 2020}. Det gjennomføres to seminarer av samme format som over, et for digitale ferdigheter (numerikk og programmering) og et for eksperimentelle ferdigheter. Tilbakemeldinger inkluderes i gapsanalysen av evalueringen (\ref{sec:evaluation2020}) og tiltakslisten (\ref{sec:changes}). Basert på forslagene påbegynnes en oversikt over målsetninger for disse kompetanseområdene. Gjennomført innen utgangen av 11/2022.
	\item \textbf{Sammenlikning med andre programmer}. Det samles inn en oversikt over hvordan studieprogram som likner på MTFYMA og BFY ved andre studiesteder som vi ønsker å sammenlikne oss med har lagt opp sine studieløp. Dette gjøres primært på emnenivå. Dette resulterer i en kort rapport som beskriver studieprogrammene. Opplegg som virker spesielt spennende for MTFYMA og BFY å lære noe av fremheves. Dette arbeidet gjennomføres av en liten arbeidsgruppe hvor minst en person som kan representere hver studieretning er representert. Denne rapporten blir viktig for at revisjonen ikke skal bli for fastlåst i eksisterende struktur og blir viktig når programmets struktur skal legges senere i prosjektet. Ferdigstilt innen 10/2022.
	\item \textbf{Fagmiljøenes ønsker}. Fagmiljøene (seksjoner, faggrupper) inviteres til å sende inn synspunkter på hva som er viktig å inkludere i de første årene av studieprogrammet for videre fordypning i deres fagområde. Dette kan være ting som er viktig å bevare eller ting som er viktig å inkludere. Dette kan omfatte teoretisk kunnskap men også praktiske eller personlig ferdigheter. Dette materialet vil brukes til å begynne utvikling av målsetninger for studieprogrammene. Ferdigstilt innen 11/2022.
	\item Studieprogrammet gjennomfører i tillegg følgende aktiviteter.
		\begin{itemize}
			\item Samtaler med andre program som har gjennomført endringer i studieprogram for å lære av disse (minimum elsys...).
			\item Innledende gjennomgang av CDIO standardene for å vurdere hvordan de eventuelt skal brukes i den videre styringen av studieprogrammene.
			\item Samtaler med CEED for å diskutere hvordan de kan involveres i prosjektet.
			\item Gjennomgang av rapporter fra IOP og APS (eventuelt flere) om studieprogram i fysikk for å ta med eventuelle anbefalinger derfra.
		\end{itemize}
\end{itemize}

I løpet av 12/2022 gjennomføres møte med fakultet- og instituttledelse for å beslutte om man skal gå videre med planene om en gjennomgripende omlegging av studieprogrammene eller om man skal begrense seg til mindre inkrementelle justeringer. Dette vil baseres på gapsanalysen og tiltakslisten som er utviklte gjennom høsten.

%Høst Innhold- og strukturprosjekter: Overordnet struktur, beregning, eksperimentelt, læringsmiljø, vurderingsformer, undervisningsformer, mekanikk, elmag, matematikk, ønsker faggrupper.

%Revider prosjektplan. Større/mindre aktiviteter

%Ressursbehov er 20\% frikjøp i prosjekgruppen (50\% i tillegg til programledere) De andre komiteene leverer arbeider som ikke er for arbeidskrevende. 

%2023 

%Vår Overordnet struktur legges litt fastere, undervisningsformer.

%Ressursbehov. som over. Det er relativt usikkert hva ressursbehovet her er, ettersom dette er en relativt ny arbeidsmetode.

% Det kan være behov her for en gruppe som kan gå litt dypere, men det kan være vanskelig å få til på så kort varsel. Høsten 2022 bør det identifiseres evt. ressursbehov for høsten 2023. Må undersøke for raskt man kan få frigjort ressurser. 

%Høst

%Flere innholdsprosjekter

% Testing av formater underveis

