\section{MSPHYS}
\label{sect:strategy-msphys}

\subsection{Faglig innhold}

\vision{
Studenter ved MSPHYS fordyper seg i et gitt område, ofte nær forskningsfronten
}

\vision{
MSPHYS gir studetene en reel smakebit på hva forsking er, og utvikler kompetanse som er av interesse både for akademia og instituttsektor, eller videre forskerutdanning.
}

\vision{
Studenter ved MSPHYS jobber selvstendig med faglige problemstillinger og er trent i kritisk tenkning, systematisk arbeid, samt å samarbeide med forskere innen sitt fagfelt.
}

\vision{
MSPHYS gir mulighet til dyp spesialisering for å kunne bidra med kunnskap nær forskningsfronten for arbeidslivet.
}

\subsection{Organisering}

\vision{
MSPHYS definerer tydelige spesialiseringsløp.
}

\vision{
Innholdet i MSPHYS kan designes til å rette seg inn mot industriell forskning.
}
\visiontext{
Om studentene er interessert i å få en mer arbeidslivsnær utdanning, har man mulighet for å ta en ekstern MSc oppgave i samarbeid med en industriell aktør.
}

\vision{
Studenter ved MSPHYS får møter definerte tydelige spesialiseringsløp.
}

\subsubsection{Pedagogisk utforming}

\vision{
Studenter ved MSPHYS blir en del av en aktiv forskningsgruppe og lærer å jobbe og diskutere med dens medlemmer (spesielt hovedveileder).
}