\section{MTFYMA}

\vision{MTFYMA tiltrekker seg kandidater med spesiell intresse for fysikk \emph{og} matematikk og er organisert rundt disse fagdisplinene.}
\visiontext{I dette ligger det at studenter som søker seg til programmet er interessert i \emph{både} fysikk og matematikk, ikke enten eller. I tillegg presiserer det at fysikk og matematikk skal være kjernen i studiet og at det ikke skal være en overvekt av nærliggende fagområder som eksempelvis programmering, data science eller kunstig intelligens. De nevnte tema kan likevel opptre som spesialsieringer i programmet. At kjernen er fysikk og matematikk er kjernen i programmet står likevel ikke i motsetning til at MTFYMA skal være et tydelig ingeniørprogram med tydelig fokus på teknologiske anvendelser eller at arbeidslivet skal være en viktig premissleverandør for programmets innhold.}

\vision{MTFYMA høster synergier fra å være et delt ansvar mellom IFY og IMF og kandidater fra programmet står støtt i begge fagdisipliner}
\visiontext{Fysikk og matematikk er fagfelt som historisk har inspirert og hatt nytt av hverandre. Kandidater som er et solid fundament i begge displiner kan bidra til og dra nytte av denne tverrfagligheten. Likevel er det en stor adminstrativ kostnad og kulturelle barrierer i et slikt samarbeid. Programmet må derfor tydelig kunne høste syneriger i emneinnhold og læringsaktiviteter. Disse må være større enn bare summen av vært institutt for å være hensiktsmessig.}

\vision{MTFYMA er et umiskjennelig ingeniørprogram}
\visiontext{Det skal legges fokus på å kunne anvende kunnskap på industrielle problemstillinger og de ekstra ferdigheter som er nødvendig for å sette teori ut i praksis. Det legge også vekt på å utvikle personlige ferdigheter som er viktige i arbeidslivet (samarbeid, kommunikasjon m.m.). }

\vision{MTFYMA skal ha et tett og godt forhold til industri og andre avtakere av våre studenter, lytte til deres ønsker og behov og være åpen for å modernisere og tilpasse studieprogrammet etter hva som er etterspurt}
\visiontext{Dette innebærer at studieprogrammet må ha etablerte kanaler til industri, og effektive mekanismer for å justere studieprogrammet}.

\vision{MTFYMA skal utdanne kandidater som tilfører arbeidslivet kunnskaper om den til enhver tid nyeste metodikken som er tilgjengelig innen fysikk og matematikk og sentral verktøy som brukes i industrien.}

\vision{studenter ved MTFYMA opplever et studiemilljø som er støttende og motiverende, samt et studium som oppleves utviklende og utfordrende.}

\vision{MTFYMA bruker effektive vurderingsformer og undervisningsaktiviter for å nå programmets mål}

\vision{Studenter fra MTFYMA har en solid faglig basis og effektive læringsstrategier for å effektivt kunne sette seg inn i nye teknologier og metoder}