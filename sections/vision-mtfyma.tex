\section{MTFYMA}

\vision{MTFYMA tiltrekker seg kandidater med spesiell intresse for fysikk \emph{og} matematikk og er organisert rundt disse fagdisplinene.}
\visiontext{I dette ligger det at studenter som søker seg til programmet er interessert i \emph{både} fysikk og matematikk, ikke enten eller. I tillegg presiserer det at fysikk og matematikk skal være kjernen i studiet og at det ikke skal være en overvekt av nærliggende fagområder som eksempelvis programmering, data science eller kunstig intelligens. De nevnte tema kan likevel opptre som spesialsieringer i programmet. At fysikk og matematikk er kjernen i programmet står likevel ikke i motsetning til at MTFYMA skal være et tydelig ingeniørprogram med tydelig fokus på teknologiske anvendelser eller at arbeidslivet skal være en viktig premissleverandør for programmets innhold.}

\vision{MTFYMA høster synergier fra å være et delt ansvar mellom IFY og IMF og kandidater fra programmet står støtt i begge fagdisipliner}
\visiontext{Fysikk og matematikk er et fagfelt som historisk har inspirert og hatt nytte av hverandre. 
Kandidater som har et solid fundament i begge displiner kan bidra til, og dra nytte av denne tverrfagligheten.
Likevel er det en stor adminstrativ kostnad og kulturelle barrierer i et slikt samarbeid. 
Programmet må derfor tydelig kunne høste syneriger i emneinnhold og læringsaktiviteter. Disse må være større enn bare summen av vært institutt for å være hensiktsmessig.}

\vision{MTFYMA skal være et integrert studieprogram med tydelig identitet uavhengig av spesialisering med sterk vertikal og horisontal integrasjon.}
\visiontext{I dette ligger at studentene ved MTFYMA primært tenker på seg selv som MTFYMA-student, og ikke primært er knyttet til sin spesialisering.
Dette innebærer at de bør være felles emner som strekker seg gjennom store deler, kanskje hele, studiet.
Det skal være tydlige og dokumenterte koblinger mellom emner slik at de horisontalt er gjensidig støttende og vertikalt gir effektiv progresjon.}

\vision{MTFYMA har dynamiske studieretninger som tilpasses utvikling i forskningsfronten, arbeidslivet og samfunnets behov}
\visiontext{Studieprogrammet definerer tydelige studieretninger/spesialiseringer. Disse kan opprettes og legges ned i takt med nye trender i samfunnet og fagmiljøenes endrede ekspertise (men også kunne påvirke utviklingen av fagmijløene).} 

\vision{MTFYMA er et umiskjennelig ingeniørprogram}
\visiontext{Programmet skal ha et sterkt fokus på å kunne \emph{anvende} fysikk og matematikk mot teknologiske problemstillinger og inneha de ekstra ferdigheter som er nødvendige for å sette teori ut i praksis.
Dette innebærer et sterkt fokus på nødvendige ferdigheter som digitale ferdigheter, eksperimentelle ferdigheter og personlige ferdigheter for arbeidslivet, slik som samarbeid, kommunikasjon og prosjektstyring.
Alternativt kan man si at programmet ikke skal kunne mistas for et klassisk disiplinprogram.
Dette innebærer også at de skal være effektive problemløsere og må eksponeres for åpne oppgaver hvor de må kombinere matematikk, fysikk og eksperimentelle og digitale ferdigheter.}

\vision{MTFYMA har klare kompetansemål for utdanningen som gjør våre kandidater attraktive for framtidige arbeidsgivere, og tilfører arbeidslivet kunnskaper om de til enhver tid nyeste metoder.}
\visiontext{Hovedpoenget her er at programmets innhold skal være styrt av hva som gjør dem til attraktive kandidater for arbeidslivet, i motsetning til attraktive for masteroppgaver og phd studier (i de tilfeller det skulle være en motsetning).
Samtidig skal programmet bidra til å heve det teknologiske nivået på arbeidslivet ved å tilføre ny kunnskap og innsikt, med den forutsetning at en solid basis i fysikk og matematikk alltid vil være en nødvendig grunnsten i utvikling av ny teknologi.}

\vision{MTFYMA skal ha et tett og godt forhold til industri og andre avtakere av våre studenter, lytte til deres ønsker og behov og være åpen for å modernisere og tilpasse studieprogrammet etter hva som er etterspurt}
\visiontext{Dette innebærer at studieprogrammet må ha etablerte kanaler til industri, og effektive mekanismer for å justere studieprogrammet}.

\vision{studenter ved MTFYMA opplever et studiemilljø som er støttende og motiverende, samt et studium som oppleves utviklende og utfordrende.}
\visiontext{Dette ligger føringer for hvordan utdanning utformes, fysiske arealer og sosial aktiviteter. Programmet skal tiltrekke seg studenter som ønsker å utfordre seg, men samtidig får god støtte i læringsprosessen. Læringsaktiviteter og helheten i programmet skal også utformes slik at de styrker motivasjon.}

\vision{Studenter fra MTFYMA har en solid faglig basis og effektive læringsstrategier for å effektivt kunne sette seg inn i nye teknologier og metoder gjennom hele livet}
\visiontext{Studenter vil måtte tilegne seg mye ny kunnskap gjennom karrieren. En forutsetning for at dette skal skje effektivt er at den faglig basisen er tilstrekkelig solid, det er vanskelig å bygge videre på usikker kunnskap. I tillegg må studentene ha god forståelse for hvordan læring fungerer og ha innarbeidet gode læringsstrategier.}

\vision{MTFYMA bruker effektive vurderingsformer og undervisningsaktiviter for å nå programmets mål}
\visiontext{
Studieprogrammet anvender dokumentert effektive vurderings- og undervisningsmetoder. 
Spesielt i de først semstrene er dette styrt av programmet og fagmiljøet som helhet for å etablere det ønskede studiemiljø.
}

%\vision{studenter fra MTFYMA har god forståelse av UNESCOs 17 bærekraftsmål og hvordan deres kunnskap i fysikk og matematikk er essensielle for å nå målene}
%\visiontext{Studentene har oppnådd læringsmålene som beskrevet i \name{UNESCOs education for sustainable develeopment goals}, og har gjennomført prosjekter hvor de anvender sin kunnskap i fysikk og matematikk for å utorske nye løsninger på bærekraftsutfordringene.}

\vision{studenter ved MTFYMA forstår viktigheten av en god forståelse av samfunn, politikk og økonomi for å effektivt kunne bidra til verdiskapning.}
\visiontext{Dette innebærer at disse temaen må inkluderes i studiet på en måte som er meningsfulle for studentene. Disse temaene på integreres i programmet og det er neppe tilstrekkelig å la dette kun diskuteres i eksterne emner som ikke er knyttet til kjernen i programmet.}

\vision{
Studenter fra MTFYMA har bred, kvantitativ kunnskap om det vitenskapelige grunnlag for bærekraftsspørsmål, og evner å analysere disse i forhold til økonomiske og politiske virkemiddel.
}
\visiontext{
Med sin brede naturvitenskapelige basis, har studenter fra MTFYMA en unik komptanse i å kunne forstå det vitenskapelige grunnlaget som ligger til grunn for mange bærekraftsspørsmål. De kan sette seg inn i den vitenskapelige primærlitteraturen og forstå modeller, konklusjoner og eventuelle begrensninger og usikkerheter. For å kunne bidra til en bærekraftig utvikling må studentene kunne koble denne forståelsen til politiske (e.g.bærekraftsmålene) og økonomiske (e.g. EUs kvotesystem) virkemiddel.
}

%\vision{MTFYMA skal utdanne kandidater som tilfører arbeidslivet kunnskaper om den til enhver tid nyeste metodikken som er tilgjengelig innen fysikk og matematikk og sentral verktøy som brukes i industrien.}





