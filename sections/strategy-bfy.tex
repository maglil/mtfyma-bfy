\section{BFY}

\subsection{faglig innhold}

\vision{BFY tiltrekker seg studenter som er interessert i å forstå naturens grunnleggende virkemåte.}
\visiontext{
Dette er den primære kandidatprofilen programmet ønsker å tiltrekke seg. 
Sekundært kan det også være studenter som er litt usikre på hva de ønsker og hvor BFY kan være et godt springbrett for andre fagfelt.
}

\vision{BFY gir en bred og allsidig  BSc i fysikk som danner et godt grunnlag for videre studier.}
\visiontext{
Programmet skal gi studentene en brei og solid bakgrunn i fysikk samt støttefagene matematikk og informatikk.
BFY danner grunnlaget for videre studier i fysikk og andre nærliggende fagfelt.
}

\vision{BFY gir mulighet for dyp spesialisering innen et avgrenset område av fysikken som et grunnlag for videre fordypning.}
\visiontext{
Ved å starte fordypning innen et avgrenset område kan kandidater fra BFY gjennomføre en mastergrad på høyt nivå, nær forskningsfronten.
Studiet gir også erfaring med forskningsarbeid, f.eks. gjennom bachelorprosjekt, som også bidrar til å gjennomføre en mastergrad med høy kvalitet.
}

\subsection{Organisering}

\vision{BFY bidrar til internasjonalisering av fagområdet og styrkning av nasjonal kompetanse.}
\visiontext{
Gjennom å ha studieprogram som er tilpasset strukturen i European higher education area, definert gjennom Bolognaprosessen, kan BFY-kandidater reise ut for å følge masterprogram, spesielt for å studere fagområder som det ikke eksisterer ekspertise på ved NTNU eller i Norge.
Dette er viktig for nasjonal kompetanseutvikling. 
Samtidig er programmet viktig for et velfungerende masterprogram som kan rekruttere internasjonalt og gi et faglig og kulturelt rikere studiemiljø.
}

\vision{BFY gir stor frihet til å følge egne interesser.}
\visiontext{
Friheten balanseres mot basiskunnskap, det som alle fysikere \emph{må} kunne bør være obligatorisk.
Likevel er studentenes individuell søken etter forståelse en viktig verdi for programmet og dermed skal studentens frithet til å følge egne interesser stå sterkt.
Ingen vet hva fremtiden bringer og hvilken kunnskap som vil være viktig og verdifull.
Hva som skal være obligatorisk må bestemmes av fagmiljø i samråde med andre interessenter (f.eks. faglige foreninger).
}

\vision{BFY definerer tydelige studieløp rettet mot bestemte spesialiseringer.}
\visiontext{
Disse spesialiseringene er definert av valgfrie emner og er basert på IFY sine nåværende forskningsgrupper.
Dette endres dynamisk ettersom faggrupper endrer seg. 
Emnepakker kan også planlegges sammen med andre enheter, enten på NTNU eller ved andre universitet.
}

\vision{Med passende fagvalg er BFY en selvstendig og avsluttende grad.}
\visiontext{
Om studentene valger tilstrekkelige med informatikk og andre anvendte emner, bør en BFY grad kunne være en stelvstendig og avsluttende grad.  
Kandidatene skulle da kunne gå ut i næringslivet med jobber innen programmering, data analyse etc. (stillinger som BSc ingeniører ofte har idag)
}

\vision{BFY legger til rette for å ta mastergrad i andre fagområder og spesifiserer disse veiene.}
\visiontext{
Noen studenter vil kanskje oppdage at søken etter fundamental forståelse ikke var helt for dem, fant et annet spennende fagfelt eller ønske en mer anvendt utdannelse. 
Studieprogrammet prøver derfor å legge opp til at man kan ta emner som kvalifiserer for mastergrad i tilstøtende fagfelt.
Dette kan bidra til at studentene fullfører BFY-graden før de fortsetter studiene mot et annet fagfelt.
Disse muligheten spesifiseres og kommuniseres tydelig til studentene.
}

\subsection{Pedagogisk utforming}

\vision{Studenter ved BFY opplever et godt studiemiljø som pirrer den faglige nysgjerrigheten og samtidig som fostrer kollegialt samarbeid.}
\visiontext{
Dette ligger føringer for hvordan utdanning utformes, fysiske arealer og sosial aktiviteter. 
Programmet skal tiltrekke seg studenter som ønsker å utfordre seg, men samtidig får god støtte i læringsprosessen.
Læringsaktiviteter og helheten i programmet skal utformes for å styrke motivasjonen.
}