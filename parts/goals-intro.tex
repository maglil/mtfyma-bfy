Denne delen beskriver målsetningene med studieprogrammene MTFYMA, BFY og MSPHYS. En tydelig beskrivelse av målsetninger gjør de enklere å utvikle målrettede læringsaktiviteter.

I et studieprogram som strekker seg over flere år vil det være mange ulike læringsmål. For å gjøre det mulig å få oversikt, må læringsmålene organiseres i et hierarki. På det nederste nivået må læringsmålene være tilstrekkelig detaljerte og konkrete til at de enkelt kan brukes til å utforme læringsaktiviteter og vurdere om læringsmålet er oppnådd.

Denne delene er delt opp i følgende kapitler:

\begin{enumerate}
   \item \textbf{Samfunnsoppdrag} beskriver hvilket behov samfunn eller individ studieprogrammet imøtekommer. Dette vil på overordnet nivå diktere hvilke læringsmål som prioriteres.
   \item \textbf{Vision} beskriver på overordnet nivå hvordan studieprogrammet imøtekommer samfunnsoppdraget.
   \item \textbf{Læringsmål} beskriver hierarkiet av læringsmål for BFY og fellesdelen av MTFYMA.
   \item \textbf{MTFYMA studieretninger} beskriver ulike studieretninger (spesialiseringer/minors) på MTYFMA.
   \item \textbf{BFY spesialiseringer} beskriver ulike retninger man kan spesialisere seg på BFY.
   \item \textbf{MSPHYS} beskriver hierarkiet av læringsmål for MSPHYS.
\end{enumerate}