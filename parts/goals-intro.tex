Denne delen beskriver målsetningene med studieprogrammene MTFYMA, BFY og MSPHYS. 
En tydelig beskrivelse av målsetninger gjør de enklere å utvikle målrettede læringsaktiviteter.

I et studieprogram som strekker seg over flere år vil det være mange ulike læringsmål. For å gjøre det mulig å få oversikt, må læringsmålene organiseres i et hierarki. 
På det nederste nivået må læringsmålene være tilstrekkelig detaljerte og konkrete til at de enkelt kan brukes til å utforme læringsaktiviteter og vurdere om læringsmålet er oppnådd i enkeltemner.

Denne delene er delt opp i følgende kapitler:

\begin{enumerate}
	\item \textbf{Samfunnsoppdrag} beskriver hvilket behov i samfunn eller hos individ studieprogrammet imøtekommer. Dette vil på overordnet nivå diktere hvilke læringsmål som prioriteres.
	\item \textbf{Visjon} gir på overordnet nivå en beskrivelse av hva som skal karakterisere studieprogrammet for å imøtekomme samfunnsoppdraget.
	\item \textbf{Strategiske valg} gir en beskrivelse av strategisk valg som gjøres og som antas å gi størst sannsynelighet for å imøtekomme samfunnsoppdrag, visjon og læringsmålene.
	\item \textbf{Læringsmål} beskriver hierarkiet av læringsmål for studieprogrammene
\end{enumerate}